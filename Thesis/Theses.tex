
\documentclass{unswthesis}
%%%%%%%%%%%%%%%%%%%%%%%%%%%%%%%%%%%%%%%%%%%%%%%%%%%%%%%%%%%%%%%%%%%%%%%%%%%%%%%%%%%%%%%%%%%%%%%%%%%%%%%%%%%%%%%%%%%%%%%%%%%%%%%%%%%%%%%%%%%%%%%%%%%%%%%%%%%%%%%%%%%%%%%%%%%%%%%%%%%%%%%%%%%%%%%%%%%%%%%%%%%%%%%%%%%%%%%%%%%%%%%%%%%%%%%%%%%%%%%%%%%%%%%%%%%%
\usepackage{amssymb}
\usepackage{amsmath}
\usepackage{amsfonts}

\setcounter{MaxMatrixCols}{10}
%TCIDATA{OutputFilter=LATEX.DLL}
%TCIDATA{Version=5.50.0.2960}
%TCIDATA{<META NAME="SaveForMode" CONTENT="2">}
%TCIDATA{BibliographyScheme=Manual}
%TCIDATA{Created=Friday, July 08, 2016 00:30:17}
%TCIDATA{LastRevised=Monday, July 11, 2016 21:18:34}
%TCIDATA{<META NAME="GraphicsSave" CONTENT="32">}
%TCIDATA{<META NAME="Title" CONTENT="Universal Projective Quadrangle Geometry">}
%TCIDATA{<META NAME="DocumentShell" CONTENT="Theses\SW\University of New South Wales">}
%TCIDATA{CSTFile=unswthesis.cst}
%TCIDATA{<META NAME="PrintViewPercent" CONTENT="100">}

\newtheorem{theorem}{Theorem}
\newtheorem{acknowledgement}[theorem]{Acknowledgement}
\newtheorem{algorithm}[theorem]{Algorithm}
\newtheorem{axiom}[theorem]{Axiom}
\newtheorem{case}[theorem]{Case}
\newtheorem{claim}[theorem]{Claim}
\newtheorem{conclusion}[theorem]{Conclusion}
\newtheorem{condition}[theorem]{Condition}
\newtheorem{conjecture}[theorem]{Conjecture}
\newtheorem{corollary}[theorem]{Corollary}
\newtheorem{criterion}[theorem]{Criterion}
\newtheorem{definition}[theorem]{Definition}
\newtheorem{example}[theorem]{Example}
\newtheorem{exercise}[theorem]{Exercise}
\newtheorem{lemma}[theorem]{Lemma}
\newtheorem{notation}[theorem]{Notation}
\newtheorem{problem}[theorem]{Problem}
\newtheorem{proposition}[theorem]{Proposition}
\newtheorem{remark}[theorem]{Remark}
\newtheorem{solution}[theorem]{Solution}
\newtheorem{summary}[theorem]{Summary}
\newenvironment{proof}[1][Proof]{\noindent\textbf{#1.} }{\ \rule{0.5em}{0.5em}}
\input{tcilatex}
\begin{document}

\thesisschool{School Of Mathematics and Statistics}
\thesistitle{A Look into Projective Quadrangle Geometry}
\thesisauthor{Sebastian E. Blefari (z3416129)}
\thesisdegree{Bachelor of Advanced Mathematics (Honours)}
\thesisdate{}
\thesissupervisor{A/Prof.\ Norman J. Wildberger}
\thesisassessor{}
\title{Not used (see Thesis Title above), but \textsl{SWP} always adds when
using the Make Title field.}
\author{Not used (see Thesis Author above), but \textsl{SWP} always adds
when using the Make Title field.}
\maketitle
\tableofcontents
\listoffigures
\listoftables

\chapter{Bilinear Form}

Fundamental Theorem of Projective Geometry:\newline
We are able to use a transformation to send any four arbitrary non-collinear
points of a quadrangle to the four points; 
\begin{equation*}
a_{1}\equiv \lbrack 1:1:1],\;\;a_{2}\equiv \lbrack -1:-1:1],\;\;a_{3}\equiv
\lbrack 1:-1:1],\;\;a_{4}\equiv \lbrack -1:1:1].
\end{equation*}%
This transformation results in a change of Bilinear form to 
\begin{equation*}
\mathbf{A}\equiv 
\begin{bmatrix}
af & d & f \\ 
d & b & g \\ 
f & g & c%
\end{bmatrix}%
,\;\;\text{for some}\;\;a,b,c,d,e,f\in \text{Nat},
\end{equation*}%
with inverse 
\begin{equation*}
\mathbf{B}\equiv 
\begin{bmatrix}
g^{2}-bc & dc-fg & fb-dg \\ 
dc-fg & f^{2}-ac & ga-df \\ 
fb-dg & ga-df & d^{2}-ab%
\end{bmatrix}%
.
\end{equation*}%
We will refer to the quadrangle $\overline{a_{1}a_{2}a_{3}a_{4}}$ as the
standard quadrangle. The standard quadrangle has six sides $\overline{%
a_{i}a_{j}}$ for $i\neq j\in \lbrack 6]$, which determine the six lines 
\begin{align*}
L_{\{1,2\}}& \equiv a_{1}a_{2}\equiv \langle 1:-1:0\rangle , & L_{\{1,3\}}&
\equiv a_{1}a_{3}\equiv \langle 1:0:-1\rangle , & L_{\{1,4\}}& \equiv
a_{1}a_{4}\equiv \langle 0:1:-1\rangle  \\
L_{\{3,4\}}& \equiv a_{3}a_{4}\equiv \langle 1:1:0\rangle , & L_{\{2,4\}}&
\equiv a_{2}a_{4}\equiv \langle 1:0:1\rangle , & L_{\{2,3\}}& \equiv
a_{2}a_{3}\equiv \langle 0:1:1\rangle .
\end{align*}

These six lines of the respective sides let us find the diagonal triangle $%
\overline{d_{1}d_{2}d_{3}}$ of the standard quadrangle, 
\begin{eqnarray*}
d_{\alpha } &\equiv &L_{\{1,2\}}L_{\{3,4\}}\equiv \lbrack 0:0:1],\;\; \\
d_{\beta } &\equiv &L_{\{1,3\}}L_{\{2,4\}}\equiv \lbrack 0:1:0],\;\; \\
d_{\gamma } &\equiv &L_{\{1,4\}}L_{\{2,3\}}\equiv \lbrack 1:0:0].
\end{eqnarray*}%
The labeling will become more obvious below. Define 
\begin{equation*}
D\equiv abc+2fdg-ag^{2}-bf^{2}-cd^{2}.
\end{equation*}%
Then indeed we have that 
\begin{align*}
det(A)& =det%
\begin{pmatrix}
a & d & f \\ 
d & b & g \\ 
f & g & c%
\end{pmatrix}%
=D,\;\;\text{and} \\
det(B)& =det%
\begin{pmatrix}
g^{2}-bc & dc-fg & fb-dg \\ 
dc-fg & f^{2}-ac & ga-df \\ 
fb-dg & ga-df & d^{2}-ab%
\end{pmatrix}%
=-D^{2}.
\end{align*}

Also define the variable

\begin{eqnarray*}
A_{1} &=&a_{1}\cdot a_{1}\equiv a+b+c+2\left( d+f+g\right) ,~\quad
A_{2}=a_{2}\cdot a_{2}\equiv a+b+c+2\left( d-f-g\right) ,~~ \\
A_{3} &=&a_{3}\cdot a_{3}\equiv a+b+c+2\left( -d+f-g\right)
,~~A_{4}=a_{4}\cdot a_{4}\equiv a+b+c+2(-d-f+g),
\end{eqnarray*}

in an effort to simplify the expressions in the following theorem.

\begin{theorem}[Quadrangle quadrances and spreads]
Using these coordinates described above, the quadrances of the quadrangle
are 
\begin{align*}
q(a_{1},a_{2})& =4\frac{c(a+b+2d)-(f+g)^{2}}{A_{1}A_{2}}, & q(a_{3},a_{4})&
=4\frac{c(a+b-2d)-(f-g)^{2}}{A_{3}A_{4}}, \\
q(a_{1},a_{3})& =4\frac{b(a+c+2f)-(d+g)^{2}}{A_{1}A_{3}}, & q(a_{2},a_{4})&
=4\frac{b(a+c-2f)-(d-g)^{2}}{A_{2}A_{4}}, \\
q(a_{1},a_{4})& =4\frac{a(b+c+2g)-(d+f)^{2}}{A_{1}A_{4}}, & q(a_{2},a_{3})&
=4\frac{a(b+c-2g)-(d-f)^{2}}{A_{2}A_{3}}
\end{align*}%
These numbers also satisfy 
\begin{align*}
1-q(a_{1},a_{2})=\frac{(c-b-a-2d)^{2}}{A_{1}A_{2}},& & 1-q(a_{3},a_{4})& =%
\frac{(c-b-a+2d)^{2}}{A_{3}A_{4}}, \\
1-q(a_{1},a_{3})=\frac{(a-b+c+2f)^{2}}{A_{1}A_{3}},& & 1-q(a_{2},a_{4})& =%
\frac{(a-b+c-2f)^{2}}{A_{2}A_{4}}, \\
1-q(a_{1},a_{4})=\frac{(b-a+c+2g)^{2}}{A_{1}A_{4}},& & 1-q(a_{2},a_{3})& =%
\frac{(b-a+c-2g)^{2}}{A_{2}A_{3}}.
\end{align*}
\end{theorem}

\begin{proof}
Computations give you these results. $\square $
\end{proof}

\begin{theorem}
\begin{theorem}[Side Midpoints]
Suppose that $p_{1}$ and $p_{2}$ are non-null, non-perpendicular points,
forming a non-null side $\overline{p_{1}p_{2}}$. Then $\overline{p_{1}p_{2}}$
has a non-null midpoint $m$ precisely when $1-q(p_{1},p_{2})$ is a square,
and in this case there are exactly two perpendicular midpoints $m$.
\end{theorem}
\end{theorem}

\begin{proof}
\textit{Proof \ }We suppose that without loss of generality that $%
p_{1}=a_{1}\equiv \lbrack 1:1:1]$ and $p_{2}=a_{2}\equiv \lbrack -1:-1:1]$
so that by the spreads and quadrances theorem%
\begin{equation*}
1-q(p_{1},p_{2})=\frac{(c-b-a-2d)^{2}}{A_{1}A_{2}} 
\end{equation*}%
By assumption each of the variables $A_{1}$ and $A_{2}$ are nonzero. An
arbitrary point $m$ on the line $L_{\{1,2\}}\equiv \left\langle
1:-1:0\right\rangle $ has the form $m=[x-y:x-y:x+y]$, which is null
precisely when 
\begin{equation*}
(a+b+2d)(x-y)^{2}+c(x+y)^{2}+2\left( f+g\right) (x^{2}-y^{2})=0 
\end{equation*}%
by the Null point theorem. Assuming that $m$ is non-null, we compute that%
\begin{eqnarray*}
q(p_{1},m) &=&\frac{4y^{2}\left( c\left( a+b+d\right) -\left( f+g\right)
^{2}\right) }{A_{1}\left( (a+b+2d)(x-y)^{2}+c(x+y)^{2}+2\left( f+g\right)
(x^{2}-y^{2})\right) }, \\
q(p_{2},m) &=&\frac{4x^{2}\left( c\left( a+b+d\right) -\left( f+g\right)
^{2}\right) }{A_{2}\left( (a+b+2d)(x-y)^{2}+c(x+y)^{2}+2\left( f+g\right)
(x^{2}-y^{2})\right) }
\end{eqnarray*}%
\linebreak By assumption $\overline{p_{1}p_{2}}$ is non-null, so by the
Corollary to the Null points/lines theorem, $c\left( a+b+d\right) -\left(
f+g\right) ^{2}\neq 0$, and so the above expressions are equal precisely
when 
\begin{equation*}
y^{2}A_{2}=x^{2}A_{1} 
\end{equation*}%
has a solution, which occurs precisely when $1-q(p_{1},p_{2})$ is a square.
In fact if%
\begin{equation*}
\frac{1}{A_{1}A_{2}}=\sigma _{\left\{ 1,2\right\} }^{2} 
\end{equation*}%
then the two midpoints are%
\begin{equation*}
m\equiv \left[ 1\pm \sigma _{\left\{ 1,2\right\} }A_{1}:1\pm \sigma
_{\left\{ 1,2\right\} }A_{1}:1\mp \sigma _{\left\{ 1,2\right\} }A_{1}\right] 
\end{equation*}%
and they are perpendicular, since%
\begin{equation*}
m_{+}\mathbf{A}m_{-}^{T}=0 
\end{equation*}%
by computations. $\square $
\end{proof}

We denote the pair of midpoints as \textbf{opposites}. Where it follows that
the dual line $M$ of a midpoint $m$ is incident with the opposite midpoint.

\begin{theorem}[Vertex bilines]
Dual
\end{theorem}

We want to impose the conditions that all six sides $\overline{a_{i}a_{j}}$
for $i\neq j\in \lbrack 4]$ have midpoints. From the Side midpoint theorem
we know this is true precisely when $1-q(a_{i},a_{j})$ is a square. This
condition is equivalent to having numbers $\sigma _{\left\{ 1,2\right\}
},\sigma _{\left\{ 3,4\right\} },\sigma _{\left\{ 1,3\right\} },\sigma
_{\left\{ 2,4\right\} },\sigma _{\left\{ 1,4\right\} },\sigma _{\left\{
2,3\right\} }\in \text{Rat}$, satisfying the \textbf{quadratic relations}

\begin{align}
\frac{1}{A_{1}A_{2}}& =\sigma _{\left\{ 1,2\right\} }^{2},\quad \frac{1}{%
A_{1}A_{3}}=\sigma _{\left\{ 1,3\right\} }^{2},\quad \frac{1}{A_{1}A_{4}}%
=\sigma _{\left\{ 1,4\right\} }^{2},  \label{Quadratic Relations} \\
\frac{1}{A_{3}A_{4}}& =\sigma _{\left\{ 3,4\right\} }^{2},\quad \frac{1}{%
A_{2}A_{4}}=\sigma _{\left\{ 2,4\right\} }^{2},\quad \frac{1}{A_{2}A_{3}}%
=\sigma _{\left\{ 2,3\right\} }^{2}.
\end{align}%
\qquad Clearly all of $\sigma _{\left\{ 1,2\right\} },\sigma _{\left\{
3,4\right\} },\sigma _{\left\{ 1,3\right\} },\sigma _{\left\{ 2,4\right\}
},\sigma _{\left\{ 1,4\right\} },\sigma _{\left\{ 2,3\right\} }$ are
nonzero. We can further take the product of these quadratic relations in
threes, say $\frac{1}{A_{1}A_{2}}=\sigma _{\left\{ 1,2\right\} }^{2}$,$\frac{%
1}{A_{1}A_{3}}=\sigma _{\left\{ 1,3\right\} }^{2}$,$\ $and $\frac{1}{%
A_{2}A_{3}}=\sigma _{\left\{ 2,3\right\} }^{2}$, with possibly changing the
sign of any or all of \textit{sigma} values to produce the following \textbf{%
cubic relations}%
\begin{eqnarray}
\frac{1}{A_{1}A_{2}A_{3}} &=&\sigma _{\left\{ 1,2\right\} }\sigma _{\left\{
2,3\right\} }\sigma _{\left\{ 1,3\right\} },\quad \frac{1}{A_{1}A_{2}A_{4}}%
=\sigma _{\left\{ 1,2\right\} }\sigma _{\left\{ 2,4\right\} }\sigma
_{\left\{ 1,4\right\} },  \label{Cubic Relations} \\
\frac{1}{A_{1}A_{3}A_{4}} &=&\sigma _{\left\{ 1,3\right\} }\sigma _{\left\{
3,4\right\} }\sigma _{\left\{ 1,4\right\} },\quad \frac{1}{A_{2}A_{3}A_{4}}%
=\sigma _{\left\{ 2,3\right\} }\sigma _{\left\{ 3,4\right\} }\sigma
_{\left\{ 2,4\right\} }.  \notag
\end{eqnarray}%
From these relations we get%
\begin{eqnarray*}
A_{1} &=&\frac{\sigma _{\left\{ 2,3\right\} }}{\sigma _{\left\{ 1,2\right\}
}\sigma _{\left\{ 1,3\right\} }}=\frac{\sigma _{\left\{ 2,4\right\} }}{%
\sigma _{\left\{ 1,2\right\} }\sigma _{\left\{ 1,4\right\} }}=\frac{\sigma
_{\left\{ 3,4\right\} }}{\sigma _{\left\{ 1,3\right\} }\sigma _{\left\{
1,4\right\} }}, \\
A_{2} &=&\frac{\sigma _{\left\{ 1,3\right\} }}{\sigma _{\left\{ 1,2\right\}
}\sigma _{\left\{ 2,3\right\} }}=\frac{\sigma _{\left\{ 1,4\right\} }}{%
\sigma _{\left\{ 1,2\right\} }\sigma _{\left\{ 2,4\right\} }}=\frac{\sigma
_{\left\{ 3,4\right\} }}{\sigma _{\left\{ 2,3\right\} }\sigma _{\left\{
2,4\right\} }}, \\
A_{3} &=&\frac{\sigma _{\left\{ 1,2\right\} }}{\sigma _{\left\{ 1,3\right\}
}\sigma _{\left\{ 2,3\right\} }}=\frac{\sigma _{\left\{ 1,4\right\} }}{%
\sigma _{\left\{ 1,3\right\} }\sigma _{\left\{ 3,4\right\} }}=\frac{\sigma
_{\left\{ 2,4\right\} }}{\sigma _{\left\{ 2,3\right\} }\sigma _{\left\{
3,4\right\} }}, \\
A_{4} &=&\frac{\sigma _{\left\{ 1,2\right\} }}{\sigma _{\left\{ 1,4\right\}
}\sigma _{\left\{ 2,4\right\} }}=\frac{\sigma _{\left\{ 1,3\right\} }}{%
\sigma _{\left\{ 1,4\right\} }\sigma _{\left\{ 3,4\right\} }}=\frac{\sigma
_{\left\{ 2,3\right\} }}{\sigma _{\left\{ 2,4\right\} }\sigma _{\left\{
3,4\right\} }}.
\end{eqnarray*}%
Furthermore these in turn imply the relations%
\begin{equation}
\frac{\sigma _{\left\{ 2,3\right\} }}{\sigma _{\left\{ 1,3\right\} }}=\frac{%
\sigma _{\left\{ 2,4\right\} }}{\sigma _{\left\{ 1,4\right\} }},\qquad \frac{%
\sigma _{\left\{ 2,3\right\} }}{\sigma _{\left\{ 1,2\right\} }}=\frac{\sigma
_{\left\{ 3,4\right\} }}{\sigma _{\left\{ 1,4\right\} }},\qquad \frac{\sigma
_{\left\{ 2,4\right\} }}{\sigma _{\left\{ 1,2\right\} }}=\frac{\sigma
_{\left\{ 3,4\right\} }}{\sigma _{\left\{ 1,3\right\} }},
\label{Sigma Relations}
\end{equation}%
or simply%
\begin{equation*}
\sigma _{\left\{ 1,2\right\} }\sigma _{\left\{ 3,4\right\} }=\sigma
_{\left\{ 1,3\right\} }\sigma _{\left\{ 2,4\right\} }=\sigma _{\left\{
1,4\right\} }\sigma _{\left\{ 2,3\right\} }. 
\end{equation*}

All these realtions have a strong correlation with the symmetries of four
objects, namely that described realtion is the division of four objects into
three pairs of two, that is the pairing%
\begin{equation*}
\left\{ \left\{ 1,2\right\} ,\left\{ 3,4\right\} \right\} ,~\left\{ \left\{
1,3\right\} ,\left\{ 2,4\right\} \right\} ,~\left\{ \left\{ 1,4\right\}
,\left\{ 2,3\right\} \right\} 
\end{equation*}

with respect to the order given. We will from now on refer to the three
pairings aboves as \textbf{pairing }$\alpha ,\beta $ and $\gamma $
respectfully.

\section{Midpoints}

By the Side midpoint theorem for a side $\overline{a_{i}a_{j}}$ where $%
a_{i}=[v_{i}]$ and $a_{j}=[v_{j}]$ we are able to normalise $v_{i}$ and $%
v_{j}$ such that $v_{i}^{2}=v_{j}^{2}$ giving the midpoints $m_{ij}\equiv
\lbrack v_{i}+v_{j}]$ and $m_{ji}\equiv \lbrack v_{i}-v_{j}]$ of $\overline{%
a_{i}a_{j}}$, where the ordering is arbitrary. In the end of the proof of
the Side midpoints Theorem we see that the midpoints for the side $\overline{%
a_{1}a_{2}}$ are $m\equiv \left[ 1\pm \sigma _{\left\{ 1,2\right\}
}A_{1}:1\pm \sigma _{\left\{ 1,2\right\} }A_{1}:1\mp \sigma _{\left\{
1,2\right\} }A_{1}\right] $, but from above these can be rewritten as 
\begin{equation*}
m\equiv \left[ \sigma _{\left\{ 1,3\right\} }\pm \sigma _{\left\{
2,3\right\} }:\sigma _{\left\{ 1,3\right\} }\pm \sigma _{\left\{ 2,3\right\}
}:\sigma _{\left\{ 1,3\right\} }\mp \sigma _{\left\{ 2,3\right\} }\right] =%
\left[ \sigma _{\left\{ 1,4\right\} }\pm \sigma _{\left\{ 2,4\right\}
}:\sigma _{\left\{ 1,4\right\} }\pm \sigma _{\left\{ 2,4\right\} }:\sigma
_{\left\{ 1,4\right\} }\mp \sigma _{\left\{ 2,4\right\} }\right] , 
\end{equation*}%
or some other combination with respect to the $\alpha $ \textit{sigma
relations}.

\begin{theorem}[Quadrangle Midpoints]
The side midpoints of the quadrangle $\overline{a_{1}a_{2}a_{3}a_{4}}$ have
the form:

Midpoints of the side $\overline{a_{1}a_{2}}$;%
\begin{eqnarray*}
m_{12} &\equiv &\left[ 1-\sigma _{\left\{ 1,2\right\} }A_{1}:1-\sigma
_{\left\{ 1,2\right\} }A_{1}:1+\sigma _{\left\{ 1,2\right\} }A_{1}\right]  \\
&=&\left[ \sigma _{\left\{ 1,3\right\} }-\sigma _{\left\{ 2,3\right\}
}:\sigma _{\left\{ 1,3\right\} }-\sigma _{\left\{ 2,3\right\} }:\sigma
_{\left\{ 1,3\right\} }+\sigma _{\left\{ 2,3\right\} }\right]  \\
&=&\left[ \sigma _{\left\{ 1,4\right\} }-\sigma _{\left\{ 2,4\right\}
}:\sigma _{\left\{ 1,4\right\} }-\sigma _{\left\{ 2,4\right\} }:\sigma
_{\left\{ 1,4\right\} }+\sigma _{\left\{ 2,4\right\} }\right] , \\
m_{21} &\equiv &\left[ 1+\sigma _{\left\{ 1,2\right\} }A_{1}:1+\sigma
_{\left\{ 1,2\right\} }A_{1}:1-\sigma _{\left\{ 1,2\right\} }A_{1}\right]  \\
&=&\left[ \sigma _{\left\{ 1,3\right\} }+\sigma _{\left\{ 2,3\right\}
}:\sigma _{\left\{ 1,3\right\} }+\sigma _{\left\{ 2,3\right\} }:\sigma
_{\left\{ 1,3\right\} }-\sigma _{\left\{ 2,3\right\} }\right]  \\
&=&\left[ \sigma _{\left\{ 1,4\right\} }+\sigma _{\left\{ 2,4\right\}
}:\sigma _{\left\{ 1,4\right\} }+\sigma _{\left\{ 2,4\right\} }:\sigma
_{\left\{ 1,4\right\} }-\sigma _{\left\{ 2,4\right\} }\right] ,
\end{eqnarray*}%
Midpoints of the side $\overline{a_{3}a_{4}}$;%
\begin{eqnarray*}
m_{34} &\equiv &\left[ 1-\sigma _{\left\{ 3,4\right\} }A_{3}:\sigma
_{\left\{ 3,4\right\} }A_{3}-1:1+\sigma _{\left\{ 3,4\right\} }A_{3}\right] 
\\
&=&\left[ \sigma _{\left\{ 1,3\right\} }-\sigma _{\left\{ 1,4\right\}
}:\sigma _{\left\{ 1,4\right\} }-\sigma _{\left\{ 1,3\right\} }:\sigma
_{\left\{ 1,3\right\} }+\sigma _{\left\{ 1,4\right\} }\right]  \\
&=&\left[ \sigma _{\left\{ 2,3\right\} }-\sigma _{\left\{ 2,4\right\}
}:\sigma _{\left\{ 2,4\right\} }-\sigma _{\left\{ 2,3\right\} }:\sigma
_{\left\{ 2,3\right\} }+\sigma _{\left\{ 2,4\right\} }\right] , \\
m_{43} &\equiv &\left[ 1+\sigma _{\left\{ 3,4\right\} }A_{3}:-1-\sigma
_{\left\{ 3,4\right\} }A_{3}:1-\sigma _{\left\{ 3,4\right\} }A_{3}\right]  \\
&=&\left[ \sigma _{\left\{ 1,3\right\} }+\sigma _{\left\{ 1,4\right\}
}:-\sigma _{\left\{ 1,3\right\} }-\sigma _{\left\{ 1,4\right\} }:\sigma
_{\left\{ 1,3\right\} }-\sigma _{\left\{ 1,4\right\} }\right]  \\
&=&\left[ \sigma _{\left\{ 2,3\right\} }+\sigma _{\left\{ 2,4\right\}
}:-\sigma _{\left\{ 2,3\right\} }-\sigma _{\left\{ 2,4\right\} }:\sigma
_{\left\{ 2,3\right\} }-\sigma _{\left\{ 2,4\right\} }\right] ,
\end{eqnarray*}%
Midpoints of the side $\overline{a_{1}a_{3}}$;%
\begin{eqnarray*}
m_{13} &\equiv &\left[ 1+\sigma _{\left\{ 1,3\right\} }A_{1}:1-\sigma
_{\left\{ 1,3\right\} }A_{1}:1+\sigma _{\left\{ 1,3\right\} }A_{1}\right]  \\
&=&\left[ \sigma _{\left\{ 1,2\right\} }+\sigma _{\left\{ 2,3\right\}
}:\sigma _{\left\{ 1,2\right\} }-\sigma _{\left\{ 2,3\right\} }:\sigma
_{\left\{ 1,2\right\} }+\sigma _{\left\{ 2,3\right\} }\right]  \\
&=&\left[ \sigma _{\left\{ 1,4\right\} }+\sigma _{\left\{ 3,4\right\}
}:\sigma _{\left\{ 1,4\right\} }-\sigma _{\left\{ 3,4\right\} }:\sigma
_{\left\{ 1,4\right\} }+\sigma _{\left\{ 3,4\right\} }\right]  \\
m_{31} &\equiv &\left[ 1-\sigma _{\left\{ 1,3\right\} }A_{1}:1+\sigma
_{\left\{ 1,3\right\} }A_{1}:1-\sigma _{\left\{ 1,3\right\} }A_{1}\right]  \\
&=&\left[ \sigma _{\left\{ 1,2\right\} }-\sigma _{\left\{ 2,3\right\}
}:\sigma _{\left\{ 1,2\right\} }+\sigma _{\left\{ 2,3\right\} }:\sigma
_{\left\{ 1,2\right\} }-\sigma _{\left\{ 2,3\right\} }\right]  \\
&=&\left[ \sigma _{\left\{ 1,4\right\} }-\sigma _{\left\{ 3,4\right\}
}:\sigma _{\left\{ 1,4\right\} }+\sigma _{\left\{ 3,4\right\} }:\sigma
_{\left\{ 1,4\right\} }-\sigma _{\left\{ 3,4\right\} }\right] 
\end{eqnarray*}%
Midpoints of the side $\overline{a_{2}a_{4}}$;%
\begin{eqnarray*}
m_{24} &\equiv &\left[ -1-\sigma _{\left\{ 2,4\right\} }A_{2}:\sigma
_{\left\{ 2,4\right\} }A_{2}-1:1+\sigma _{\left\{ 2,4\right\} }A_{2}\right] 
\\
&=&\left[ -\sigma _{\left\{ 1,2\right\} }-\sigma _{\left\{ 1,4\right\}
}:\sigma _{\left\{ 1,4\right\} }-\sigma _{\left\{ 1,2\right\} }:\sigma
_{\left\{ 1,2\right\} }+\sigma _{\left\{ 1,4\right\} }\right]  \\
&=&\left[ -\sigma _{\left\{ 2,3\right\} }-\sigma _{\left\{ 3,4\right\}
}:\sigma _{\left\{ 3,4\right\} }-\sigma _{\left\{ 2,3\right\} }:\sigma
_{\left\{ 2,3\right\} }+\sigma _{\left\{ 3,4\right\} }\right]  \\
m_{42} &\equiv &\left[ 1-\sigma _{\left\{ 2,4\right\} }A_{2}:1+\sigma
_{\left\{ 2,4\right\} }A_{2}:\sigma _{\left\{ 2,4\right\} }A_{2}-1\right]  \\
&=&\left[ \sigma _{\left\{ 1,2\right\} }-\sigma _{\left\{ 1,4\right\}
}:\sigma _{\left\{ 1,4\right\} }+\sigma _{\left\{ 1,2\right\} }:\sigma
_{\left\{ 1,4\right\} }-\sigma _{\left\{ 1,2\right\} }\right]  \\
&=&\left[ \sigma _{\left\{ 2,3\right\} }-\sigma _{\left\{ 3,4\right\}
}:\sigma _{\left\{ 3,4\right\} }+\sigma _{\left\{ 2,3\right\} }:\sigma
_{\left\{ 3,4\right\} }-\sigma _{\left\{ 2,3\right\} }\right] 
\end{eqnarray*}%
Midpoints of the side $\overline{a_{1}a_{4}}$;%
\begin{eqnarray*}
m_{14} &\equiv &\left[ 1-\sigma _{\left\{ 1,4\right\} }A_{1}:1+\sigma
_{\left\{ 1,4\right\} }A_{1}:1+\sigma _{\left\{ 1,4\right\} }A_{1}\right]  \\
&=&\left[ \sigma _{\left\{ 1,4\right\} }-\sigma _{\left\{ 2,4\right\}
}:\sigma _{\left\{ 1,4\right\} }+\sigma _{\left\{ 2,4\right\} }:\sigma
_{\left\{ 1,4\right\} }+\sigma _{\left\{ 2,4\right\} }\right]  \\
&=&\left[ \sigma _{\left\{ 1,3\right\} }-\sigma _{\left\{ 3,4\right\}
}:\sigma _{\left\{ 1,3\right\} }+\sigma _{\left\{ 3,4\right\} }:\sigma
_{\left\{ 1,3\right\} }+\sigma _{\left\{ 3,4\right\} }\right]  \\
m_{41} &\equiv &\left[ 1+\sigma _{\left\{ 1,4\right\} }A_{1}:1-\sigma
_{\left\{ 1,4\right\} }A_{1}:1-\sigma _{\left\{ 1,4\right\} }A_{1}\right]  \\
&=&\left[ \sigma _{\left\{ 1,4\right\} }+\sigma _{\left\{ 2,4\right\}
}:\sigma _{\left\{ 1,4\right\} }-\sigma _{\left\{ 2,4\right\} }:\sigma
_{\left\{ 1,4\right\} }-\sigma _{\left\{ 2,4\right\} }\right]  \\
&=&\left[ \sigma _{\left\{ 1,3\right\} }+\sigma _{\left\{ 3,4\right\}
}:\sigma _{\left\{ 1,3\right\} }-\sigma _{\left\{ 3,4\right\} }:\sigma
_{\left\{ 1,3\right\} }-\sigma _{\left\{ 3,4\right\} }\right] 
\end{eqnarray*}%
Midpoints of the side $\overline{a_{2}a_{3}}$; 
\begin{eqnarray*}
m_{23} &\equiv &\left[ \sigma _{\left\{ 2,3\right\} }A_{2}-1:-1-\sigma
_{\left\{ 2,3\right\} }A_{2}:1+\sigma _{\left\{ 2,3\right\} }A_{2}\right]  \\
&=&\left[ \sigma _{\left\{ 1,3\right\} }-\sigma _{\left\{ 2,3\right\}
}:-\sigma _{\left\{ 2,3\right\} }-\sigma _{\left\{ 1,3\right\} }:\sigma
_{\left\{ 2,3\right\} }+\sigma _{\left\{ 1,3\right\} }\right]  \\
&=&\left[ \sigma _{\left\{ 3,4\right\} }-\sigma _{\left\{ 2,4\right\}
}:-\sigma _{\left\{ 2,4\right\} }-\sigma _{\left\{ 3,4\right\} }:\sigma
_{\left\{ 2,4\right\} }+\sigma _{\left\{ 3,4\right\} }\right]  \\
m_{32} &\equiv &\left[ 1+\sigma _{\left\{ 2,3\right\} }A_{2}:1-\sigma
_{\left\{ 2,3\right\} }A_{2}:\sigma _{\left\{ 2,3\right\} }A_{2}-1\right]  \\
&=&\left[ \sigma _{\left\{ 2,3\right\} }+\sigma _{\left\{ 1,3\right\}
}:\sigma _{\left\{ 2,3\right\} }-\sigma _{\left\{ 1,3\right\} }:\sigma
_{\left\{ 1,3\right\} }-\sigma _{\left\{ 2,3\right\} }\right]  \\
&=&\left[ \sigma _{\left\{ 2,4\right\} }+\sigma _{\left\{ 3,4\right\}
}:\sigma _{\left\{ 2,4\right\} }-\sigma _{\left\{ 3,4\right\} }:\sigma
_{\left\{ 3,4\right\} }-\sigma _{\left\{ 2,4\right\} }\right] .
\end{eqnarray*}

\begin{proof}
This is shown through computations and then careful use of the identities
described above.$\ $
\end{proof}
\end{theorem}

These side midpoints have corresponding \textit{side midlines}, which are
precisely the duals to each side midpoint. That is they are given by the
matrix multiplications $M_{ij}\equiv \mathbf{A}m_{ij}^{T},$ which highlights
the opposite relations aluuded to above.

A \textbf{subtriangle} is one of the natural divisons of the quadrangle $%
\overline{a_{1}a_{2}a_{3}a_{4}}$ into distinct triangles triangles $%
\triangle _{4}\equiv \overline{a_{1}a_{2}a_{3}},\triangle _{3}\equiv 
\overline{a_{1}a_{2}a_{4}},\triangle _{2}\equiv \overline{a_{1}a_{3}a_{4}}$
and $\triangle _{1}\equiv \overline{a_{2}a_{3}a_{4}}$.

\bigskip

\subsection{Circumlines and Circumcenters}

\begin{theorem}[Circumlines and Circumcenters]
Midpoints $m_{ij}$ for $i\neq j\in \left\{ 1,2,3\right\} $ of the
subtriangle $\triangle _{4}$ are collinear three at a time, lying on four
distinct \textbf{Circumlines} $C_{1}^{4},C_{2}^{4},C_{2}^{4},$ and $%
C_{3}^{4}.$ Midlines $M_{ij}$ for $i\neq j\in \left\{ 1,2,3\right\} $ of the
subtriangle $\triangle _{4}$ are concurrent three at a time, meeting at four
distinct \textbf{Circumcenters} $c_{1}^{4},c_{2}^{4},c_{3}^{4}$ and $%
c_{4}^{4}$\textbf{.}
\end{theorem}

\begin{proof}
The following triples of midpoints $m_{ij}$ for $i\neq j\in \left\{
1,2,3\right\} $ are colinear:

$C_{1}^{4}\equiv \left\langle \sigma _{\left\{ 1,2\right\} }-\sigma
_{\left\{ 1,3\right\} }:\sigma _{\left\{ 2,3\right\} }-\sigma _{\left\{
1,2\right\} }:\sigma _{\left\{ 1,3\right\} }+\sigma _{\left\{ 2,3\right\}
}\right\rangle $ through,%
\begin{eqnarray*}
m_{21} &\equiv &\left[ \sigma _{\left\{ 1,3\right\} }+\sigma _{\left\{
2,3\right\} }:\sigma _{\left\{ 1,3\right\} }+\sigma _{\left\{ 2,3\right\}
}:\sigma _{\left\{ 1,3\right\} }-\sigma _{\left\{ 2,3\right\} }\right] , \\
m_{31} &\equiv &\left[ \sigma _{\left\{ 1,2\right\} }-\sigma _{\left\{
2,3\right\} }:\sigma _{\left\{ 1,2\right\} }+\sigma _{\left\{ 2,3\right\}
}:\sigma _{\left\{ 1,2\right\} }-\sigma _{\left\{ 2,3\right\} }\right] , \\
m_{32} &\equiv &\left[ \sigma _{\left\{ 2,3\right\} }+\sigma _{\left\{
1,3\right\} }:\sigma _{\left\{ 2,3\right\} }-\sigma _{\left\{ 1,3\right\}
}:\sigma _{\left\{ 1,3\right\} }-\sigma _{\left\{ 2,3\right\} }\right] ,
\end{eqnarray*}%
$C_{2}^{4}\equiv \left\langle \sigma _{\left\{ 1,2\right\} }+\sigma
_{\left\{ 1,3\right\} }:-\sigma _{\left\{ 1,2\right\} }-\sigma _{\left\{
2,3\right\} }:-\sigma _{\left\{ 1,3\right\} }-\sigma _{\left\{ 2,3\right\}
}\right\rangle $ through,%
\begin{eqnarray*}
m_{21} &\equiv &\left[ \sigma _{\left\{ 1,3\right\} }+\sigma _{\left\{
2,3\right\} }:\sigma _{\left\{ 1,3\right\} }+\sigma _{\left\{ 2,3\right\}
}:\sigma _{\left\{ 1,3\right\} }-\sigma _{\left\{ 2,3\right\} }\right] , \\
m_{13} &\equiv &\left[ \sigma _{\left\{ 1,2\right\} }+\sigma _{\left\{
2,3\right\} }:\sigma _{\left\{ 1,2\right\} }-\sigma _{\left\{ 2,3\right\}
}:\sigma _{\left\{ 1,2\right\} }+\sigma _{\left\{ 2,3\right\} }\right] , \\
m_{23} &\equiv &\left[ \sigma _{\left\{ 1,3\right\} }-\sigma _{\left\{
2,3\right\} }:-\sigma _{\left\{ 2,3\right\} }-\sigma _{\left\{ 1,3\right\}
}:\sigma _{\left\{ 2,3\right\} }+\sigma _{\left\{ 1,3\right\} }\right] ,
\end{eqnarray*}%
$C_{3}^{4}\equiv \left\langle -\sigma _{\left\{ 1,2\right\} }-\sigma
_{\left\{ 1,3\right\} }:\sigma _{\left\{ 1,2\right\} }-\sigma _{\left\{
2,3\right\} }:\sigma _{\left\{ 1,3\right\} }-\sigma _{\left\{ 2,3\right\}
}\right\rangle $ through,%
\begin{eqnarray*}
m_{12} &\equiv &\left[ \sigma _{\left\{ 1,3\right\} }-\sigma _{\left\{
2,3\right\} }:\sigma _{\left\{ 1,3\right\} }-\sigma _{\left\{ 2,3\right\}
}:\sigma _{\left\{ 1,3\right\} }+\sigma _{\left\{ 2,3\right\} }\right] , \\
m_{31} &\equiv &\left[ \sigma _{\left\{ 1,2\right\} }-\sigma _{\left\{
2,3\right\} }:\sigma _{\left\{ 1,2\right\} }+\sigma _{\left\{ 2,3\right\}
}:\sigma _{\left\{ 1,2\right\} }-\sigma _{\left\{ 2,3\right\} }\right] , \\
m_{23} &\equiv &\left[ \sigma _{\left\{ 1,3\right\} }-\sigma _{\left\{
2,3\right\} }:-\sigma _{\left\{ 2,3\right\} }-\sigma _{\left\{ 1,3\right\}
}:\sigma _{\left\{ 2,3\right\} }+\sigma _{\left\{ 1,3\right\} }\right] ,
\end{eqnarray*}%
$C_{4}^{4}\equiv \left\langle \sigma _{\left\{ 1,3\right\} }-\sigma
_{\left\{ 1,2\right\} }:\sigma _{\left\{ 1,2\right\} }+\sigma _{\left\{
2,3\right\} }:\sigma _{\left\{ 2,3\right\} }-\sigma _{\left\{ 1,3\right\}
}\right\rangle $ through,%
\begin{eqnarray*}
m_{12} &\equiv &\left[ \sigma _{\left\{ 1,3\right\} }-\sigma _{\left\{
2,3\right\} }:\sigma _{\left\{ 1,3\right\} }-\sigma _{\left\{ 2,3\right\}
}:\sigma _{\left\{ 1,3\right\} }+\sigma _{\left\{ 2,3\right\} }\right] , \\
m_{13} &\equiv &\left[ \sigma _{\left\{ 1,2\right\} }+\sigma _{\left\{
2,3\right\} }:\sigma _{\left\{ 1,2\right\} }-\sigma _{\left\{ 2,3\right\}
}:\sigma _{\left\{ 1,2\right\} }+\sigma _{\left\{ 2,3\right\} }\right] , \\
m_{32} &\equiv &\left[ \sigma _{\left\{ 2,3\right\} }+\sigma _{\left\{
1,3\right\} }:\sigma _{\left\{ 2,3\right\} }-\sigma _{\left\{ 1,3\right\}
}:\sigma _{\left\{ 1,3\right\} }-\sigma _{\left\{ 2,3\right\} }\right] .
\end{eqnarray*}%
This is checked by computing%
\begin{eqnarray*}
0 &=&\det \left( 
\begin{array}{ccc}
\sigma _{\left\{ 1,3\right\} }+\sigma _{\left\{ 2,3\right\} } & \sigma
_{\left\{ 1,3\right\} }+\sigma _{\left\{ 2,3\right\} } & \sigma _{\left\{
1,3\right\} }-\sigma _{\left\{ 2,3\right\} } \\ 
\sigma _{\left\{ 1,2\right\} }-\sigma _{\left\{ 2,3\right\} } & \sigma
_{\left\{ 1,2\right\} }+\sigma _{\left\{ 2,3\right\} } & \sigma _{\left\{
1,2\right\} }-\sigma _{\left\{ 2,3\right\} } \\ 
\sigma _{\left\{ 2,3\right\} }+\sigma _{\left\{ 1,3\right\} } & \sigma
_{\left\{ 2,3\right\} }-\sigma _{\left\{ 1,3\right\} } & \sigma _{\left\{
1,3\right\} }-\sigma _{\left\{ 2,3\right\} }%
\end{array}%
\right) \\
&=&\det \left( 
\begin{array}{ccc}
\sigma _{\left\{ 1,3\right\} }+\sigma _{\left\{ 2,3\right\} } & \sigma
_{\left\{ 1,3\right\} }+\sigma _{\left\{ 2,3\right\} } & \sigma _{\left\{
1,3\right\} }-\sigma _{\left\{ 2,3\right\} } \\ 
\sigma _{\left\{ 1,2\right\} }+\sigma _{\left\{ 2,3\right\} } & \sigma
_{\left\{ 1,2\right\} }-\sigma _{\left\{ 2,3\right\} } & \sigma _{\left\{
1,2\right\} }+\sigma _{\left\{ 2,3\right\} } \\ 
\sigma _{\left\{ 1,3\right\} }-\sigma _{\left\{ 2,3\right\} } & -\sigma
_{\left\{ 2,3\right\} }-\sigma _{\left\{ 1,3\right\} } & \sigma _{\left\{
2,3\right\} }+\sigma _{\left\{ 1,3\right\} }%
\end{array}%
\right) \\
&=&\det \left( 
\begin{array}{ccc}
\sigma _{\left\{ 1,3\right\} }-\sigma _{\left\{ 2,3\right\} } & \sigma
_{\left\{ 1,3\right\} }-\sigma _{\left\{ 2,3\right\} } & \sigma _{\left\{
1,3\right\} }+\sigma _{\left\{ 2,3\right\} } \\ 
\sigma _{\left\{ 1,2\right\} }-\sigma _{\left\{ 2,3\right\} } & \sigma
_{\left\{ 1,2\right\} }+\sigma _{\left\{ 2,3\right\} } & \sigma _{\left\{
1,2\right\} }-\sigma _{\left\{ 2,3\right\} } \\ 
\sigma _{\left\{ 1,3\right\} }-\sigma _{\left\{ 2,3\right\} } & -\sigma
_{\left\{ 2,3\right\} }-\sigma _{\left\{ 1,3\right\} } & \sigma _{\left\{
2,3\right\} }+\sigma _{\left\{ 1,3\right\} }%
\end{array}%
\right) \\
&=&\det \left( 
\begin{array}{ccc}
\sigma _{\left\{ 1,3\right\} }-\sigma _{\left\{ 2,3\right\} } & \sigma
_{\left\{ 1,3\right\} }-\sigma _{\left\{ 2,3\right\} } & \sigma _{\left\{
1,3\right\} }+\sigma _{\left\{ 2,3\right\} } \\ 
\sigma _{\left\{ 1,2\right\} }+\sigma _{\left\{ 2,3\right\} } & \sigma
_{\left\{ 1,2\right\} }-\sigma _{\left\{ 2,3\right\} } & \sigma _{\left\{
1,2\right\} }+\sigma _{\left\{ 2,3\right\} } \\ 
\sigma _{\left\{ 2,3\right\} }+\sigma _{\left\{ 1,3\right\} } & \sigma
_{\left\{ 2,3\right\} }-\sigma _{\left\{ 1,3\right\} } & \sigma _{\left\{
1,3\right\} }-\sigma _{\left\{ 2,3\right\} }%
\end{array}%
\right) .
\end{eqnarray*}%
The corresponding meets are%
\begin{eqnarray*}
&&\left[ \sigma _{\left\{ 1,3\right\} }+\sigma _{\left\{ 2,3\right\}
}:\sigma _{\left\{ 1,3\right\} }+\sigma _{\left\{ 2,3\right\} }:\sigma
_{\left\{ 1,3\right\} }-\sigma _{\left\{ 2,3\right\} }\right] \times \left[
\sigma _{\left\{ 1,2\right\} }-\sigma _{\left\{ 2,3\right\} }:\sigma
_{\left\{ 1,2\right\} }+\sigma _{\left\{ 2,3\right\} }:\sigma _{\left\{
1,2\right\} }-\sigma _{\left\{ 2,3\right\} }\right] \\
&=&\left\langle \sigma _{\left\{ 1,2\right\} }-\sigma _{\left\{ 1,3\right\}
}:\sigma _{\left\{ 2,3\right\} }-\sigma _{\left\{ 1,2\right\} }:\sigma
_{\left\{ 1,3\right\} }+\sigma _{\left\{ 2,3\right\} }\right\rangle \equiv
C_{1}^{4},
\end{eqnarray*}%
and similarly for the other circumlines. The situation with midlines $M_{ij}$
for $i\neq j\in \left\{ 1,2,3\right\} $ is precisely dual.
\end{proof}

\bigskip

\begin{theorem}[Subtriangle Circumlines of the Quadrangle]
The Circumlines for the Subtriangles $\triangle _{3},\triangle _{2}$, and $%
\triangle _{1}$, are as follows;

The Circumlines for the subtriangle $\triangle _{3}\,:$

$C_{1}^{3}\equiv \left\langle \sigma _{\left\{ 1,2\right\} }-\sigma
_{\left\{ 2,4\right\} }:\sigma _{\left\{ 1,4\right\} }-\sigma _{\left\{
1,2\right\} }:-\sigma _{\left\{ 1,4\right\} }-\sigma _{\left\{ 2,4\right\}
}\right\rangle $ through%
\begin{eqnarray*}
m_{21} &\equiv &\left[ \sigma _{\left\{ 1,4\right\} }+\sigma _{\left\{
2,4\right\} }:\sigma _{\left\{ 1,4\right\} }+\sigma _{\left\{ 2,4\right\}
}:\sigma _{\left\{ 1,4\right\} }-\sigma _{\left\{ 2,4\right\} }\right] , \\
m_{42} &\equiv &\left[ \sigma _{\left\{ 1,2\right\} }-\sigma _{\left\{
1,4\right\} }:\sigma _{\left\{ 1,2\right\} }+\sigma _{\left\{ 1,4\right\}
}:\sigma _{\left\{ 1,4\right\} }-\sigma _{\left\{ 1,2\right\} }\right] , \\
m_{41} &\equiv &\left[ \sigma _{\left\{ 1,4\right\} }+\sigma _{\left\{
2,4\right\} }:\sigma _{\left\{ 1,4\right\} }-\sigma _{\left\{ 2,4\right\}
}:\sigma _{\left\{ 1,4\right\} }-\sigma _{\left\{ 2,4\right\} }\right] ,
\end{eqnarray*}%
$C_{2}^{3}\equiv \left\langle \sigma _{\left\{ 1,2\right\} }+\sigma
_{\left\{ 2,4\right\} }:-\sigma _{\left\{ 1,2\right\} }-\sigma _{\left\{
1,4\right\} }:\sigma _{\left\{ 1,4\right\} }+\sigma _{\left\{ 2,4\right\}
}\right\rangle $ through%
\begin{eqnarray*}
m_{21} &\equiv &\left[ \sigma _{\left\{ 1,4\right\} }+\sigma _{\left\{
2,4\right\} }:\sigma _{\left\{ 1,4\right\} }+\sigma _{\left\{ 2,4\right\}
}:\sigma _{\left\{ 1,4\right\} }-\sigma _{\left\{ 2,4\right\} }\right] , \\
m_{24} &\equiv &\left[ -\sigma _{\left\{ 1,2\right\} }-\sigma _{\left\{
1,4\right\} }:\sigma _{\left\{ 1,4\right\} }-\sigma _{\left\{ 1,2\right\}
}:\sigma _{\left\{ 1,2\right\} }+\sigma _{\left\{ 1,4\right\} }\right] , \\
m_{14} &\equiv &\left[ \sigma _{\left\{ 1,4\right\} }-\sigma _{\left\{
2,4\right\} }:\sigma _{\left\{ 1,4\right\} }+\sigma _{\left\{ 2,4\right\}
}:\sigma _{\left\{ 1,4\right\} }+\sigma _{\left\{ 2,4\right\} }\right] ,
\end{eqnarray*}%
$C_{3}^{3}\equiv \left\langle \sigma _{\left\{ 1,2\right\} }+\sigma
_{\left\{ 2,4\right\} }:\sigma _{\left\{ 1,4\right\} }-\sigma _{\left\{
1,2\right\} }:\sigma _{\left\{ 2,4\right\} }-\sigma _{\left\{ 1,4\right\}
}\right\rangle $ through%
\begin{eqnarray*}
m_{12} &\equiv &\left[ \sigma _{\left\{ 1,4\right\} }-\sigma _{\left\{
2,4\right\} }:\sigma _{\left\{ 1,4\right\} }-\sigma _{\left\{ 2,4\right\}
}:\sigma _{\left\{ 1,4\right\} }+\sigma _{\left\{ 2,4\right\} }\right] , \\
m_{42} &\equiv &\left[ \sigma _{\left\{ 1,2\right\} }-\sigma _{\left\{
1,4\right\} }:\sigma _{\left\{ 1,2\right\} }+\sigma _{\left\{ 1,4\right\}
}:\sigma _{\left\{ 1,4\right\} }-\sigma _{\left\{ 1,2\right\} }\right] , \\
m_{14} &\equiv &\left[ \sigma _{\left\{ 1,4\right\} }-\sigma _{\left\{
2,4\right\} }:\sigma _{\left\{ 1,4\right\} }+\sigma _{\left\{ 2,4\right\}
}:\sigma _{\left\{ 1,4\right\} }+\sigma _{\left\{ 2,4\right\} }\right] ,
\end{eqnarray*}%
$C_{4}^{3}\equiv \left\langle \sigma _{\left\{ 1,2\right\} }-\sigma
_{\left\{ 2,4\right\} }:-\sigma _{\left\{ 1,2\right\} }-\sigma _{\left\{
1,4\right\} }:\sigma _{\left\{ 1,4\right\} }-\sigma _{\left\{ 2,4\right\}
}\right\rangle $ through%
\begin{eqnarray*}
m_{12} &\equiv &\left[ \sigma _{\left\{ 1,4\right\} }-\sigma _{\left\{
2,4\right\} }:\sigma _{\left\{ 1,4\right\} }-\sigma _{\left\{ 2,4\right\}
}:\sigma _{\left\{ 1,4\right\} }+\sigma _{\left\{ 2,4\right\} }\right] , \\
m_{24} &\equiv &\left[ -\sigma _{\left\{ 1,2\right\} }-\sigma _{\left\{
1,4\right\} }:\sigma _{\left\{ 1,4\right\} }-\sigma _{\left\{ 1,2\right\}
}:\sigma _{\left\{ 1,2\right\} }+\sigma _{\left\{ 1,4\right\} }\right] , \\
m_{41} &\equiv &\left[ \sigma _{\left\{ 1,4\right\} }+\sigma _{\left\{
2,4\right\} }:\sigma _{\left\{ 1,4\right\} }-\sigma _{\left\{ 2,4\right\}
}:\sigma _{\left\{ 1,4\right\} }-\sigma _{\left\{ 2,4\right\} }\right] .
\end{eqnarray*}%
The Circumlines for the subtriangle $\triangle _{2}\,:$%
\begin{eqnarray*}
C_{1}^{2} &\equiv &\left\langle \sigma _{\left\{ 3,4\right\} }-\sigma
_{\left\{ 1,3\right\} }:\sigma _{\left\{ 3,4\right\} }-\sigma _{\left\{
1,4\right\} }:\sigma _{\left\{ 1,3\right\} }+\sigma _{\left\{ 1,4\right\}
}\right\rangle ,\text{ through }m_{43},m_{31},m_{41}, \\
C_{2}^{2} &\equiv &\left\langle \sigma _{\left\{ 1,3\right\} }+\sigma
_{\left\{ 3,4\right\} }:\sigma _{\left\{ 1,4\right\} }+\sigma _{\left\{
3,4\right\} }:-\sigma _{\left\{ 1,3\right\} }-\sigma _{\left\{ 1,4\right\}
}\right\rangle ,\text{ through }m_{43},m_{13},m_{14}, \\
C_{3}^{2} &\equiv &\left\langle \sigma _{\left\{ 1,3\right\} }+\sigma
_{\left\{ 3,4\right\} }:\sigma _{\left\{ 3,4\right\} }-\sigma _{\left\{
1,4\right\} }:\sigma _{\left\{ 1,4\right\} }-\sigma _{\left\{ 1,3\right\}
}\right\rangle ,\text{ through }m_{34},m_{31},m_{14}, \\
C_{4}^{2} &\equiv &\left\langle \sigma _{\left\{ 3,4\right\} }-\sigma
_{\left\{ 1,3\right\} }:\sigma _{\left\{ 1,4\right\} }+\sigma _{\left\{
3,4\right\} }:\sigma _{\left\{ 1,3\right\} }-\sigma _{\left\{ 1,4\right\}
}\right\rangle ,\text{ through }m_{34},m_{13},m_{41},
\end{eqnarray*}%
The Circumlines for the subtriangle $\triangle _{2}\,:$%
\begin{eqnarray*}
C_{1}^{1} &\equiv &\left\langle \sigma _{\left\{ 2,4\right\} }-\sigma
_{\left\{ 3,4\right\} }:\sigma _{\left\{ 2,3\right\} }-\sigma _{\left\{
3,4\right\} }:\sigma _{\left\{ 2,3\right\} }+\sigma _{\left\{ 2,4\right\}
}\right\rangle ,\text{ through }m_{43},m_{32},m_{42} \\
C_{2}^{1} &\equiv &\left\langle \sigma _{\left\{ 2,4\right\} }+\sigma
_{\left\{ 3,4\right\} }:\sigma _{\left\{ 2,3\right\} }+\sigma _{\left\{
3,4\right\} }:\sigma _{\left\{ 2,3\right\} }+\sigma _{\left\{ 2,4\right\}
}\right\rangle ,\text{ through }m_{43},m_{23},m_{24}, \\
C_{3}^{1} &\equiv &\left\langle \sigma _{\left\{ 2,4\right\} }+\sigma
_{\left\{ 3,4\right\} }:\sigma _{\left\{ 3,4\right\} }-\sigma _{\left\{
2,3\right\} }:\sigma _{\left\{ 2,4\right\} }-\sigma _{\left\{ 2,3\right\}
}\right\rangle ,\text{ through }m_{34},m_{23},m_{42}, \\
C_{4}^{1} &\equiv &\left\langle \sigma _{\left\{ 3,4\right\} }-\sigma
_{\left\{ 2,4\right\} }:\sigma _{\left\{ 2,3\right\} }+\sigma _{\left\{
3,4\right\} }:\sigma _{\left\{ 2,3\right\} }-\sigma _{\left\{ 2,4\right\}
}\right\rangle ,\text{ through }m_{34},m_{32},m_{24}.
\end{eqnarray*}

\begin{proof}
The computations are analogous to those in the Circumline and Circumcenter
Theorems.
\end{proof}
\end{theorem}

Though the labeling may seem arbitrary infact it is a little more subtle
than this. Now for any triangle there are four circumlines which are the
joins of three distinct midpoints. Therefore by counting we see that each
midpoint is incident with two distinct circumlines.

For the quadrangle $\square $ any two subtriangles share exactly one side.
Say subtriangle $\triangle _{i}$ and $\triangle _{i}$ for $i\neq j\in
\left\{ 1,2,3,4\right\} $ share the side $\overline{a_{\ell }a_{k}}$ where $%
\ell \neq k\in \left\{ 1,2,3,4\right\} \setminus \left\{ i,j\right\} $, that
is for subtriangles $\triangle _{1}$ and $\triangle _{2}$ they share the
side $\overline{a_{3}a_{4}}$, and so on. Thus each midpoint $m$ of the
quadrangle $\square $ is associated to four circumlines $C$, two distinct
circumlines from each of the subtriangles that share the side corresponding
to the midpoint $m$. Therefore there is a type of pairing of pairs of
circumlines from different subtriangles $P_{ij}\equiv \left\{
C_{r}^{i},C_{s}^{i},C_{\ell }^{j},C_{k}^{j}\right\} $ for some $r\neq s,\ell
\neq k\in \left\{ 1,2,3,4\right\} $, for each midpoint $m_{ij}$. The set $%
P_{ij}$ will be called a \textbf{pairing} of circumlines.

We will say that circumlines $C_{k}^{i}$ and $C_{\ell }^{j}$ are \textbf{%
(midpoint) neighbours} if one of the pairings $P_{ij}$ and $P_{ji}$\ induced
from the midpoints $m_{ij}$ and $m_{ji}$ contains both $C_{k}^{i}$ and $%
C_{\ell }^{j}$.

\bigskip

After forcing a label on one set of circumlines (in this case for the
subtriangle $\triangle _{4}$) the aim is to label the rest in such a way so
that for each pairing $P_{ij}=\left\{ C_{r}^{i},C_{s}^{i},C_{\ell
}^{j},C_{k}^{j}\right\} $ we have that $\ell =r$ and $k=s$. It turns out
there are exactly two such labelings one is giving above;

\begin{equation*}
\begin{array}{cc}
P_{12}\equiv \left\{ C_{3}^{3},C_{4}^{3},C_{3}^{4},C_{4}^{4}\right\} , & 
P_{21}\equiv \left\{ C_{1}^{3},C_{2}^{3},C_{1}^{4},C_{2}^{4}\right\} , \\ 
P_{34}\equiv \left\{ C_{3}^{1},C_{4}^{1},C_{3}^{2},C_{4}^{2}\right\} , & 
P_{43}\equiv \left\{ C_{1}^{1},C_{2}^{1},C_{1}^{2},C_{2}^{2}\right\} , \\ 
P_{13}\equiv \left\{ C_{2}^{2},C_{4}^{2},C_{2}^{4},C_{4}^{4}\right\} , & 
P_{31}\equiv \left\{ C_{1}^{2},C_{3}^{2},C_{1}^{4},C_{3}^{4}\right\} , \\ 
P_{24}\equiv \left\{ C_{2}^{1},C_{4}^{1},C_{2}^{3},C_{4}^{3}\right\} , & 
P_{42}\equiv \left\{ C_{1}^{1},C_{3}^{1},C_{1}^{3},C_{3}^{3}\right\} , \\ 
P_{14}\equiv \left\{ C_{2}^{2},C_{3}^{2},C_{2}^{3},C_{3}^{3}\right\} , & 
P_{41}\equiv \left\{ C_{1}^{2},C_{4}^{2},C_{1}^{3},C_{4}^{3}\right\} , \\ 
P_{23}\equiv \left\{ C_{2}^{1},C_{3}^{1},C_{2}^{4},C_{3}^{4}\right\} , & 
P_{32}\equiv \left\{ C_{1}^{1},C_{4}^{1},C_{1}^{4},C_{4}^{4}\right\} .%
\end{array}%
\newline
\end{equation*}

Moreover if we define $\pi _{4}\equiv \left( 1\right) ,\pi _{3}\equiv \left(
12\right) \left( 34\right) ,\pi _{2}\equiv \left( 13\right) \left( 24\right) 
$ and $\pi _{1}\equiv \left( 14\right) \left( 23\right) $, then the other
labeling is giving as follows $C_{\pi _{i}\left( j\right) }^{i}$ for $i,j\in
\left\{ 1,2,3,4\right\} $.

Furthermore both of these labelings of the circumlines we get that the
elements of the sets $\mathcal{C}_{i}\equiv \left\{
C_{i}^{1},C_{i}^{2},C_{i}^{3},C_{i}^{4}\right\} $ for $i=1,2,3,4$ are
neighbours.

\bigskip

Define $\mathcal{C}^{i}\equiv \left\{
C_{1}^{i},C_{2}^{i},C_{3}^{i},C_{4}^{i}\right\} $ to be the set of
circumlines assosciated with the subtriangle $\triangle _{i},$ and $\mathcal{%
C}_{i}$ as above. Lets consider the circumlines of the the quadrangle $%
\square $ as vertices and say that two vertices share an edge precisely when
the \textit{corresponding circumlines are not midpoint neighbours}. Note
that by definition the induced graph is quadpartite with respect to the
vertex partition $\cup _{i}\mathcal{C}^{i}$. Furthermore as each $\mathcal{C}%
_{i}$ contains no neighbours, the vertex quad-partion $\cup _{i}\mathcal{C}%
_{i}$ is consistent with the previous one. Moreover as graphs they are
isomorphic with respect to the map $C_{j}^{i}\mapsto C_{i}^{j}$.

Now each vetex $C_{j}^{i}$ for some subtriangle $\triangle _{i}$ and index $%
j $, has exactly two midpoint neighbours in $\mathcal{C}^{k}$ and $\mathcal{C%
}_{k}$ for $k\neq i$ and $k\neq j$ respectfully. Thus each vertex has degree
six and so by the Handshanking lemma from Graph Theory there are exactly
fourtyeight edges and hence distinct meets of non-neighbouring circumlines.
It turns out that these meets are collinear four at a time producing twelve
distinct lines, but before we get to that some structure of the circumlines
needs to be explored.

\bigskip

Looking at the bipartite subgraph with vertex set $\mathcal{C}_{1}\cup 
\mathcal{C}_{2}$ the edge set can be worked out by exaiming the pairings $%
P_{ij}$ for $i\neq j\in \left[ 4\right] $, above. What results is the set,

\begin{equation*}
\left\{ \left( C_{1}^{1}C_{2}^{4}\right) ,\left( C_{2}^{4}C_{1}^{2}\right)
,\left( C_{1}^{2}C_{2}^{3}\right) ,\left( C_{2}^{3}C_{1}^{1}\right) ,\left(
C_{2}^{1}C_{1}^{4}\right) ,\left( C_{1}^{4}C_{2}^{2}\right) ,\left(
C_{2}^{2}C_{1}^{3}\right) ,\left( C_{1}^{3}C_{2}^{1}\right) \right\} , 
\end{equation*}

and so the edge set of the bipartite subgraph is the union of distjoint
cycles%
\begin{equation}
\mathcal{C}_{12}\equiv C_{1}^{1}C_{2}^{3}C_{1}^{2}C_{2}^{4}C_{1}^{1}\text{
and }\mathcal{C}_{21}\equiv C_{2}^{1}C_{1}^{3}C_{2}^{2}C_{1}^{4}C_{2}^{1}. 
\notag
\end{equation}

This is similarly true for the remaining bipartite subgraphs with respect to
the partitions $\mathcal{C}_{i}$,%
\begin{equation*}
\begin{array}{cc}
\mathcal{C}_{3}\cup \mathcal{C}_{4}: & \mathcal{C}_{34}\equiv
C_{3}^{1}C_{4}^{3}C_{3}^{2}C_{4}^{4}C_{3}^{1}\text{ and }\mathcal{C}%
_{43}\equiv C_{4}^{1}C_{3}^{3}C_{4}^{2}C_{3}^{4}C_{4}^{1}, \\ 
\mathcal{C}_{1}\cup \mathcal{C}_{3}: & \mathcal{C}_{13}\equiv
C_{1}^{1}C_{3}^{2}C_{1}^{3}C_{3}^{4}C_{1}^{1}\text{ and }\mathcal{C}%
_{31}\equiv C_{3}^{1}C_{1}^{2}C_{3}^{3}C_{1}^{4}C_{3}^{1}, \\ 
\mathcal{C}_{2}\cup \mathcal{C}_{4}: & \mathcal{C}_{24}\equiv
C_{2}^{1}C_{4}^{2}C_{2}^{3}C_{4}^{4}C_{2}^{1}\text{ and }\mathcal{C}%
_{42}\equiv C_{4}^{1}C_{2}^{2}C_{4}^{3}C_{2}^{4}C_{4}^{1}, \\ 
\mathcal{C}_{1}\cup \mathcal{C}_{4}: & \mathcal{C}_{14}\equiv
C_{1}^{1}C_{4}^{2}C_{1}^{4}C_{4}^{3}C_{1}^{1}\text{ and }\mathcal{C}%
_{41}\equiv C_{4}^{1}C_{1}^{2}C_{4}^{4}C_{1}^{3}C_{4}^{1}, \\ 
\mathcal{C}_{2}\cup \mathcal{C}_{3}: & \mathcal{C}_{23}\equiv
C_{2}^{1}C_{3}^{2}C_{2}^{4}C_{3}^{3}C_{2}^{1}\text{ and }\mathcal{C}%
_{32}\equiv C_{3}^{1}C_{2}^{2}C_{3}^{4}C_{2}^{3}C_{3}^{1}.%
\end{array}%
\end{equation*}

There is an anologous structure to the bipartite subgraphs with repect to
the partitions $\mathcal{C}^{i}$, and it is as follows,%
\begin{equation*}
\begin{array}{cc}
\mathcal{C}^{1}\cup \mathcal{C}^{2}: & \mathcal{C}^{12}\equiv
C_{1}^{1}C_{3}^{2}C_{2}^{1}C_{4}^{2}C_{1}^{1}\text{ and }\mathcal{C}%
^{21}\equiv C_{1}^{2}C_{3}^{1}C_{2}^{2}C_{4}^{1}C_{1}^{2}, \\ 
\mathcal{C}^{3}\cup \mathcal{C}^{4}: & \mathcal{C}^{34}\equiv
C_{1}^{3}C_{3}^{4}C_{2}^{3}C_{4}^{4}C_{1}^{3}\text{ and }\mathcal{C}%
^{43}\equiv C_{1}^{4}C_{3}^{3}C_{2}^{4}C_{4}^{3}C_{1}^{4}, \\ 
\mathcal{C}^{1}\cup \mathcal{C}^{3}: & \mathcal{C}^{13}\equiv
C_{1}^{1}C_{2}^{3}C_{3}^{1}C_{4}^{3}C_{1}^{1}\text{ and }\mathcal{C}%
^{31}\equiv C_{1}^{3}C_{2}^{1}C_{3}^{3}C_{4}^{1}C_{1}^{3}, \\ 
\mathcal{C}^{2}\cup \mathcal{C}^{4}: & \mathcal{C}^{24}\equiv
C_{1}^{2}C_{2}^{4}C_{3}^{2}C_{4}^{4}C_{1}^{2}\text{ and }\mathcal{C}%
^{42}\equiv C_{1}^{4}C_{2}^{2}C_{3}^{4}C_{4}^{2}C_{1}^{4}, \\ 
\mathcal{C}^{1}\cup \mathcal{C}^{4}: & \mathcal{C}^{14}\equiv
C_{1}^{1}C_{2}^{4}C_{4}^{1}C_{3}^{4}C_{1}^{1}\text{ and }\mathcal{C}%
^{41}\equiv C_{1}^{4}C_{2}^{1}C_{4}^{4}C_{3}^{1}C_{1}^{4}, \\ 
\mathcal{C}^{2}\cup \mathcal{C}^{3}: & \mathcal{C}^{23}\equiv
C_{1}^{2}C_{2}^{3}C_{4}^{2}C_{3}^{3}C_{1}^{2}\text{ and }\mathcal{C}%
^{32}\equiv C_{1}^{3}C_{2}^{2}C_{4}^{3}C_{3}^{2}C_{1}^{3}.%
\end{array}%
\end{equation*}

\begin{theorem}[Meets of Circumlines]
The meets of non-neighbouring circumlines are collinear four at a time,
producing twelve distinct c-lines.
\end{theorem}

\begin{proof}
Now as the graphs are isomorphic each edge, say $\left(
C_{r}^{i}C_{s}^{j}\right) $, must appear exactly two cycles, $\mathcal{C}%
_{rs}$ or $\mathcal{C}_{sr}$ and $\mathcal{C}^{ij}$ or $\mathcal{C}^{ji}$,\
one from each of the respective partitions. For example the edge
corresponding to the meet $\left( C_{1}^{1}C_{2}^{3}\right) $\ is in the
cycles $\mathcal{C}_{12}$ and $\mathcal{C}^{13}$.

Without loss of generality since meets are symetric lets assume that the
edge $\left( C_{r}^{i}C_{s}^{j}\right) $ appears in the cycles $\mathcal{C}%
_{rs}$ and $\mathcal{C}^{ij}$. There are six vertices in the union of these
two cycles $\mathcal{C}_{rs}\cup \mathcal{C}^{ij}$, and two remaining edges
when removing the vertices $C_{r}^{i},C_{s}^{j}$. In our example the six
vertices are%
\begin{equation*}
\text{ }C_{1}^{2},C_{2}^{4},C_{1}^{1},C_{2}^{3},C_{3}^{1},C_{4}^{3}\in 
\mathcal{C}_{12}\cup \mathcal{C}^{13} 
\end{equation*}%
and the two edges%
\begin{equation*}
\left( C_{1}^{2}C_{2}^{4}\right) ,\left( C_{3}^{1}C_{4}^{3}\right) \in
\left( \mathcal{C}_{12}\cup \mathcal{C}^{13}\right) \setminus \left\{
C_{1}^{1},C_{2}^{3}\right\} . 
\end{equation*}%
Now computations show us that the three meets corresponding to the edges
found above are collinear. That is the line%
\begin{equation*}
\left\langle \sigma _{\left\{ 1,2\right\} }+\sigma _{\left\{ 3,4\right\}
}:\sigma _{\left\{ 3,4\right\} }-\sigma _{\left\{ 1,2\right\} }-\sigma
_{\left\{ 1,4\right\} }-\sigma _{\left\{ 2,3\right\} }:\sigma _{\left\{
1,4\right\} }-\sigma _{\left\{ 2,3\right\} }\right\rangle 
\end{equation*}%
goes through the points%
\begin{equation*}
\left( C_{1}^{1}C_{2}^{3}\right) \ ,\left( C_{1}^{2}C_{2}^{4}\right) ,\text{
and }\left( C_{3}^{1}C_{4}^{3}\right) . 
\end{equation*}%
Now this is true for every vertex, and so there is some double counting.
Infact the triples produced by picking vertices will overlap four at a time.
So the line above is associated with the cycles $\mathcal{C}^{13},\mathcal{C}%
^{24},\mathcal{C}_{12}$ and $\mathcal{C}_{34}$\ and so is also incident with
the meet%
\begin{equation*}
\left( C_{3}^{2}C_{4}^{4}\right) . 
\end{equation*}%
That is we can group the circumlines into sets of fours which are colinear
producing twelve distinct lines%
\begin{equation*}
C_{\left\{ 14,32\right\} }^{\left\{ 13,42\right\} }\equiv \left\langle
\sigma _{\left\{ 1,2\right\} }-\sigma _{\left\{ 3,4\right\} }:\sigma
_{\left\{ 2,3\right\} }-\sigma _{\left\{ 1,2\right\} }-\sigma _{\left\{
1,4\right\} }-\sigma _{\left\{ 3,4\right\} }:\sigma _{\left\{ 1,4\right\}
}+\sigma _{\left\{ 2,3\right\} }\right\rangle 
\end{equation*}%
\begin{equation*}
\text{through }\left( C_{1}^{1}C_{4}^{3}\right) \ ,\left(
C_{3}^{1}C_{2}^{3}\right) ,\left( C_{1}^{4}C_{4}^{2}\right) ,\text{ and }%
\left( C_{3}^{4}C_{2}^{2}\right) , 
\end{equation*}%
\begin{equation*}
C_{\left\{ 21,43\right\} }^{\left\{ 31,42\right\} }\equiv \left\langle
\sigma _{\left\{ 1,2\right\} }+\sigma _{\left\{ 3,4\right\} }:\sigma
_{\left\{ 1,4\right\} }-\sigma _{\left\{ 1,2\right\} }+\sigma _{\left\{
2,3\right\} }+\sigma _{\left\{ 3,4\right\} }:\sigma _{\left\{ 2,3\right\}
}-\sigma _{\left\{ 1,4\right\} }\right\rangle 
\end{equation*}%
\begin{equation*}
\text{through }\left( C_{2}^{1}C_{1}^{3}\right) \ ,\left(
C_{4}^{1}C_{3}^{3}\right) ,\left( C_{2}^{2}C_{1}^{4}\right) ,\text{ and }%
\left( C_{4}^{2}C_{3}^{4}\right) , 
\end{equation*}%
\begin{equation*}
C_{\left\{ 41,23\right\} }^{\left\{ 31,24\right\} }\equiv \left\langle
\sigma _{\left\{ 3,4\right\} }-\sigma _{\left\{ 1,2\right\} }:\sigma
_{\left\{ 1,2\right\} }-\sigma _{\left\{ 1,4\right\} }+\sigma _{\left\{
2,3\right\} }+\sigma _{\left\{ 3,4\right\} }:\sigma _{\left\{ 1,4\right\}
}+\sigma _{\left\{ 2,3\right\} }\right\rangle 
\end{equation*}%
\begin{equation*}
\text{through }\left( C_{2}^{1}C_{3}^{3}\right) \ ,\left(
C_{4}^{1}C_{1}^{3}\right) ,\left( C_{2}^{4}C_{3}^{2}\right) ,\text{ and }%
\left( C_{4}^{4}C_{1}^{2}\right) , 
\end{equation*}%
\begin{equation*}
C_{\left\{ 13,24\right\} }^{\left\{ 12,34\right\} }\equiv \left\langle
\sigma _{\left\{ 1,3\right\} }+\sigma _{\left\{ 2,4\right\} }:\sigma
_{\left\{ 2,3\right\} }-\sigma _{\left\{ 1,4\right\} }:\sigma _{\left\{
1,4\right\} }-\sigma _{\left\{ 1,3\right\} }+\sigma _{\left\{ 2,3\right\}
}+\sigma _{\left\{ 2,4\right\} }\right\rangle 
\end{equation*}%
\begin{equation*}
\text{through }\left( C_{1}^{1}C_{3}^{2}\right) \ ,\left(
C_{2}^{1}C_{4}^{2}\right) ,\left( C_{1}^{3}C_{3}^{4}\right) ,\text{ and }%
\left( C_{2}^{3}C_{4}^{4}\right) , 
\end{equation*}%
\begin{equation*}
C_{\left\{ 14,23\right\} }^{\left\{ 12,43\right\} }\equiv \left\langle
\sigma _{\left\{ 2,4\right\} }-\sigma _{\left\{ 1,3\right\} }:\sigma
_{\left\{ 2,3\right\} }+\sigma _{\left\{ 1,4\right\} }:\sigma _{\left\{
1,3\right\} }-\sigma _{\left\{ 1,4\right\} }+\sigma _{\left\{ 2,3\right\}
}+\sigma _{\left\{ 2,4\right\} }\right\rangle 
\end{equation*}%
\begin{equation*}
\text{through }\left( C_{1}^{1}C_{4}^{2}\right) \ ,\left(
C_{2}^{1}C_{3}^{2}\right) ,\left( C_{1}^{4}C_{2}^{3}\right) ,\text{ and }%
\left( C_{2}^{4}C_{3}^{3}\right) , 
\end{equation*}%
\begin{equation*}
C_{\left\{ 31,42\right\} }^{\left\{ 21,43\right\} }\equiv \left\langle
\sigma _{\left\{ 1,3\right\} }+\sigma _{\left\{ 2,4\right\} }:\sigma
_{\left\{ 1,4\right\} }-\sigma _{\left\{ 2,3\right\} }:\sigma _{\left\{
2,4\right\} }-\sigma _{\left\{ 1,4\right\} }-\sigma _{\left\{ 1,3\right\}
}-\sigma _{\left\{ 2,3\right\} }\right\rangle 
\end{equation*}%
\begin{equation*}
\text{through }\left( C_{3}^{1}C_{1}^{2}\right) \ ,\left(
C_{4}^{1}C_{2}^{2}\right) ,\left( C_{3}^{3}C_{1}^{4}\right) ,\text{ and }%
\left( C_{4}^{3}C_{2}^{4}\right) , 
\end{equation*}%
\begin{equation*}
C_{\left\{ 41,32\right\} }^{\left\{ 21,34\right\} }\equiv \left\langle
\sigma _{\left\{ 1,3\right\} }-\sigma _{\left\{ 2,4\right\} }:\sigma
_{\left\{ 2,3\right\} }+\sigma _{\left\{ 1,4\right\} }:\sigma _{\left\{
2,3\right\} }-\sigma _{\left\{ 1,4\right\} }-\sigma _{\left\{ 1,3\right\}
}-\sigma _{\left\{ 2,4\right\} }\right\rangle 
\end{equation*}%
\begin{equation*}
\text{through }\left( C_{3}^{1}C_{2}^{2}\right) \ ,\left(
C_{4}^{1}C_{1}^{2}\right) ,\left( C_{3}^{4}C_{2}^{3}\right) ,\text{ and }%
\left( C_{4}^{4}C_{1}^{3}\right) , 
\end{equation*}%
\begin{equation*}
C_{\left\{ 12,43\right\} }^{\left\{ 14,23\right\} }\equiv \left\langle
\sigma _{\left\{ 3,4\right\} }-\sigma _{\left\{ 1,2\right\} }-\sigma
_{\left\{ 1,3\right\} }-\sigma _{\left\{ 2,4\right\} }:\sigma _{\left\{
1,2\right\} }+\sigma _{\left\{ 3,4\right\} }:\sigma _{\left\{ 1,3\right\}
}-\sigma _{\left\{ 2,4\right\} }\right\rangle 
\end{equation*}%
\begin{equation*}
\text{through }\left( C_{1}^{1}C_{2}^{4}\right) \ ,\left(
C_{4}^{1}C_{3}^{4}\right) ,\left( C_{1}^{3}C_{3}^{4}\right) ,\text{ and }%
\left( C_{2}^{3}C_{4}^{4}\right) , 
\end{equation*}%
\begin{equation*}
C_{\left\{ 13,42\right\} }^{\left\{ 14,32\right\} }\equiv \left\langle
\sigma _{\left\{ 2,4\right\} }-\sigma _{\left\{ 1,2\right\} }-\sigma
_{\left\{ 1,3\right\} }-\sigma _{\left\{ 3,4\right\} }:\sigma _{\left\{
1,2\right\} }-\sigma _{\left\{ 3,4\right\} }:\sigma _{\left\{ 1,3\right\}
}+\sigma _{\left\{ 2,4\right\} }\right\rangle 
\end{equation*}%
\begin{equation*}
\text{through }\left( C_{1}^{1}C_{3}^{4}\right) \ ,\left(
C_{4}^{1}C_{2}^{4}\right) ,\left( C_{1}^{4}C_{2}^{3}\right) ,\text{ and }%
\left( C_{2}^{4}C_{3}^{3}\right) , 
\end{equation*}%
\begin{equation*}
C_{\left\{ 21,34\right\} }^{\left\{ 41,32\right\} }\equiv \left\langle
\sigma _{\left\{ 1,3\right\} }-\sigma _{\left\{ 1,2\right\} }+\sigma
_{\left\{ 2,4\right\} }+\sigma _{\left\{ 3,4\right\} }:\sigma _{\left\{
1,2\right\} }+\sigma _{\left\{ 3,4\right\} }:\sigma _{\left\{ 2,4\right\}
}-\sigma _{\left\{ 1,3\right\} }\right\rangle 
\end{equation*}%
\begin{equation*}
\text{through }\left( C_{2}^{1}C_{1}^{4}\right) \ ,\left(
C_{3}^{1}C_{4}^{4}\right) ,\left( C_{3}^{4}C_{2}^{3}\right) ,\text{ and }%
\left( C_{4}^{4}C_{1}^{3}\right) , 
\end{equation*}%
\begin{equation*}
C_{\left\{ 31,24\right\} }^{\left\{ 41,43\right\} }\equiv \left\langle
\sigma _{\left\{ 1,2\right\} }-\sigma _{\left\{ 1,3\right\} }+\sigma
_{\left\{ 2,4\right\} }+\sigma _{\left\{ 3,4\right\} }:\sigma _{\left\{
3,4\right\} }-\sigma _{\left\{ 1,2\right\} }:\sigma _{\left\{ 1,3\right\}
}+\sigma _{\left\{ 2,4\right\} }\right\rangle 
\end{equation*}%
\begin{equation*}
\text{through }\left( C_{2}^{1}C_{4}^{4}\right) \ ,\left(
C_{3}^{1}C_{1}^{4}\right) ,\left( C_{3}^{3}C_{1}^{4}\right) ,\text{ and }%
\left( C_{4}^{3}C_{2}^{4}\right) . 
\end{equation*}%
These computations are all done in matlab, the github address will be
provided at the end.
\end{proof}

\subsection{\protect\bigskip}

\subsection{Centroids}

\textbf{Median lines} (or just \textbf{medians}) $D$ of a Triangle $%
\triangle $ are the joins of corresponding Midpoints $m$ and Points $a$.
There are six Medians, two passing every Point of the Triangle. The Dual to
these are the Median points $d$ of a Triangle, which are the meets of
corresponding Midlines $M$ and Dual lines $A$. Since a Quadrangle has a
natural divide into Triangles, it possesses median structures.

\begin{theorem}[Subtriangle Medians of the Quadrangle]
The Medians of the Quadrangle $\square =\overline{a_{1}a_{2}a_{3}a_{4}}$ are
given as follows;

The Medians for the subtriangle $\triangle _{1}\,:$%
\begin{equation*}
\begin{array}{cc}
D_{23}^{1}\equiv \left\langle \sigma _{\left\{ 2,4\right\} }+\sigma
_{\left\{ 3,4\right\} }:\sigma _{\left\{ 3,4\right\} }:\sigma _{\left\{
2,4\right\} }\right\rangle , & D_{32}^{1}\equiv \left\langle \sigma
_{\left\{ 2,4\right\} }-\sigma _{\left\{ 3,4\right\} }:-\sigma _{\left\{
3,4\right\} }:\sigma _{\left\{ 2,4\right\} }\right\rangle , \\ 
D_{24}^{1}\equiv \left\langle \sigma _{\left\{ 3,4\right\} }:\sigma
_{\left\{ 2,3\right\} }+\sigma _{\left\{ 3,4\right\} }:\sigma _{\left\{
2,3\right\} }\right\rangle , & D_{42}^{1}\equiv \left\langle \sigma
_{\left\{ 3,4\right\} }:\sigma _{\left\{ 3,4\right\} }-\sigma _{\left\{
2,3\right\} }:-\sigma _{\left\{ 2,3\right\} }\right\rangle , \\ 
D_{34}^{1}\equiv \left\langle -\sigma _{\left\{ 2,4\right\} }:\sigma
_{\left\{ 2,3\right\} }:\sigma _{\left\{ 2,3\right\} }-\sigma _{\left\{
2,4\right\} }\right\rangle , & D_{43}^{1}\equiv \left\langle \sigma
_{\left\{ 2,4\right\} }:\sigma _{\left\{ 2,3\right\} }:\sigma _{\left\{
2,3\right\} }+\sigma _{\left\{ 2,4\right\} }\right\rangle .%
\end{array}%
\end{equation*}%
The Medians for the subtriangle $\triangle _{2}\,:$%
\begin{equation*}
\begin{array}{cc}
D_{13}^{2}\equiv \left\langle \sigma _{\left\{ 3,4\right\} }:\sigma
_{\left\{ 1,4\right\} }+\sigma _{\left\{ 3,4\right\} }:-\sigma _{\left\{
1,4\right\} }\right\rangle , & D_{31}^{2}\equiv \left\langle \sigma
_{\left\{ 3,4\right\} }:\sigma _{\left\{ 3,4\right\} }-\sigma _{\left\{
1,4\right\} }:\sigma _{\left\{ 1,4\right\} }\right\rangle , \\ 
D_{14}^{2}\equiv \left\langle \sigma _{\left\{ 1,3\right\} }+\sigma
_{\left\{ 3,4\right\} }:\sigma _{\left\{ 3,4\right\} }:-\sigma _{\left\{
1,3\right\} }\right\rangle , & D_{41}^{2}\equiv \left\langle \sigma
_{\left\{ 3,4\right\} }-\sigma _{\left\{ 1,3\right\} }:\sigma _{\left\{
3,4\right\} }:\sigma _{\left\{ 1,3\right\} }\right\rangle , \\ 
D_{34}^{2}\equiv \left\langle \sigma _{\left\{ 1,3\right\} }:-\sigma
_{\left\{ 1,4\right\} }:\sigma _{\left\{ 1,4\right\} }-\sigma _{\left\{
1,3\right\} }\right\rangle , & D_{43}^{2}\equiv \left\langle \sigma
_{\left\{ 1,3\right\} }:\sigma _{\left\{ 1,4\right\} }:-\sigma _{\left\{
1,4\right\} }-\sigma _{\left\{ 1,3\right\} }\right\rangle .%
\end{array}%
\end{equation*}%
The Medians for the subtriangle $\triangle _{3}\,:$%
\begin{equation*}
\begin{array}{cc}
D_{12}^{3}\equiv \left\langle \sigma _{\left\{ 2,4\right\} }:\sigma
_{\left\{ 1,4\right\} }:\sigma _{\left\{ 2,4\right\} }-\sigma _{\left\{
1,4\right\} }\right\rangle , & D_{21}^{3}\equiv \left\langle \sigma
_{\left\{ 2,4\right\} }:-\sigma _{\left\{ 1,4\right\} }:\sigma _{\left\{
2,4\right\} }+\sigma _{\left\{ 1,4\right\} }\right\rangle , \\ 
D_{14}^{3}\equiv \left\langle \sigma _{\left\{ 1,2\right\} }+\sigma
_{\left\{ 2,4\right\} }:-\sigma _{\left\{ 1,2\right\} }:\sigma _{\left\{
2,4\right\} }\right\rangle , & D_{41}^{3}\equiv \left\langle \sigma
_{\left\{ 2,4\right\} }-\sigma _{\left\{ 1,2\right\} }:\sigma _{\left\{
1,2\right\} }:\sigma _{\left\{ 2,4\right\} }\right\rangle , \\ 
D_{24}^{3}\equiv \left\langle \sigma _{\left\{ 2,3\right\} }:-\sigma
_{\left\{ 2,3\right\} }-\sigma _{\left\{ 3,4\right\} }:\sigma _{\left\{
3,4\right\} }\right\rangle , & D_{42}^{3}\equiv \left\langle \sigma
_{\left\{ 2,3\right\} }:\sigma _{\left\{ 3,4\right\} }-\sigma _{\left\{
2,3\right\} }:-\sigma _{\left\{ 3,4\right\} }\right\rangle .%
\end{array}%
\end{equation*}%
The Medians for the subtriangle $\triangle _{4}\,:$%
\begin{equation*}
\begin{array}{cc}
D_{12}^{4}\equiv \left\langle \sigma _{\left\{ 1,3\right\} }:\sigma
_{\left\{ 2,3\right\} }:\sigma _{\left\{ 2,3\right\} }-\sigma _{\left\{
1,3\right\} }\right\rangle , & D_{21}^{4}\equiv \left\langle -\sigma
_{\left\{ 1,3\right\} }:\sigma _{\left\{ 2,3\right\} }:\sigma _{\left\{
2,3\right\} }+\sigma _{\left\{ 1,3\right\} }\right\rangle , \\ 
D_{13}^{4}\equiv \left\langle -\sigma _{\left\{ 1,2\right\} }:\sigma
_{\left\{ 1,2\right\} }+\sigma _{\left\{ 2,3\right\} }:\sigma _{\left\{
2,3\right\} }\right\rangle , & D_{31}^{4}\equiv \left\langle \sigma
_{\left\{ 1,2\right\} }:\sigma _{\left\{ 2,3\right\} }-\sigma _{\left\{
1,2\right\} }:\sigma _{\left\{ 2,3\right\} }\right\rangle , \\ 
D_{23}^{4}\equiv \left\langle -\sigma _{\left\{ 1,2\right\} }-\sigma
_{\left\{ 1,3\right\} }:\sigma _{\left\{ 1,2\right\} }:\sigma _{\left\{
1,3\right\} }\right\rangle , & D_{32}^{4}\equiv \left\langle \sigma
_{\left\{ 1,2\right\} }-\sigma _{\left\{ 1,3\right\} }:-\sigma _{\left\{
1,2\right\} }:\sigma _{\left\{ 1,3\right\} }\right\rangle .%
\end{array}%
\end{equation*}
\end{theorem}

\begin{proof}
Simple computations will show this.
\end{proof}

\begin{theorem}[Centroids]
The Medians lines $D$ of a Triangle $\triangle =\overline{p_{1}p_{2}p_{3}}$
are concurrent in threes, meeting at four Centroid points $g$. The \textbf{%
Median points} $d$ of a Triangle are collinear in threes, joining on four 
\textbf{Centroid lines} $G$.
\end{theorem}

\begin{proof}
By the fundamental theorem of projective geometry we may assume without any
loss of generality that $p_{1}=a_{1}\equiv \left[ 1:1:1\right]
,~p_{2}=a_{2}\equiv \left[ -1:-1:1\right] $, and $p_{3}=a_{3}\equiv \left[
-1:1:1\right] $, and so by the Medians of the Quadrangle Theorem the Median
lines of the Triangle $\triangle =\overline{p_{1}p_{2}p_{3}}$ are; 
\begin{equation*}
\begin{array}{cc}
D_{12}^{4}\equiv \left\langle \sigma _{\left\{ 1,3\right\} }:\sigma
_{\left\{ 2,3\right\} }:\sigma _{\left\{ 2,3\right\} }-\sigma _{\left\{
1,3\right\} }\right\rangle , & D_{21}^{4}\equiv \left\langle -\sigma
_{\left\{ 1,3\right\} }:\sigma _{\left\{ 2,3\right\} }:\sigma _{\left\{
2,3\right\} }+\sigma _{\left\{ 1,3\right\} }\right\rangle , \\ 
D_{13}^{4}\equiv \left\langle -\sigma _{\left\{ 1,2\right\} }:\sigma
_{\left\{ 1,2\right\} }+\sigma _{\left\{ 2,3\right\} }:\sigma _{\left\{
2,3\right\} }\right\rangle , & D_{31}^{4}\equiv \left\langle \sigma
_{\left\{ 1,2\right\} }:\sigma _{\left\{ 2,3\right\} }-\sigma _{\left\{
1,2\right\} }:\sigma _{\left\{ 2,3\right\} }\right\rangle , \\ 
D_{23}^{4}\equiv \left\langle -\sigma _{\left\{ 1,2\right\} }-\sigma
_{\left\{ 1,3\right\} }:\sigma _{\left\{ 1,2\right\} }:\sigma _{\left\{
1,3\right\} }\right\rangle , & D_{32}^{4}\equiv \left\langle \sigma
_{\left\{ 1,2\right\} }-\sigma _{\left\{ 1,3\right\} }:-\sigma _{\left\{
1,2\right\} }:\sigma _{\left\{ 1,3\right\} }\right\rangle .%
\end{array}%
\end{equation*}%
The following triples of Medians are concurrent;%
\begin{equation*}
D_{12}^{4},D_{13}^{4},D_{23}^{4}, 
\end{equation*}%
passing through 
\begin{equation*}
g_{1}^{4}\equiv \left[ 
\begin{array}{c}
\sigma _{\left\{ 1,2\right\} }\sigma _{\left\{ 1,3\right\} }-\sigma
_{\left\{ 1,2\right\} }\sigma _{\left\{ 2,3\right\} }+\sigma _{\left\{
1,3\right\} }\sigma _{\left\{ 2,3\right\} }: \\ 
\sigma _{\left\{ 1,2\right\} }\sigma _{\left\{ 1,3\right\} }-\sigma
_{\left\{ 1,2\right\} }\sigma _{\left\{ 2,3\right\} }-\sigma _{\left\{
1,3\right\} }\sigma _{\left\{ 2,3\right\} }:\sigma _{\left\{ 1,2\right\}
}\sigma _{\left\{ 1,3\right\} }+\sigma _{\left\{ 1,2\right\} }\sigma
_{\left\{ 2,3\right\} }+\sigma _{\left\{ 1,3\right\} }\sigma _{\left\{
2,3\right\} }%
\end{array}%
\right] , 
\end{equation*}%
\begin{equation*}
D_{12}^{4},D_{31}^{4},D_{32}^{4}, 
\end{equation*}%
passing through 
\begin{equation*}
g_{2}^{4}\equiv \left[ 
\begin{array}{c}
\sigma _{\left\{ 1,2\right\} }\sigma _{\left\{ 1,3\right\} }-\sigma
_{\left\{ 1,2\right\} }\sigma _{\left\{ 2,3\right\} }-\sigma _{\left\{
1,3\right\} }\sigma _{\left\{ 2,3\right\} }: \\ 
\sigma _{\left\{ 1,2\right\} }\sigma _{\left\{ 1,3\right\} }-\sigma
_{\left\{ 1,2\right\} }\sigma _{\left\{ 2,3\right\} }+\sigma _{\left\{
1,3\right\} }\sigma _{\left\{ 2,3\right\} }:\sigma _{\left\{ 1,2\right\}
}\sigma _{\left\{ 1,3\right\} }+\sigma _{\left\{ 1,2\right\} }\sigma
_{\left\{ 2,3\right\} }-\sigma _{\left\{ 1,3\right\} }\sigma _{\left\{
2,3\right\} }%
\end{array}%
\right] , 
\end{equation*}%
\begin{equation*}
D_{21}^{4},D_{13}^{4},D_{32}^{4}, 
\end{equation*}%
passing through 
\begin{equation*}
g_{3}^{4}\equiv \left[ 
\begin{array}{c}
\sigma _{\left\{ 1,2\right\} }\sigma _{\left\{ 1,3\right\} }+\sigma
_{\left\{ 1,2\right\} }\sigma _{\left\{ 2,3\right\} }+\sigma _{\left\{
1,3\right\} }\sigma _{\left\{ 2,3\right\} }: \\ 
\sigma _{\left\{ 1,2\right\} }\sigma _{\left\{ 1,3\right\} }+\sigma
_{\left\{ 1,2\right\} }\sigma _{\left\{ 2,3\right\} }-\sigma _{\left\{
1,3\right\} }\sigma _{\left\{ 2,3\right\} }:\sigma _{\left\{ 1,2\right\}
}\sigma _{\left\{ 1,3\right\} }-\sigma _{\left\{ 1,2\right\} }\sigma
_{\left\{ 2,3\right\} }+\sigma _{\left\{ 1,3\right\} }\sigma _{\left\{
2,3\right\} }%
\end{array}%
\right] , 
\end{equation*}%
\begin{equation*}
D_{21}^{4},D_{31}^{4},D_{23}^{4}, 
\end{equation*}%
passing through 
\begin{equation*}
g_{4}^{4}\equiv \left[ 
\begin{array}{c}
\sigma _{\left\{ 1,2\right\} }\sigma _{\left\{ 1,3\right\} }+\sigma
_{\left\{ 1,2\right\} }\sigma _{\left\{ 2,3\right\} }-\sigma _{\left\{
1,3\right\} }\sigma _{\left\{ 2,3\right\} }: \\ 
\sigma _{\left\{ 1,2\right\} }\sigma _{\left\{ 1,3\right\} }+\sigma
_{\left\{ 1,2\right\} }\sigma _{\left\{ 2,3\right\} }+\sigma _{\left\{
1,3\right\} }\sigma _{\left\{ 2,3\right\} }:\sigma _{\left\{ 1,2\right\}
}\sigma _{\left\{ 1,3\right\} }-\sigma _{\left\{ 1,2\right\} }\sigma
_{\left\{ 2,3\right\} }-\sigma _{\left\{ 1,3\right\} }\sigma _{\left\{
2,3\right\} }%
\end{array}%
\right] . 
\end{equation*}%
This is checked by computing%
\begin{eqnarray*}
0 &=&\det \left( 
\begin{array}{ccc}
\sigma _{\left\{ 1,3\right\} } & \sigma _{\left\{ 2,3\right\} } & \sigma
_{\left\{ 2,3\right\} }-\sigma _{\left\{ 1,3\right\} } \\ 
-\sigma _{\left\{ 1,2\right\} } & \sigma _{\left\{ 1,2\right\} }+\sigma
_{\left\{ 2,3\right\} } & \sigma _{\left\{ 2,3\right\} } \\ 
-\sigma _{\left\{ 1,2\right\} }-\sigma _{\left\{ 1,3\right\} } & \sigma
_{\left\{ 1,2\right\} } & \sigma _{\left\{ 1,3\right\} }%
\end{array}%
\right) \\
&=&\det 
\begin{pmatrix}
\sigma _{\left\{ 1,3\right\} } & \sigma _{\left\{ 2,3\right\} } & \sigma
_{\left\{ 2,3\right\} }-\sigma _{\left\{ 1,3\right\} } \\ 
\sigma _{\left\{ 1,2\right\} } & \sigma _{\left\{ 2,3\right\} }-\sigma
_{\left\{ 1,2\right\} } & \sigma _{\left\{ 2,3\right\} } \\ 
\sigma _{\left\{ 1,2\right\} }-\sigma _{\left\{ 1,3\right\} } & -\sigma
_{\left\{ 1,2\right\} } & \sigma _{\left\{ 1,3\right\} }%
\end{pmatrix}
\\
&=&\det 
\begin{pmatrix}
-\sigma _{\left\{ 1,3\right\} } & \sigma _{\left\{ 2,3\right\} } & \sigma
_{\left\{ 2,3\right\} }+\sigma _{\left\{ 1,3\right\} } \\ 
-\sigma _{\left\{ 1,2\right\} } & \sigma _{\left\{ 1,2\right\} }+\sigma
_{\left\{ 2,3\right\} } & \sigma _{\left\{ 2,3\right\} } \\ 
\sigma _{\left\{ 1,2\right\} }-\sigma _{\left\{ 1,3\right\} } & -\sigma
_{\left\{ 1,2\right\} } & \sigma _{\left\{ 1,3\right\} }%
\end{pmatrix}
\\
&=&\det 
\begin{pmatrix}
-\sigma _{\left\{ 1,3\right\} } & \sigma _{\left\{ 2,3\right\} } & \sigma
_{\left\{ 2,3\right\} }+\sigma _{\left\{ 1,3\right\} } \\ 
\sigma _{\left\{ 1,2\right\} } & \sigma _{\left\{ 2,3\right\} }-\sigma
_{\left\{ 1,2\right\} } & \sigma _{\left\{ 2,3\right\} } \\ 
-\sigma _{\left\{ 1,2\right\} }-\sigma _{\left\{ 1,3\right\} } & \sigma
_{\left\{ 1,2\right\} } & \sigma _{\left\{ 1,3\right\} }%
\end{pmatrix}%
\end{eqnarray*}%
The coresponding meets are given above. The situations with the Centroid
lines $G$ is precisely dual.
\end{proof}

\begin{theorem}[Subtraingle Centroids of the Quadrangle]
The Subtriangle Centroids of the Quadrangle $\square =\overline{%
a_{1}a_{2}a_{3}a_{4}}$ are given as follows;

The Centoids for the Subtriangle $\triangle _{1}\,:$%
\begin{equation*}
g_{1}^{1}\equiv \left[ 
\begin{array}{c}
\sigma _{\left\{ 2,3\right\} }\sigma _{\left\{ 3,4\right\} }-\sigma
_{\left\{ 2,3\right\} }\sigma _{\left\{ 2,4\right\} }-\sigma _{\left\{
2,4\right\} }\sigma _{\left\{ 3,4\right\} }: \\ 
\sigma _{\left\{ 2,4\right\} }\sigma _{\left\{ 3,4\right\} }-\sigma
_{\left\{ 2,3\right\} }\sigma _{\left\{ 3,4\right\} }-\sigma _{\left\{
2,3\right\} }\sigma _{\left\{ 2,4\right\} }:\sigma _{\left\{ 2,3\right\}
}\sigma _{\left\{ 2,4\right\} }+\sigma _{\left\{ 2,3\right\} }\sigma
_{\left\{ 3,4\right\} }+\sigma _{\left\{ 2,4\right\} }\sigma _{\left\{
3,4\right\} }%
\end{array}%
\right] ,
\end{equation*}%
\begin{equation*}
\text{incident with the Medians }D_{23}^{1},D_{24}^{1},D_{34}^{1},
\end{equation*}%
\begin{equation*}
g_{2}^{1}\equiv \left[ 
\begin{array}{c}
\sigma _{\left\{ 2,3\right\} }\sigma _{\left\{ 2,4\right\} }+\sigma
_{\left\{ 2,3\right\} }\sigma _{\left\{ 3,4\right\} }-\sigma _{\left\{
2,4\right\} }\sigma _{\left\{ 3,4\right\} }: \\ 
\sigma _{\left\{ 2,3\right\} }\sigma _{\left\{ 2,4\right\} }-\sigma
_{\left\{ 2,3\right\} }\sigma _{\left\{ 3,4\right\} }+\sigma _{\left\{
2,4\right\} }\sigma _{\left\{ 3,4\right\} }:\sigma _{\left\{ 2,3\right\}
}\sigma _{\left\{ 3,4\right\} }-\sigma _{\left\{ 2,3\right\} }\sigma
_{\left\{ 2,4\right\} }+\sigma _{\left\{ 2,3\right\} }\sigma _{\left\{
3,4\right\} }%
\end{array}%
\right] ,
\end{equation*}%
\begin{equation*}
\text{incident with the Medians }D_{32}^{1},D_{42}^{1},D_{34}^{1};
\end{equation*}%
\begin{equation*}
g_{3}^{1}\equiv \left[ 
\begin{array}{c}
\sigma _{\left\{ 2,3\right\} }\sigma _{\left\{ 2,4\right\} }+\sigma
_{\left\{ 2,3\right\} }\sigma _{\left\{ 3,4\right\} }+\sigma _{\left\{
2,4\right\} }\sigma _{\left\{ 3,4\right\} }: \\ 
\sigma _{\left\{ 2,3\right\} }\sigma _{\left\{ 2,4\right\} }-\sigma
_{\left\{ 2,3\right\} }\sigma _{\left\{ 3,4\right\} }-\sigma _{\left\{
2,4\right\} }\sigma _{\left\{ 3,4\right\} }:\sigma _{\left\{ 2,3\right\}
}\sigma _{\left\{ 3,4\right\} }-\sigma _{\left\{ 2,3\right\} }\sigma
_{\left\{ 2,4\right\} }-\sigma _{\left\{ 2,3\right\} }\sigma _{\left\{
3,4\right\} }%
\end{array}%
\right] ,
\end{equation*}%
\begin{equation*}
\text{incident with the Medians }D_{32}^{1},D_{24}^{1},D_{43}^{1},
\end{equation*}%
\begin{equation*}
g_{4}^{1}\equiv \left[ 
\begin{array}{c}
\sigma _{\left\{ 2,3\right\} }\sigma _{\left\{ 2,4\right\} }-\sigma
_{\left\{ 2,3\right\} }\sigma _{\left\{ 3,4\right\} }-\sigma _{\left\{
2,4\right\} }\sigma _{\left\{ 3,4\right\} }: \\ 
\sigma _{\left\{ 2,3\right\} }\sigma _{\left\{ 2,4\right\} }+\sigma
_{\left\{ 2,3\right\} }\sigma _{\left\{ 3,4\right\} }+\sigma _{\left\{
2,4\right\} }\sigma _{\left\{ 3,4\right\} }:\sigma _{\left\{ 2,4\right\}
}\sigma _{\left\{ 3,4\right\} }-\sigma _{\left\{ 2,3\right\} }\sigma
_{\left\{ 3,4\right\} }-\sigma _{\left\{ 2,3\right\} }\sigma _{\left\{
2,4\right\} }%
\end{array}%
\right] ,
\end{equation*}%
\begin{equation*}
\text{incident with the Medians }D_{23}^{1},D_{42}^{1},D_{43}^{1},
\end{equation*}%
The Centroids for the Subtriangle $\triangle _{2}\,:$%
\begin{equation*}
g_{1}^{2}\equiv \left[ 
\begin{array}{c}
\sigma _{\left\{ 1,3\right\} }\sigma _{\left\{ 1,4\right\} }+\sigma
_{\left\{ 1,3\right\} }\sigma _{\left\{ 3,4\right\} }-\sigma _{\left\{
1,4\right\} }\sigma _{\left\{ 3,4\right\} }: \\ 
\sigma _{\left\{ 1,3\right\} }\sigma _{\left\{ 1,4\right\} }-\sigma
_{\left\{ 1,3\right\} }\sigma _{\left\{ 3,4\right\} }+\sigma _{\left\{
1,4\right\} }\sigma _{\left\{ 3,4\right\} }:\sigma _{\left\{ 1,3\right\}
}\sigma _{\left\{ 1,4\right\} }+\sigma _{\left\{ 1,3\right\} }\sigma
_{\left\{ 3,4\right\} }+\sigma _{\left\{ 1,4\right\} }\sigma _{\left\{
3,4\right\} }%
\end{array}%
\right] ,
\end{equation*}%
\begin{equation*}
\text{incident with the Medians }D_{13}^{2},D_{14}^{2},D_{34}^{2},
\end{equation*}%
\begin{equation*}
g_{2}^{2}\equiv \left[ 
\begin{array}{c}
\sigma _{\left\{ 1,3\right\} }\sigma _{\left\{ 1,4\right\} }-\sigma
_{\left\{ 1,3\right\} }\sigma _{\left\{ 3,4\right\} }+\sigma _{\left\{
1,4\right\} }\sigma _{\left\{ 3,4\right\} }: \\ 
\sigma _{\left\{ 1,3\right\} }\sigma _{\left\{ 1,4\right\} }+\sigma
_{\left\{ 1,3\right\} }\sigma _{\left\{ 3,4\right\} }-\sigma _{\left\{
1,4\right\} }\sigma _{\left\{ 3,4\right\} }:\sigma _{\left\{ 1,3\right\}
}\sigma _{\left\{ 1,4\right\} }-\sigma _{\left\{ 1,3\right\} }\sigma
_{\left\{ 3,4\right\} }-\sigma _{\left\{ 1,4\right\} }\sigma _{\left\{
3,4\right\} }%
\end{array}%
\right] ,
\end{equation*}%
\begin{equation*}
\text{incident with the Medians }D_{31}^{2},D_{41}^{2},D_{34}^{2};
\end{equation*}%
\begin{equation*}
g_{3}^{2}\equiv \left[ 
\begin{array}{c}
\sigma _{\left\{ 1,3\right\} }\sigma _{\left\{ 1,4\right\} }+\sigma
_{\left\{ 1,3\right\} }\sigma _{\left\{ 3,4\right\} }+\sigma _{\left\{
1,4\right\} }\sigma _{\left\{ 3,4\right\} }: \\ 
\sigma _{\left\{ 1,3\right\} }\sigma _{\left\{ 1,4\right\} }-\sigma
_{\left\{ 1,3\right\} }\sigma _{\left\{ 3,4\right\} }-\sigma _{\left\{
1,4\right\} }\sigma _{\left\{ 3,4\right\} }:\sigma _{\left\{ 1,3\right\}
}\sigma _{\left\{ 1,4\right\} }+\sigma _{\left\{ 1,3\right\} }\sigma
_{\left\{ 3,4\right\} }-\sigma _{\left\{ 1,4\right\} }\sigma _{\left\{
3,4\right\} }%
\end{array}%
\right] ,
\end{equation*}%
\begin{equation*}
\text{incident with the Medians }D_{13}^{2},D_{41}^{2},D_{43}^{2},
\end{equation*}%
\begin{equation*}
g_{4}^{2}\equiv \left[ 
\begin{array}{c}
\sigma _{\left\{ 1,3\right\} }\sigma _{\left\{ 1,4\right\} }-\sigma
_{\left\{ 1,3\right\} }\sigma _{\left\{ 3,4\right\} }-\sigma _{\left\{
1,4\right\} }\sigma _{\left\{ 3,4\right\} }: \\ 
\sigma _{\left\{ 1,3\right\} }\sigma _{\left\{ 1,4\right\} }+\sigma
_{\left\{ 1,3\right\} }\sigma _{\left\{ 3,4\right\} }+\sigma _{\left\{
1,4\right\} }\sigma _{\left\{ 3,4\right\} }:\sigma _{\left\{ 1,3\right\}
}\sigma _{\left\{ 1,4\right\} }-\sigma _{\left\{ 1,3\right\} }\sigma
_{\left\{ 3,4\right\} }+\sigma _{\left\{ 1,4\right\} }\sigma _{\left\{
3,4\right\} }%
\end{array}%
\right] ,
\end{equation*}%
\begin{equation*}
\text{incident with the Medians }D_{31}^{2},D_{14}^{2},D_{43}^{2},
\end{equation*}%
The Centroids for the Subtriangle $\triangle _{3}\,:$%
\begin{equation*}
g_{1}^{3}\equiv \left[ 
\begin{array}{c}
\sigma _{\left\{ 1,2\right\} }\sigma _{\left\{ 1,4\right\} }-\sigma
_{\left\{ 1,2\right\} }\sigma _{\left\{ 2,4\right\} }-\sigma _{\left\{
1,4\right\} }\sigma _{\left\{ 2,4\right\} }: \\ 
\sigma _{\left\{ 1,2\right\} }\sigma _{\left\{ 1,4\right\} }-\sigma
_{\left\{ 1,2\right\} }\sigma _{\left\{ 2,4\right\} }+\sigma _{\left\{
1,4\right\} }\sigma _{\left\{ 2,4\right\} }:\sigma _{\left\{ 1,2\right\}
}\sigma _{\left\{ 1,4\right\} }+\sigma _{\left\{ 1,2\right\} }\sigma
_{\left\{ 2,4\right\} }+\sigma _{\left\{ 1,4\right\} }\sigma _{\left\{
2,4\right\} }%
\end{array}%
\right] ,
\end{equation*}%
\begin{equation*}
\text{incident with the Medians }D_{12}^{3},D_{14}^{3},D_{24}^{3},
\end{equation*}%
\begin{equation*}
g_{2}^{3}\equiv \left[ 
\begin{array}{c}
\sigma _{\left\{ 1,2\right\} }\sigma _{\left\{ 1,4\right\} }-\sigma
_{\left\{ 1,2\right\} }\sigma _{\left\{ 2,4\right\} }+\sigma _{\left\{
1,4\right\} }\sigma _{\left\{ 2,4\right\} }: \\ 
\sigma _{\left\{ 1,2\right\} }\sigma _{\left\{ 1,4\right\} }-\sigma
_{\left\{ 1,2\right\} }\sigma _{\left\{ 2,4\right\} }-\sigma _{\left\{
1,4\right\} }\sigma _{\left\{ 2,4\right\} }:\sigma _{\left\{ 1,2\right\}
}\sigma _{\left\{ 1,4\right\} }+\sigma _{\left\{ 1,2\right\} }\sigma
_{\left\{ 2,4\right\} }-\sigma _{\left\{ 1,4\right\} }\sigma _{\left\{
2,4\right\} }%
\end{array}%
\right] ,
\end{equation*}%
\begin{equation*}
\text{incident with the Medians }D_{12}^{3},D_{41}^{3},D_{42}^{3},
\end{equation*}%
\begin{equation*}
g_{3}^{3}\equiv \left[ 
\begin{array}{c}
\sigma _{\left\{ 1,2\right\} }\sigma _{\left\{ 1,4\right\} }+\sigma
_{\left\{ 1,2\right\} }\sigma _{\left\{ 2,4\right\} }+\sigma _{\left\{
1,4\right\} }\sigma _{\left\{ 2,4\right\} }: \\ 
\sigma _{\left\{ 1,2\right\} }\sigma _{\left\{ 1,4\right\} }+\sigma
_{\left\{ 1,2\right\} }\sigma _{\left\{ 2,4\right\} }-\sigma _{\left\{
1,4\right\} }\sigma _{\left\{ 2,4\right\} }:\sigma _{\left\{ 1,2\right\}
}\sigma _{\left\{ 1,4\right\} }-\sigma _{\left\{ 1,2\right\} }\sigma
_{\left\{ 2,4\right\} }-\sigma _{\left\{ 1,4\right\} }\sigma _{\left\{
2,4\right\} }%
\end{array}%
\right] ,
\end{equation*}%
\begin{equation*}
\text{incident with the Medians }D_{21}^{3},D_{41}^{3},D_{24}^{3};
\end{equation*}%
\begin{equation*}
g_{4}^{3}\equiv \left[ 
\begin{array}{c}
\sigma _{\left\{ 1,2\right\} }\sigma _{\left\{ 1,4\right\} }+\sigma
_{\left\{ 1,2\right\} }\sigma _{\left\{ 2,4\right\} }-\sigma _{\left\{
1,4\right\} }\sigma _{\left\{ 2,4\right\} }: \\ 
\sigma _{\left\{ 1,2\right\} }\sigma _{\left\{ 1,4\right\} }+\sigma
_{\left\{ 1,2\right\} }\sigma _{\left\{ 2,4\right\} }+\sigma _{\left\{
1,4\right\} }\sigma _{\left\{ 2,4\right\} }:\sigma _{\left\{ 1,2\right\}
}\sigma _{\left\{ 1,4\right\} }-\sigma _{\left\{ 1,2\right\} }\sigma
_{\left\{ 2,4\right\} }+\sigma _{\left\{ 1,4\right\} }\sigma _{\left\{
2,4\right\} }%
\end{array}%
\right] ,
\end{equation*}%
\begin{equation*}
\text{incident with the Medians }D_{21}^{3},D_{14}^{3},D_{42}^{3},
\end{equation*}%
The Centroids for the Subtriangle $\triangle _{4}\,:$%
\begin{equation*}
g_{1}^{4}\equiv \left[ 
\begin{array}{c}
\sigma _{\left\{ 1,2\right\} }\sigma _{\left\{ 1,3\right\} }-\sigma
_{\left\{ 1,2\right\} }\sigma _{\left\{ 2,3\right\} }+\sigma _{\left\{
1,3\right\} }\sigma _{\left\{ 2,3\right\} }: \\ 
\sigma _{\left\{ 1,2\right\} }\sigma _{\left\{ 1,3\right\} }-\sigma
_{\left\{ 1,2\right\} }\sigma _{\left\{ 2,3\right\} }-\sigma _{\left\{
1,3\right\} }\sigma _{\left\{ 2,3\right\} }:\sigma _{\left\{ 1,2\right\}
}\sigma _{\left\{ 1,3\right\} }+\sigma _{\left\{ 1,2\right\} }\sigma
_{\left\{ 2,3\right\} }+\sigma _{\left\{ 1,3\right\} }\sigma _{\left\{
2,3\right\} }%
\end{array}%
\right] ,
\end{equation*}%
\begin{equation*}
\text{incident with the Medians }D_{12}^{4},D_{13}^{4},D_{23}^{4},
\end{equation*}%
\begin{equation*}
g_{2}^{4}\equiv \left[ 
\begin{array}{c}
\sigma _{\left\{ 1,2\right\} }\sigma _{\left\{ 1,3\right\} }-\sigma
_{\left\{ 1,2\right\} }\sigma _{\left\{ 2,3\right\} }-\sigma _{\left\{
1,3\right\} }\sigma _{\left\{ 2,3\right\} }: \\ 
\sigma _{\left\{ 1,2\right\} }\sigma _{\left\{ 1,3\right\} }-\sigma
_{\left\{ 1,2\right\} }\sigma _{\left\{ 2,3\right\} }+\sigma _{\left\{
1,3\right\} }\sigma _{\left\{ 2,3\right\} }:\sigma _{\left\{ 1,2\right\}
}\sigma _{\left\{ 1,3\right\} }+\sigma _{\left\{ 1,2\right\} }\sigma
_{\left\{ 2,3\right\} }-\sigma _{\left\{ 1,3\right\} }\sigma _{\left\{
2,3\right\} }%
\end{array}%
\right] ,
\end{equation*}%
\begin{equation*}
\text{incident with the Medians }D_{12}^{4},D_{31}^{4},D_{32}^{4},
\end{equation*}%
\begin{equation*}
g_{3}^{4}\equiv \left[ 
\begin{array}{c}
\sigma _{\left\{ 1,2\right\} }\sigma _{\left\{ 1,3\right\} }+\sigma
_{\left\{ 1,2\right\} }\sigma _{\left\{ 2,3\right\} }+\sigma _{\left\{
1,3\right\} }\sigma _{\left\{ 2,3\right\} }: \\ 
\sigma _{\left\{ 1,2\right\} }\sigma _{\left\{ 1,3\right\} }+\sigma
_{\left\{ 1,2\right\} }\sigma _{\left\{ 2,3\right\} }-\sigma _{\left\{
1,3\right\} }\sigma _{\left\{ 2,3\right\} }:\sigma _{\left\{ 1,2\right\}
}\sigma _{\left\{ 1,3\right\} }-\sigma _{\left\{ 1,2\right\} }\sigma
_{\left\{ 2,3\right\} }+\sigma _{\left\{ 1,3\right\} }\sigma _{\left\{
2,3\right\} }%
\end{array}%
\right] ,
\end{equation*}%
\begin{equation*}
\text{incident with the Medians }D_{21}^{4},D_{13}^{4},D_{32}^{4},
\end{equation*}%
\begin{equation*}
g_{4}^{4}\equiv \left[ 
\begin{array}{c}
\sigma _{\left\{ 1,2\right\} }\sigma _{\left\{ 1,3\right\} }+\sigma
_{\left\{ 1,2\right\} }\sigma _{\left\{ 2,3\right\} }-\sigma _{\left\{
1,3\right\} }\sigma _{\left\{ 2,3\right\} }: \\ 
\sigma _{\left\{ 1,2\right\} }\sigma _{\left\{ 1,3\right\} }+\sigma
_{\left\{ 1,2\right\} }\sigma _{\left\{ 2,3\right\} }+\sigma _{\left\{
1,3\right\} }\sigma _{\left\{ 2,3\right\} }:\sigma _{\left\{ 1,2\right\}
}\sigma _{\left\{ 1,3\right\} }-\sigma _{\left\{ 1,2\right\} }\sigma
_{\left\{ 2,3\right\} }-\sigma _{\left\{ 1,3\right\} }\sigma _{\left\{
2,3\right\} }%
\end{array}%
\right] .
\end{equation*}%
\begin{equation*}
\text{incident with the Medians }D_{21}^{4},D_{31}^{4},D_{23}^{4}.
\end{equation*}
\end{theorem}

\begin{proof}
This is worked out computationally.
\end{proof}

\bigskip 

Note that the labeling used mirrors the labeling for the Circumlines. That
is for Circumline $C_{j}^{i}$ that goes through the midpoints $%
m_{i_{1}j_{1}},m_{i_{2}j_{2}}$, and $m_{i_{3}j_{3}}$, then the Centroid $%
g_{j}^{i}$ is the meet of the Median lines $%
D_{j_{1}i_{1}}^{i},D_{j_{2}i_{2}}^{i}$, and $D_{j_{3}i_{3}}^{i}.$

\bigskip 

The \textbf{set of associated midpoints} (or \textbf{assosciated midpoints}) 
$S_{g}$ for a Centroid point $g$ of a Triangle $\triangle $, is the set of
three distinct midpoints used to construct it. For example the Centroid $%
g_{1}^{4}$ of the Subtriangle $\triangle _{4}$ is constructed from the
Median lines $D_{12}^{4},D_{13}^{4}$, and $D_{23}^{4}$, and hence $%
S_{1}^{4}\equiv \left\{ m_{12},m_{13},m_{23}\right\} $ is the set of
associated midpoints.

Note that the set of associated midpoints is unique and distinct for every
distinct Centroid.

A set $\left\{
g_{i_{1}}^{1},g_{i_{2}}^{2},g_{i_{3}}^{3},g_{i_{4}}^{4}\right\} $ containing
one Centroid from each Subtriangle is said to be\textbf{\ midpoint consistent%
} if the union of the associated midpoints $S_{m}=\cup S_{i_{k}}^{k}$,
conatins exactly one midpoint for every side $\overline{a_{i}a_{j}}$ of the
Quadrangle.

For example the set%
\begin{equation*}
S=\left\{ g_{1}^{1},g_{1}^{2},g_{1}^{3},g_{1}^{4}\right\} ,
\end{equation*}%
is midpoint consistent since the union of assosciated midpoints is%
\begin{equation*}
S_{m}=\left\{ m_{12},m_{34},m_{13},m_{24},m_{14},m_{23}\right\} .
\end{equation*}

Whereas the set%
\begin{equation*}
S=\left\{ g_{1}^{1},g_{1}^{2},g_{2}^{3},g_{2}^{4}\right\} 
\end{equation*}%
is not midpoint consitent since the union of assosciated midpoints%
\begin{equation*}
S_{m}=\left\{
m_{12},m_{34},m_{13},m_{31},m_{42},m_{14},m_{41},m_{32}\right\} 
\end{equation*}

contains both midpoints for the sides $\overline{a_{1}a_{3}}$ and $\overline{%
a_{1}a_{4}}$.

\begin{theorem}[Midpoint consistent sets of Subtriangle Centroids]
In total there are eight distinct sets of Subtriangle Centroids $S=\left\{
g_{i_{1}}^{1},g_{i_{2}}^{2},g_{i_{3}}^{3},g_{i_{4}}^{4}\right\} $ which are
midpoint consistent.
\end{theorem}

\begin{proof}
We prove this by trying to construct a midpoint consistent set of
Subtriangle Centroids $S=\left\{
g_{i_{1}}^{1},g_{i_{2}}^{2},g_{i_{3}}^{3},g_{i_{4}}^{4}\right\} $. First
let's choose $i_{1}=1$, that is let $g_{1}^{1}$ be in the set. The Centroid $%
g_{1}^{1}$ has assosciated midpoints $S_{1}^{1}\equiv \left\{
m_{23},m_{24},m_{34}\right\} $, and so for any other Centroid in S, the
midpoints $m_{32},m_{42}$, and $m_{43}$ cannot be in their respective set of
assosciated midpoints.

So looking at the Centroids $g_{i}^{2}$ of the Subtriangle $\triangle _{2}$
and their assosciated midpoints%
\begin{equation*}
\begin{array}{cc}
g_{1}^{2}:S_{1}^{2}\equiv \left\{ m_{13},m_{14},m_{34}\right\} , & 
g_{2}^{2}:S_{2}^{2}\equiv \left\{ m_{31},m_{41},m_{34}\right\} , \\ 
g_{3}^{2}:S_{3}^{2}\equiv \left\{ m_{31},m_{14},m_{43}\right\} , & 
g_{4}^{2}:S_{4}^{2}\equiv \left\{ m_{13},m_{41},m_{43}\right\} ,%
\end{array}%
\end{equation*}%
we see that the Centroids $g_{1}^{2}$ and $g_{4}^{2}$ are the only options
for $S$. If we choose $i_{2}=1$, then the set 
\begin{equation*}
\left\{ m_{34},m_{13},m_{24,}m_{14},m_{23}\right\} \subseteq S_{m},
\end{equation*}%
which forces $i_{3}=i_{4}=1$, as $S_{1}^{3}\equiv \left\{
m_{12},m_{14},m_{24}\right\} $ and $S_{1}^{4}\equiv \left\{
m_{12},m_{13},m_{23}\right\} $. Else if $i_{2}=2$, then the set 
\begin{equation*}
\left\{ m_{34},m_{31},m_{24},m_{41},m_{23}\right\} \subseteq S_{m},
\end{equation*}%
which forces $i_{3}=3$, and $i_{4}=4$, as $S_{3}^{3}\equiv \left\{
m_{21},m_{31},m_{24}\right\} $ and $S_{4}^{4}\equiv \left\{
m_{21},m_{31},m_{23}\right\} $. Therefore there are two distinct midpoint
consistent sets of Subtriangle Centroids which contain the Centroid $%
g_{1}^{1}$. The above method can be used for any choice of $i_{1}\in \left[ 4%
\right] $, and hence there are in total exactly eight distinct sets of
Subtriangle Centroids which are midpoint consistent.
\end{proof}

\begin{corollary}
The complete list of midpoint consistent set is given below;%
\begin{equation*}
\begin{array}{cc}
S: & S_{m}: \\ 
\left\{ g_{1}^{1},g_{1}^{2},g_{1}^{3},g_{1}^{4}\right\}  & \left\{
m_{12},m_{34},m_{13},m_{24},m_{14},m_{23}\right\}  \\ 
\left\{ g_{2}^{1},g_{2}^{2},g_{2}^{3},g_{2}^{4}\right\}  & \left\{
m_{12},m_{34},m_{31},m_{42},m_{41},m_{32}\right\}  \\ 
\left\{ g_{3}^{1},g_{3}^{2},g_{3}^{3},g_{3}^{4}\right\}  & \left\{
m_{21},m_{43},m_{13},m_{24},m_{41},m_{32}\right\}  \\ 
\left\{ g_{4}^{1},g_{4}^{2},g_{4}^{3},g_{4}^{4}\right\}  & \left\{
m_{21},m_{43},m_{31},m_{42},m_{14},m_{23}\right\}  \\ 
\left\{ g_{1}^{1},g_{2}^{2},g_{3}^{3},g_{4}^{4}\right\}  & \left\{
m_{21},m_{34},m_{31},m_{24},m_{41},m_{23}\right\}  \\ 
\left\{ g_{2}^{1},g_{1}^{2},g_{4}^{3},g_{3}^{4}\right\}  & \left\{
m_{21},m_{34},m_{13},m_{42},m_{14},m_{32}\right\}  \\ 
\left\{ g_{4}^{1},g_{3}^{2},g_{2}^{3},g_{1}^{4}\right\}  & \left\{
m_{12},m_{43},m_{13},m_{42},m_{41},m_{23}\right\}  \\ 
\left\{ g_{3}^{1},g_{4}^{2},g_{1}^{3},g_{2}^{4}\right\}  & \left\{
m_{21},m_{43},m_{31},m_{24},m_{14},m_{32}\right\} 
\end{array}%
\end{equation*}
\end{corollary}

\bigskip

A \textbf{Bimedian line} $B_{\left\{ ij,k\ell \right\} }$ is the join of two
midpoints $m_{ij}$, and $m_{k\ell }$ from opposite sides of the Quadrangle $%
\left\{ a_{i},a_{j}\right\} $ and $\left\{ a_{k},a_{\ell }\right\} $, where $%
\left[ 4\right] =\left\{ i,j,k,\ell \right\} $.

\begin{theorem}[Bimedian Lines of the Quadrangle]
The Bimedian lines $B_{\left\{ ij,k\ell \right\} }$ of the Quadrangle $%
\square =\overline{a_{1}a_{2}a_{3}a_{4}}$ are given as follows;

The Bimedian lines corresponding the $\alpha $ opposite sides:%
\begin{eqnarray*}
B_{\left\{ 12,34\right\} } &\equiv &\left\langle \sigma _{\left\{
1,3\right\} }-\sigma _{\left\{ 2,4\right\} }:\sigma _{\left\{ 2,3\right\}
}-\sigma _{\left\{ 1,4\right\} }:\sigma _{\left\{ 2,3\right\} }+\sigma
_{\left\{ 1,4\right\} }-\sigma _{\left\{ 1,3\right\} }-\sigma _{\left\{
2,4\right\} }\right\rangle ,\star  \\
B_{\left\{ 12,43\right\} } &\equiv &\left\langle \sigma _{\left\{
1,3\right\} }+\sigma _{\left\{ 2,4\right\} }:\sigma _{\left\{ 2,3\right\}
}+\sigma _{\left\{ 1,4\right\} }:\sigma _{\left\{ 2,3\right\} }-\sigma
_{\left\{ 1,4\right\} }-\sigma _{\left\{ 1,3\right\} }+\sigma _{\left\{
2,4\right\} }\right\rangle , \\
B_{\left\{ 21,34\right\} } &\equiv &\left\langle \sigma _{\left\{
1,3\right\} }+\sigma _{\left\{ 2,4\right\} }:-\sigma _{\left\{ 2,3\right\}
}-\sigma _{\left\{ 1,4\right\} }:\sigma _{\left\{ 1,4\right\} }-\sigma
_{\left\{ 2,3\right\} }-\sigma _{\left\{ 1,3\right\} }+\sigma _{\left\{
2,4\right\} }\right\rangle , \\
B_{\left\{ 21,43\right\} } &\equiv &\left\langle \sigma _{\left\{
2,4\right\} }-\sigma _{\left\{ 1,3\right\} }:\sigma _{\left\{ 2,3\right\}
}-\sigma _{\left\{ 1,4\right\} }:\sigma _{\left\{ 2,3\right\} }+\sigma
_{\left\{ 1,4\right\} }+\sigma _{\left\{ 1,3\right\} }+\sigma _{\left\{
2,4\right\} }\right\rangle ,\star 
\end{eqnarray*}%
The Bimedian lines corresponding the $\beta $ opposite sides:%
\begin{eqnarray*}
B_{\left\{ 13,24\right\} } &\equiv &\left\langle \sigma _{\left\{
3,4\right\} }-\sigma _{\left\{ 1,2\right\} }:\sigma _{\left\{ 1,4\right\}
}+\sigma _{\left\{ 3,4\right\} }+\sigma _{\left\{ 2,3\right\} }+\sigma
_{\left\{ 1,2\right\} }:\sigma _{\left\{ 2,3\right\} }-\sigma _{\left\{
1,4\right\} }\right\rangle ,\star  \\
B_{\left\{ 13,42\right\} } &\equiv &\left\langle \sigma _{\left\{
3,4\right\} }+\sigma _{\left\{ 1,2\right\} }:\sigma _{\left\{ 1,4\right\}
}-\sigma _{\left\{ 2,3\right\} }+\sigma _{\left\{ 3,4\right\} }-\sigma
_{\left\{ 1,2\right\} }:-\sigma _{\left\{ 1,4\right\} }-\sigma _{\left\{
2,3\right\} }\right\rangle , \\
B_{\left\{ 31,24\right\} } &\equiv &\left\langle \sigma _{\left\{
3,4\right\} }+\sigma _{\left\{ 1,2\right\} }:\sigma _{\left\{ 2,3\right\}
}-\sigma _{\left\{ 1,4\right\} }+\sigma _{\left\{ 3,4\right\} }-\sigma
_{\left\{ 1,2\right\} }:\sigma _{\left\{ 1,4\right\} }+\sigma _{\left\{
2,3\right\} }\right\rangle , \\
B_{\left\{ 31,42\right\} } &\equiv &\left\langle \sigma _{\left\{
1,2\right\} }-\sigma _{\left\{ 3,4\right\} }:\sigma _{\left\{ 2,3\right\}
}+\sigma _{\left\{ 1,4\right\} }-\sigma _{\left\{ 3,4\right\} }-\sigma
_{\left\{ 1,2\right\} }:\sigma _{\left\{ 2,3\right\} }-\sigma _{\left\{
1,4\right\} }\right\rangle ,
\end{eqnarray*}%
The Bimedian lines corresponding the $\gamma $ opposite sides:%
\begin{eqnarray*}
B_{\left\{ 14,23\right\} } &\equiv &\left\langle \sigma _{\left\{
1,2\right\} }+\sigma _{\left\{ 3,4\right\} }+\sigma _{\left\{ 1,3\right\}
}+\sigma _{\left\{ 2,4\right\} }:\sigma _{\left\{ 3,4\right\} }-\sigma
_{\left\{ 1,2\right\} }:\sigma _{\left\{ 2,4\right\} }-\sigma _{\left\{
1,3\right\} }\right\rangle ,\star  \\
B_{\left\{ 14,32\right\} } &\equiv &\left\langle \sigma _{\left\{
1,2\right\} }-\sigma _{\left\{ 3,4\right\} }+\sigma _{\left\{ 2,4\right\}
}-\sigma _{\left\{ 1,3\right\} }:-\sigma _{\left\{ 3,4\right\} }-\sigma
_{\left\{ 1,2\right\} }:\sigma _{\left\{ 2,4\right\} }+\sigma _{\left\{
1,3\right\} }\right\rangle , \\
B_{\left\{ 41,23\right\} } &\equiv &\left\langle \sigma _{\left\{
3,4\right\} }-\sigma _{\left\{ 1,2\right\} }+\sigma _{\left\{ 2,4\right\}
}-\sigma _{\left\{ 1,3\right\} }:\sigma _{\left\{ 3,4\right\} }+\sigma
_{\left\{ 1,2\right\} }:\sigma _{\left\{ 2,4\right\} }+\sigma _{\left\{
1,3\right\} }\right\rangle , \\
B_{\left\{ 41,32\right\} } &\equiv &\left\langle \sigma _{\left\{
3,4\right\} }+\sigma _{\left\{ 1,2\right\} }-\sigma _{\left\{ 2,4\right\}
}-\sigma _{\left\{ 1,3\right\} }:\sigma _{\left\{ 3,4\right\} }-\sigma
_{\left\{ 1,2\right\} }:\sigma _{\left\{ 1,3\right\} }-\sigma _{\left\{
2,4\right\} }\right\rangle .
\end{eqnarray*}
\end{theorem}

\begin{proof}
Through using the cross product and then careful use of the sigma identities
the corresponding Bimedian lines are worked out to be as above.
\end{proof}

\begin{theorem}[Centroids of the Quadrangle]
The Bimedian lines $B_{\left\{ ij,k\ell \right\} }$ of the Quadrangle are
concurrent three at a time at the \textbf{Quadrangle Centroids }q.

\begin{proof}
Each Quadrangle Centroids is given by a midpoint consistent set of
Subtriangle Centroids. That is for the midpoint consistent set 
\begin{equation*}
\left\{ g_{1}^{1},g_{1}^{2},g_{1}^{3},g_{1}^{4}\right\} 
\end{equation*}%
with union of associated midpoints%
\begin{equation*}
S_{m}=\left\{ m_{12},m_{34},m_{13},m_{24},m_{14},m_{23}\right\} .
\end{equation*}%
Since $S_{m}$ has exactly one midpoint for each side of the quadrangle, we
have three corresponding Bimedian lines 
\begin{equation*}
B_{\left\{ 12,34\right\} },B_{\left\{ 13,24\right\} }\text{, and }B_{\left\{
14,23\right\} }\text{.}
\end{equation*}%
These Bimedians intersect at the point%
\begin{equation*}
\left[ \sigma _{\left\{ 1,2\right\} }\sigma _{\left\{ 1,3\right\} }-\sigma
_{\left\{ 1,2\right\} }\sigma _{\left\{ 1,4\right\} }+\sigma _{\left\{
1,3\right\} }\sigma _{\left\{ 1,4\right\} }-\sigma _{\left\{ 1,2\right\}
}\sigma _{\left\{ 2,3\right\} }+\sigma _{\left\{ 1,3\right\} }\sigma
_{\left\{ 2,3\right\} }-4\sigma _{\left\{ 1,4\right\} }\sigma _{\left\{
2,3\right\} }\right] 
\end{equation*}
\end{proof}
\end{theorem}

\bigskip \pagebreak

\chapter{\protect\bigskip Introduction}

Throughout this thesis definitions will be given in bold and italics will be
reserved for emphasis.\newline

\textbf{Universal Geometry} Through relatively recent developements in the
field of geometry, Norman Wildberger has shown that hyperbolic geometry can
be considered as an agebraic projective geometry. In the following pages we
will aim to define the fundamental objects of hyperbolic geometry and how
they interact with each other.\newline

\textbf{Projective Geometry} Universal Projective geometry is a geometry in
the space of lines through the origin of a vector space with a metrical
structure given by a symmetric bilinear form.\newline
The complete algebraic nature of Universal geometry implies that we have an
algebraic construction of Projective geometry. The focus of this chapter is
to introduce the main objects of Universal Projective geometry and define
their incidence relations in such a way to induce a complete duality between
points and lines in this projective setting. This concept of complete
duality is a defining characteristic of Projective geometry.\newline
Universal geometry is given as an algebraic geometry, where the algebraic
framework for Universal Projective geometry is given to us through \textit{%
projective linear algebra}. Projective linear algebra is much like normal
linear algebra but vectors and matrices are only defined up to non-zero
scalar multiples. In this thesis the convention of writing the usual affine
vectors and matrices with round brackets and projective vectors and matrices
with square brackets. Hence for a given row vector $v=(1\;\;2\;\;3)$ we
denote the associated projective vector $a=[v]$ as $a=[1\;\;2\;\;3]$ which
by definition also equal to $[-1\;\;-2\;\;-3]$ or to $[2\;\;4\;\;6]$. We
will also use bold labels to denote projective matrices: for for ordinary
matrices 
\begin{equation*}
A=%
\begin{pmatrix}
1 & 1 & 2 \\ 
0 & 2 & 0 \\ 
0 & 0 & 1%
\end{pmatrix}
\;\;\text{and}\;\; B=%
\begin{pmatrix}
2 & -1 & -4 \\ 
0 & 1 & 0 \\ 
0 & 0 & 2%
\end{pmatrix}%
, 
\end{equation*}
the associated projective matrices are 
\begin{align*}
\mathbf{A}=%
\begin{bmatrix}
1 & 1 & 2 \\ 
0 & 2 & 0 \\ 
0 & 0 & 1%
\end{bmatrix}%
= 
\begin{bmatrix}
-1 & -1 & -2 \\ 
0 & -2 & 0 \\ 
0 & 0 & -1%
\end{bmatrix}%
, \\
\mathbf{B}=%
\begin{bmatrix}
2 & -1 & -4 \\ 
0 & 1 & 0 \\ 
0 & 0 & 2%
\end{bmatrix}%
= 
\begin{bmatrix}
4 & -2 & -8 \\ 
0 & 2 & 0 \\ 
0 & 0 & 4%
\end{bmatrix}%
\end{align*}
Where infact $\mathbf{A}^{-1}=\mathbf{B}$ in the projective setting, as
scalar multiplies can be disregarded. It turns out that addition of
projective matrices is not well defined but multiplication is.\newline
It is now important to introduce the main objects of projective geometry in
a way that is consistent. A \textbf{(projective) point} is a \textit{non-zero%
} projective row vector $a$ and will be written in either of two ways: 
\begin{equation*}
a\equiv[x\;y\;z]\equiv[x:y:z]. 
\end{equation*}
A \textbf{(projective) line} is a \textit{non-zero} projective column vector 
$L$ written as 
\begin{equation*}
L\equiv%
\begin{bmatrix}
l \\ 
m \\ 
n%
\end{bmatrix}%
\equiv\langle l:m:n\rangle. 
\end{equation*}
For the point $a=[x:y:z]$ and line $L=\langle l:m:n\rangle$ we say they are 
\textbf{incident} precisely when 
\begin{equation}
aL\equiv[x\;y\;z]%
\begin{bmatrix}
l \\ 
m \\ 
n%
\end{bmatrix}%
\equiv0.
\end{equation}
Three or more lines are \textbf{concurrent} precisely when they are all
incident with a point $a$, and dually three or more points are \textbf{%
collinear} precisely when they are all incident with a line $L$.\newline
The \textbf{join} $a_{1}a_{2}$ of distinct points $a_{1}\equiv[%
x_{1}:y_{1}:z_{1}]$ and $a_{2}\equiv[x_{2}:y_{2}:z_{2}]$ is the line 
\begin{equation*}
a_{1}a_{2}\equiv[x_{1}:y_{1}:z_{1}]\times[x_{2}:y_{2}:z_{2}]\equiv\langle
y_{1}z_{2}-z_{1}y_{2}:z_{1}x_{2}-x_{1}z_{2}:x_{1}y_{2}-y_{1}x_{2}\rangle. 
\end{equation*}
The \textbf{meet} $L_{1}L_{2}$ of two distinct points $L_{1}\equiv\langle
l_{1}:m_{1}:n_{1}\rangle$ and $l_{2}\equiv\langle l_{2}:m_{2}:n_{2}\rangle$
is the point 
\begin{equation*}
L_{1}L_{2}\equiv\langle l_{1}:m_{1}:n_{1}\rangle\times\langle
l_{2}:m_{2}:n_{2}\rangle\equiv[%
m_{1}n_{2}-n_{1}m_{2}:n_{1}l_{2}-l_{1}n_{2}:l_{1}m_{2}-m_{1}l_{2}]. 
\end{equation*}
The cross here is the usual Euclidean cross product which is well defined.
This also induces the result that \textit{the join $a_{1}a_{2}$ is a unique
line which is incident with the points $a_{1}$ and $a_{2}$}. Dually \textit{%
the meet $L_{1}L_{2}$ is a unique point which is incident with the lines $%
L_{1}$ and $L_{2}$}.

\pagebreak A \textbf{3-proportion} $x:y:z$ is an ordered triple of numbers $%
x,y$ and $z$, \textit{not all zero}, with the convention that for any
non-zero number $\lambda$ 
\begin{equation*}
x:y:z=\lambda x:\lambda y:\lambda z. 
\end{equation*}
This is equivalent to saying that 
\begin{equation*}
x_{1}:y_{1}:z_{1}=x_{2}:y_{2}:z_{2} 
\end{equation*}
precisely when the following conditions hold 
\begin{equation}  \label{proportionequal}
x_{1}y_{2}-x_{2}y_{1}=0\;\;y_{1}z_{2}-y_{2}z_{1}=0\;%
\;z_{1}x_{2}-z_{2}x_{1}=0.
\end{equation}

Now that the notion of a proportion is set up we can define the two main
hyperbolic objects. A \textbf{(hyperbolic) point} is a 3-proportion $a\equiv[%
x:y:z]$ enclosed in square brackets. Where a \textbf{(hyperbolic) line} is a
3-proportion $L\equiv(l:m:n)$ enclosed in round brackets.\newline

The definitions of points and lines is equivalent to that of projective
geometry, where the two types of geometry differ becomes obvious in the
notion of duality. The point $a\equiv[x,y,z]$ is \textbf{dual} to the line $%
L\equiv(l:m:n)$ precisely when 
\begin{equation*}
x:y:z=l:m:n. 
\end{equation*}
In this case we say that $a^{\perp}=L$ or $L^{\perp}=a$.\newline
From the definition of points and lines we get that each point is dual to
exactly one line, and conversely. This new idea of duality induces the same
property that \textit{there is a complete duality in the theory between
points and lines}, within this new projective geometry.\newline

Now that we have set up the basic objects of hyperbolic geometry, an
important step is to define the incidence of these objects with each other.
The following theorems and definitions aim to do exactly that.\newline

The point $a\equiv[x:y:z]$ \textbf{lies on} the line $L\equiv(l:m:n)$, or
equivalently $L$ \textbf{passes through} $a$, precisely when 
\begin{equation*}
lx+my-nz=0. 
\end{equation*}%
\newline

Points $a_{1}\equiv[x_{1}:y_{1}:z_{1}]$ and $a_{2}\equiv[x_{2}:y_{2}:z_{2}]$
are \textbf{perpendicular} precisely when 
\begin{equation*}
x_{1}x_{2}+y_{1}y_{2}-z_{1}z_{2}=0. 
\end{equation*}
This is equivalent to the condition that $a_{1}$ is incident with $%
a_{2}^{\perp}$, or that $a_{2}$ is incident with $a_{1}^{\perp}$.\newline

Similarly the line $L_{1}\equiv(l_{1}:m_{1}:n_{1})$ and $L_{2}%
\equiv(l_{2}:m_{2}:n_{2})$ are \textbf{perpendicular} precisely when 
\begin{equation*}
l_{1}l_{2}+m_{1}m_{2}-n_{1}n_{2}=0. 
\end{equation*}
This is equivalent to the condition that $L_{1}$ is incident with $%
L_{2}^{\perp}$, or that $L_{2}$ is incident with $L_{1}^{\perp}$.\newline

We will denote by $\mathbb{F}^{3}$ the 3-dimensional space of \textbf{vectors%
} $v\equiv(x,y,z)$. If $v\equiv(x,y,z)$ has coordinates which are not all
zero, then let $[v]\equiv[x:y:z]$ denote the (hyperbolic) point, and $%
(v)\equiv(l:m:n)$ denote the (hyperbolic) line.\newline

\begin{theorem}[Joins of points]
If $a_{1}\equiv[x_{1}:y_{1}:z_{1}]$ and $a_{2}\equiv[x_{2}:y_{2}:z_{2}]$ are
distinct points, then there is exactly one line $L$ which passes through
them both, namely 
\begin{equation*}
L\equiv
a_{1}a_{2}%
\equiv(y_{1}z_{2}-y_{2}z_{1}:z_{1}x_{2}-z_{2}x_{1}:x_{2}y_{1}-x_{1}y_{2}). 
\end{equation*}
\end{theorem}

The line $L\equiv a_{1}a_{2}$ is the \textbf{join} of the points $a_{1}$ and 
$a_{2}$.\newline

\begin{theorem}[Meets of lines]
If $L_{1}\equiv(l_{1}:m_{1}:n_{1})$ and $L_{2}\equiv(l_{2}:m_{2}:n_{2})$ are
distinct lines, then there is exactly one point $a$ which lies on both,
namely 
\begin{equation*}
a\equiv L_{1}L_{2}\equiv[%
m_{1}n_{2}-m_{2}n_{1}:n_{1}l_{2}-n_{2}l_{1}:l_{2}m_{1}-l_{1}m_{2}]. 
\end{equation*}
\end{theorem}

The point $a\equiv L_{1}L_{2}$ is the \textbf{meet} of the lines $L_{1}$ and 
$L_{2}$. These definitions give the following consequence, for any distinct
points $a_{1}$ and $a_{2}$, and distinct lines $L_{1}$ and $L_{2}$, 
\begin{equation*}
(a_{1}a_{2})^{\perp}=a_{1}^{\perp}a_{2}\perp,\;\;\text{and}%
\;\;(L_{1}L_{2})^{\perp}=L_{1}^{\perp}L_{2}^{\perp}. 
\end{equation*}

Three or more points which lie on a common line are \textbf{collinear}.
Three or more lines which pass through a common point are \textbf{concurrent}%
.\newline

\begin{theorem}[Collinear points]
The points $a_{1}\equiv[x_{1}:y_{1}:z_{1}],\; a_{2}\equiv[x_{2}:y_{2}:z_{2}]$
and $a_{3}\equiv[x_{3}:y_{3}:z_{3}]$ are collinear precisely when 
\begin{equation*}
x_{1}y_{2}z_{3}-x_{1}y_{3}z_{2}+x_{2}y_{3}z_{1}-x_{2}y_{1}z_{3}+x_{3}y_{1}z_{2}-x_{3}y_{2}z_{1}=0. 
\end{equation*}
\end{theorem}

\begin{theorem}[Concurrent lines]
The lines $L_{1}\equiv(l_{1}:m_{1}:n_{1}),\; L_{2}\equiv(l_{2}:m_{2}:n_{2})$
and $L_{3}\equiv(l_{3}:m_{3}:n_{3})$ are concurrent precisely when 
\begin{equation*}
l_{1}m_{2}n_{3}-l_{1}m_{3}n_{2}+l_{2}m_{3}n_{1}-l_{2}m_{1}n_{3}+l_{3}m_{1}n_{2}-l_{3}m_{2}n_{1}=0. 
\end{equation*}
\end{theorem}

Now that the fundamental objects and there interactions are defined will
begin to classify these objects. Firstly we say that the point $a\equiv[x:y:z%
]$ is \textbf{null} precisely when it lies on its dual line, that is when 
\begin{equation*}
x^{2}+y^{2}-z^{2}=0, 
\end{equation*}
and analogously that the line $L\equiv(l:m:n)$ is \textbf{null} precisely
when it passes through its dual point, that is when 
\begin{equation*}
l^{2}+m^{2}-n^{2}=0. 
\end{equation*}
The dual of a null point is a null line and conversely. I'll now use the
following theorem to further this classification of objects.\newline

\begin{theorem}[Line through null points, and Point on null lines]
Any line $L$ passes through at most two null points, and any point $a$ lies
on at most two null lines.
\end{theorem}

Therefore we have a naturally classification of points and lines. For a
non-null point $a$ we say that it is \textbf{internal} precisely when it
lies on no null lines, and is \textbf{external} precisely when it lies on 2
null lines. Whereas a non-null line $L$ is said to be \textbf{external}
precisely when it passes through no null points and \textbf{internal}
precisely when it passes through no null points. That is all points and
lines are either \textit{internal, null} or \textit{external}.\newline
Unlike null points and lines we have that the dual of an internal point is
an external line, and the dual of an external point is an internal line and
conversely.

\pagebreak

We now go onto define the geometric objects of hyperbolic geometry.\newline

A \textbf{side} $\overline{a_{1}a_{2}}$ is a set $\{a_{1},a_{2}\}$ of two
points. A \textbf{vertex} $\overline{L_{1}L_{2}}$ is a set $\{L_{1},L_{2}\}$
of two lines. From the definition it is clear that 
\begin{equation*}
\overline{a_{1}a_{2}}=\overline{a_{2}a_{1}}\;\; \text{and}\;\;\overline{%
L_{1}L_{2}}=\overline{L_{2}L_{1}}. 
\end{equation*}
For a side $\overline{a_{1}a_{2}}$ we say that $a_{1}a_{2}$ is the \textbf{%
line} of the side. Whilst for a vertex $\overline{L_{1}L_{2}}$ we say that $%
L_{1}L_{2}$ is the \textbf{point} of the vertex.\newline

Much like the fundamental objects, we can continue to classify these new
objects. We say that a side $\overline{a_{1}a_{2}}$ is a \textbf{nil side}
precisely when at least one of $a_{1}$ or $a_{2}$ is a null point. Thus we
are able to further classify sides, such as the side $\overline{a_{1}a_{2}}$
as a \textbf{singly-nil side}, or a \textbf{doubly-nil side} respectively,
precisely when exactly one of the points $a_{1}$ or $a_{2}$ are null, or
exactly both of the points $a_{1}$ and $a_{2}$ are null points respectively.
Similarly we are able to classify the vertex $\overline{L_{1}L_{2}}$ as a 
\textbf{singly-nil vertex} or a \textbf{doubly-nil vertex} in a natural way.%
\newline

\begin{theorem}[Perpendicular point]
For any side $\overline{a_{1}a_{2}}$ there is a unique point $p$ which is
perpendicular to both $a_{1}$ and $a_{2}$, namely 
\begin{equation*}
p\equiv a_{1}^{\perp}a_{2}^{\perp}=(a_{1}a_{2})^{\perp}. 
\end{equation*}
\end{theorem}

The point $p$ is the \textbf{perpendicular point} of $\overline{a_{1}a_{2}}$%
. It is possible that $p$ may lie on $a_{1}a_{2}$; this occurs precisely
when $a_{1}a_{2}$ is a null line.\newline

\begin{theorem}[Perpendicular line]
For any vertex $\overline{L_{1}L_{2}}$ there is a unique line $P$ which is
perpendicular to both $L_{1}$ and $L_{2}$, namely 
\begin{equation*}
P\equiv L_{1}^{\perp}L_{2}^{\perp}=(L_{1}L_{2})^{\perp}. 
\end{equation*}
\end{theorem}

The line $P$ is the \textbf{perpendicular line} of $\overline{L_{1}L_{2}}$.
It also may happen that $P$ passes through $L_{1}L_{2}$, which occurs
precisely when $L_{1}L_{2}$ is a null point.\newline

\pagebreak

As we are in a projective setting, the definition of a (hyperbolic)
quadrangle (quadrilateral respectively,) will come from the projective
definition of a complete quadrangle (quadrilateral respectively.)\newline

A \textbf{quadrangle} $\overline{a_{1}a_{2}a_{3}a_{4}}$ is a set $%
\{a_{1},a_{2},a_{3},a_{4}\}$ of points which has the property that no three
are collinear. A \textbf{quadrilateral} $\overline{L_{1}L_{2}L_{3}L_{4}}$ is
a set $\{L_{1},L_{2},L_{3},L_{4}\}$ of lines which has the property that no
three are concurrent.\newline

The quadrangle $\square\equiv \overline{a_{1}a_{2}a_{3}a_{4}}$ has a \textbf{%
dual quadrilateral} $\square^{\perp}\equiv\overline{a_{1}^{\perp}a_{2}^{%
\perp}a_{3}^{\perp}a_{4}^{\perp}}$ consisting of four \textbf{dual lines} of
the quadrangle, namely $a_{1}^{\perp},a_{2}^{\perp},a_{3}^{\perp}$ and $%
a_{4}^{\perp}$.\newline

The quadrilateral $\lozenge\equiv \overline{L_{1}L_{2}L_{3}L_{4}}$ has a 
\textbf{dual quadrangle} $\lozenge^{\perp}\equiv\overline{%
L_{1}^{\perp}L_{2}^{\perp}L_{3}^{\perp}L_{4}^{\perp}}$ consisting of four 
\textbf{dual points} of the quadrilateral, namely $L_{1}^{\perp},L_{2}^{%
\perp},L_{3}^{\perp}$ and $L_{4}^{\perp}$.\newline

There are 6 distinct sides of a quadrangle, namely $\overline{a_{1}a_{2}},%
\overline{a_{3}a_{4}},$ $\overline{a_{1}a_{3}},\overline{a_{2}a_{4}},%
\overline{a_{1}a_{4}}$ and $\overline{a_{2}a_{3}}$. We can naturally divide
these 6 sides into 3 pairs, $\{\overline{a_{1}a_{2}},\overline{a_{3}a_{4}}\}$%
, $\{\overline{a_{1}a_{3}},\overline{a_{2}a_{4}}\}$, and $\{\overline{%
a_{1}a_{4}},\overline{a_{2}a_{3}}\}$. The intersection of these pairs of
sides give three new points called the \textbf{diagonal points} of the
quadrangle.\newline
Similarly there are 6 distinct vertices of a quadrilateral, namely $%
\overline{L_{1}L_{2}},\overline{L_{3}L_{4}},$ $\overline{L_{1}L_{3}},%
\overline{L_{2}L_{4}},\overline{L_{1}L_{4}}$and $\overline{L_{2}L_{3}}$.
These too have a natural divide into 3 pairs, $\{\overline{L_{1}L_{2}},%
\overline{L_{3}L_{4}}\}$, $\{\overline{L_{1}L_{3}},\overline{L_{2}L_{4}}\}$,
and $\{\overline{L_{1}L_{4}},\overline{L_{2}L_{3}}\}$. The join of these
pairs of vertices give three new lines called the \textbf{diagonal lines} of
the quadrilateral.\newline

\begin{theorem}[Diagonal triangle]
The diagonal points $d_{1}\equiv(a_{1}a_{4})(a_{2}a_{3}),$\newline
$d_{2}\equiv(a_{2}a_{4})(a_{1}a_{3})$ and $d_{3}%
\equiv(a_{3}a_{4})(a_{1}a_{2})$ of the quadrangle $\square\equiv\overline{%
a_{1}a_{2}a_{3}a_{4}}$ form the triangle $\triangle\equiv\overline{%
d_{1}d_{2}d_{3}}$.
\end{theorem}

\begin{proof}
\begin{itemize}
\item Use join of points and meets of lines to write out each $d_{i}$.

\item Condition for $d_{i}$s to be collinear

\item $a_{i}$s non collinear follows $d_{i}$s non collinear.
\end{itemize}
\end{proof}

\begin{theorem}[Diagonal trilateral]
The diagonal lines $D_{1}\equiv(L_{1}L_{4})(L_{2}L_{3}),$\newline
$D_{2}\equiv(L_{2}L_{4})(L_{1}L_{3})$ and $D_{3}%
\equiv(L_{3}L_{4})(L_{1}L_{2})$ of the quadrilateral $\lozenge\equiv%
\overline{L_{1}L_{2}L_{3}L_{4}}$ form the trilateral $\triangledown\equiv%
\overline{D_{1}D_{2}D_{3}}$.
\end{theorem}

\begin{proof}
Dual to the previous theorem.
\end{proof}

\pagebreak

\section{\protect\bigskip Geogebra Tools}

Tools that I've made to be used in GeoGebra for use in exploring Universal
Hyperbolic geometry. Firstly I will present a list of all the tool that I
have created within GeoGebra (using it's tool creating system), and then I
will present how I created them.\newline

\bigskip 

\begin{itemize}
\item Dual Line (Polar)

\item Dual Point (Pole)

\item Meet of Lines

\item Altitude Line/Point

\item Base Point/Line

\item Parallel Line/Point

\item Reflections (Reflect\_PiP, Reflect\_PiL, Reflect\_LiP, Reflect\_LiL)

\item Midpoints/lines

\item Bilines/points

\item Sydpoint/lines

\item Silines/points

\item Smydpoint/lines

\item Sbilines/points

\item Side Conjugate Points/Lines

\item Vertex Conjugate Lines/Points

\item Quadrance, Spread, Quadrea

\item Orthocenter/line

\item Circumcenter/lines

\item Centroid Points/Lines

\item Diagonal Triangle (DiagTri)
\end{itemize}

\bigskip 

For this section the Null conic will be given the fixed label $c$. Whereas
the non-null point in question will be given by $a$, with non-null dual $A$,
whilst Null points will be denoted by letters of the \textit{Greek alphabet}.

\bigskip 

The \textit{duality} of points and lines in Universal Hyperbolic Geometry is
visually equivalent to \textit{Apollonius Polarity} when the null conic is a
circle or more generally the usual \textit{pole-polar reciprical relationship%
} for a conic section. Hence to construct the Polar and Pole tools I used
the inbuilt Polar tool in GeoGebra, but despite this I will describe their
construction.

\bigskip 

\textbf{Duals }

For a point $a$ draw two distinct interior lines through $a$ and call the
intersection points $\alpha ,\beta $, and $\gamma ,\delta $ respectfully.
This is always possible to do. Hence we have constucted a quadrilateral
whose vertices lie on the null conic and has diagonal point $a=\left( \alpha
\beta \right) \left( \gamma \delta \right) $. Let $d=\left( \alpha \gamma
\right) \left( \beta \delta \right) $ and $e=\left( \alpha \delta \right)
\left( \beta \gamma \right) $ be the other diagonal points. Then using the
remarkable fact from projective geometry; \textit{the line }$de$ \textit{%
does not depend on the two lines choosen through }$a$,\textit{\ (hence }$%
\alpha ,\beta ,\gamma $, \textit{and }$\delta $,\textit{) but only on }$a$%
\textit{\ itself.} So we say that $de=A\equiv a^{\perp }$ is the \textbf{%
dual line} of the point $a$, and conversely that $a=A^{\perp }$ is the 
\textbf{dual point} of the line $A$. 

The GeoGebra tool is called by the sequence \textit{Polar}$\left( a,c\right)
=A$ and conversely \textit{Pole}$\left( A,c\right) =a$.

$.$

\textbf{Altitudes}

For a non dual couple $\overline{aL}$ with $l=L^{\perp }\neq a$. The
Altitude line is $N=al$ since it is the unique line perpendicualr to $L$
through $a$, where the Alitidue point is $n=AL=N^{\perp }$.

The GeoGebra tool is called by the sequence \textit{AlititudeLine}$\left(
L,a,c\right) =N$ and \textit{AltitudePoint}$\left( L,a,c\right) =n$.

\textbf{Bases}

For the non dual couple $\overline{aL}$ with $l=L^{\perp }\neq a$. The Base
point is the meet $b=NL$, and the Base Line is the join $B=nl=b^{\perp }$.

The GeoGebra tool is called by the sequence \textit{BasePoint}$\left(
L,a,c\right) =b$ and  \textit{BaseLine}$\left( L,a,c\right) =B$.

\textbf{Parallels}

For the non dual couple $\overline{aL}$, the parallel line $P$ to $L$
through $a$ is the line through $a$ perpendicular to the perpendicular line
to $L$ through $a$, that is $P=na$. Dually the parallel point is $%
p=NA=P^{\perp }$.

The GeoGebra tool is called by the sequence \textit{ParallelLine}$\left(
L,a,c\right) =P$ and \textit{ParallelPoint}$\left( L,a,c\right) =p$.

\textbf{Reflections}\newline

For the reflection of a point $a$ in the point $b$ draw the line $ab$ and
also an interior line through $b$ distinct from $ab$, such that the
intersection points $\alpha ,\beta $ are not perpendicular to $a$. Draw the
line $a\alpha $, then this is an interior line and intersections $c$ at $%
\alpha $ and $\gamma $. Draw the interior line $b\gamma $ to get the null
point $\delta $, then $a^{\prime }=\left( \delta \beta \right) \left(
ab\right) $ is the \textbf{reflection point of }$a$\textbf{\ in the point }$b
$. Furthermore for a given point $a$ and $b$ the reflection of the dual $%
a^{\perp }=A$ is equivalent to the dual of the reflection of $a\,$, that is
if $a^{\perp }=A$ then $A^{\prime }=\left( a^{\prime }\right) ^{\perp }$,
and conversely. This equivalence of the reflectee point with it's dual is
also true for the reflecter point with it's dual. Hence we are able to
construct all reflection between points and lines through the above method
and taking duals.

The GeoGebra tool is called by the sequence \textit{Reflection\_PiP}$\left(
a,b,c\right) =a^{\prime }$, \textit{Reflection\_PiL}$\left( a,B,c\right)
=a^{\prime }$, \textit{Reflection\_LiP}$\left( A,b,c\right) =A^{\prime }$,
and \textit{Reflection\_LiL}$\left( A,B,c\right) =A^{\prime }$, where PiP
stands for \textit{Point in Point.}

\bigskip 

\textbf{Midpoints and lines}

For the side $\overline{ab}$, if both the pointd $a$ and $b$ are exterior
points then their duals $A$ and $B$ respectively, are interior lines. These
lines produce a quadrilateral whose points lie on the null conic with a
diagonal point equal to $AB$. Therefore the other diagonal points of the
quadrilateral lie on the line $ab$ and are infact the Midpoints $m$ and $%
m^{\prime }$. Else if $a$ and $b$ are both interior points then the lines $%
\left( AB\right) a$ and $\left( AB\right) b$ are interior lines who produce
a quadrilateral with the same properties as before. The Midlines $M$ for the
side $\overline{ab}$ are the duals to midpoints, $M=\left( AB\right)
m^{\prime }$ and $M^{\prime }=\left( AB\right) m$.

The GeoGebra tool is called by the sequence \textit{Midpoint}$\left(
a,b,c\right) =\left\{ m,m^{\prime }\right\} $ and \textit{Midlines}$\left(
a,b,c\right) =\left\{ M^{\prime },M\right\} $.

\textbf{Bilines and points}

For the vertex $\overline{AB}$, the situation for finding the Bilines and
Bipoint is precisely dual finding the Midpoints and Midlines of a side
described above.

The GeoGebra tool is called by the sequence \textit{Biline}$\left(
A,B,c\right) =\left\{ B,B^{\prime }\right\} $ and \textit{Bipoint}$\left(
A,B,c\right) =\left\{ b^{\prime },b\right\} $.

\textbf{Sydpoints and lines}

\end{document}
