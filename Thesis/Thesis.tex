
\documentclass[11pt]{article}
%%%%%%%%%%%%%%%%%%%%%%%%%%%%%%%%%%%%%%%%%%%%%%%%%%%%%%%%%%%%%%%%%%%%%%%%%%%%%%%%%%%%%%%%%%%%%%%%%%%%%%%%%%%%%%%%%%%%%%%%%%%%%%%%%%%%%%%%%%%%%%%%%%%%%%%%%%%%%%%%%%%%%%%%%%%%%%%%%%%%%%%%%%%%%%%%%%%%%%%%%%%%%%%%%%%%%%%%%%%%%%%%%%%%%%%%%%%%%%%%%%%%%%%%%%%%
\usepackage{amsfonts}
\usepackage{amssymb}
\usepackage{amsthm}
\usepackage{latexsym,amsmath}
\usepackage{graphicx}
\usepackage{afterpage}
\usepackage[top=1in, bottom=1in, left=1in, right=1in]{geometry}
\usepackage[utf8]{inputenc}
\usepackage{pgf,tikz}

\setcounter{MaxMatrixCols}{10}
%TCIDATA{OutputFilter=LATEX.DLL}
%TCIDATA{Version=5.50.0.2960}
%TCIDATA{<META NAME="SaveForMode" CONTENT="1">}
%TCIDATA{BibliographyScheme=Manual}
%TCIDATA{LastRevised=Monday, July 04, 2016 11:06:53}
%TCIDATA{<META NAME="GraphicsSave" CONTENT="32">}
%TCIDATA{ComputeDefs=
%$m_{{}}$
%}


\newtheorem{theorem}{Theorem}
\usetikzlibrary{arrows}
\pagestyle{empty}
\input{tcilatex}
\begin{document}

\author{Sebastian Blefari}
\date{}

\section{Bilinear Form}

Fundamental Theorem of Projective Geometry:\newline
We are able to use a transformation to send any four arbitrary non-collinear
points of a quadrangle to the four points; 
\begin{equation*}
a_{1}\equiv \lbrack 1:1:1],\;\;a_{2}\equiv \lbrack -1:-1:1],\;\;a_{3}\equiv
\lbrack 1:-1:1],\;\;a_{4}\equiv \lbrack -1:1:1].
\end{equation*}%
This transformation results in a change of Bilinear form to 
\begin{equation*}
\mathbf{A}\equiv 
\begin{bmatrix}
a & d & f \\ 
d & b & g \\ 
f & g & c%
\end{bmatrix}%
,\;\;\text{for some}\;\;a,b,c,d,e,f\in \text{Nat},
\end{equation*}%
with inverse 
\begin{equation*}
\mathbf{B}\equiv 
\begin{bmatrix}
g^{2}-bc & dc-fg & fb-dg \\ 
dc-fg & f^{2}-ac & ga-df \\ 
fb-dg & ga-df & d^{2}-ab%
\end{bmatrix}%
.
\end{equation*}%
We will refer to the quadrangle $\overline{a_{1}a_{2}a_{3}a_{4}}$ as the
standard quadrangle. The standard quadrangle has six sides $\overline{%
a_{i}a_{j}}$ for $i\neq j\in \lbrack 6]$, which determine the six lines 
\begin{align*}
L_{\{1,2\}}& \equiv a_{1}a_{2}\equiv \langle 1:-1:0\rangle , & L_{\{1,3\}}&
\equiv a_{1}a_{3}\equiv \langle 1:0:-1\rangle , & L_{\{1,4\}}& \equiv
a_{1}a_{4}\equiv \langle 0:1:-1\rangle  \\
L_{\{3,4\}}& \equiv a_{3}a_{4}\equiv \langle 1:1:0\rangle , & L_{\{2,4\}}&
\equiv a_{2}a_{4}\equiv \langle 1:0:1\rangle , & L_{\{2,3\}}& \equiv
a_{2}a_{3}\equiv \langle 0:1:1\rangle .
\end{align*}%
These six lines of the respective sides let us find the diagonal triangle $%
\overline{d_{1}d_{2}d_{3}}$ of the standard quadrangle, 
\begin{eqnarray*}
d_{\alpha } &\equiv &L_{\{1,2\}}L_{\{3,4\}}\equiv \lbrack 0:0:1],\;\; \\
d_{\beta } &\equiv &L_{\{1,3\}}L_{\{2,4\}}\equiv \lbrack 0:1:0],\;\; \\
d_{\gamma } &\equiv &L_{\{1,4\}}L_{\{2,3\}}\equiv \lbrack 1:0:0].
\end{eqnarray*}%
The labeling will become more obvious below. Define 
\begin{equation*}
D\equiv abc+2fdg-ag^{2}-bf^{2}-cd^{2}.
\end{equation*}%
Then indeed we have that 
\begin{align*}
det(A)& =det%
\begin{pmatrix}
a & d & f \\ 
d & b & g \\ 
f & g & c%
\end{pmatrix}%
=D,\;\;\text{and} \\
det(B)& =det%
\begin{pmatrix}
g^{2}-bc & dc-fg & fb-dg \\ 
dc-fg & f^{2}-ac & ga-df \\ 
fb-dg & ga-df & d^{2}-ab%
\end{pmatrix}%
=-D^{2}.
\end{align*}

Also define the variable

\begin{eqnarray*}
A_{1} &=&a_{1}\cdot a_{1}\equiv a+b+c+2\left( d+f+g\right) ,~\quad
A_{2}=a_{2}\cdot a_{2}\equiv a+b+c+2\left( d-f-g\right) ,~~ \\
A_{3} &=&a_{3}\cdot a_{3}\equiv a+b+c+2\left( -d+f-g\right)
,~~A_{4}=a_{4}\cdot a_{4}\equiv a+b+c+2(-d-f+g),
\end{eqnarray*}

in an effort to simplify the expressions in the following theorem.

\begin{theorem}[Quadrangle quadrances and spreads]
Using these coordinates described above, the quadrances of the quadrangle
are 
\begin{align*}
q(a_{1},a_{2})& =4\frac{c(a+b+2d)-(f+g)^{2}}{A_{1}A_{2}}, & q(a_{3},a_{4})&
=4\frac{c(a+b-2d)-(f-g)^{2}}{A_{3}A_{4}}, \\
q(a_{1},a_{3})& =4\frac{b(a+c+2f)-(d+g)^{2}}{A_{1}A_{3}}, & q(a_{2},a_{4})&
=4\frac{b(a+c-2f)-(d-g)^{2}}{A_{2}A_{4}}, \\
q(a_{1},a_{4})& =4\frac{a(b+c+2g)-(d+f)^{2}}{A_{1}A_{4}}, & q(a_{2},a_{3})&
=4\frac{a(b+c-2g)-(d-f)^{2}}{A_{2}A_{3}}
\end{align*}%
These numbers also satisfy 
\begin{align*}
1-q(a_{1},a_{2})=\frac{(c-b-a-2d)^{2}}{A_{1}A_{2}},& & 1-q(a_{3},a_{4})& =%
\frac{(c-b-a+2d)^{2}}{A_{3}A_{4}}, \\
1-q(a_{1},a_{3})=\frac{(a-b+c+2f)^{2}}{A_{1}A_{3}},& & 1-q(a_{2},a_{4})& =%
\frac{(a-b+c-2f)^{2}}{A_{2}A_{4}}, \\
1-q(a_{1},a_{4})=\frac{(b-a+c+2g)^{2}}{A_{1}A_{4}},& & 1-q(a_{2},a_{3})& =%
\frac{(b-a+c-2g)^{2}}{A_{2}A_{3}},
\end{align*}
\end{theorem}

\textit{Proof. }Computations give you these results. $\square $

\begin{theorem}[Side midpoints]
Suppose that $p_{1}$ and $p_{2}$ are non-null, non-perpendicular points,
forming a non-null side $\overline{p_{1}p_{2}}$. Then $\overline{p_{1}p_{2}}$
has a non-null midpoint $m$ precisely when $1-q(p_{1},p_{2})$ is a square,
and in this case there are exactly two perpendicular midpoints $m$.
\end{theorem}

\textit{Proof \ }We suppose that without loss of generality that $%
p_{1}=a_{1}\equiv \lbrack 1:1:1]$ and $p_{2}=a_{2}\equiv \lbrack -1:-1:1]$
so that by the spreads and quadrances theorem

\begin{equation*}
1-q(p_{1},p_{2})=\frac{(c-b-a-2d)^{2}}{A_{1}A_{2}}
\end{equation*}

By assumption each of the variables $A_{1}$ and $A_{2}$ are nonzero. An
arbitrary point $m$ on the line $L_{\{1,2\}}\equiv \left\langle
1:-1:0\right\rangle $ has the form $m=[x-y:x-y:x+y]$, which is null
precisely when 
\begin{equation*}
(a+b+2d)(x-y)^{2}+c(x+y)^{2}+2\left( f+g\right) (x^{2}-y^{2})=0
\end{equation*}
by the Null point theorem. Assuming that $m$ is non-null, we compute that

\begin{eqnarray*}
q(p_{1},m) &=&\frac{4y^{2}\left( c\left( a+b+d\right) -\left( f+g\right)
^{2}\right) }{A_{1}\left( (a+b+2d)(x-y)^{2}+c(x+y)^{2}+2\left( f+g\right)
(x^{2}-y^{2})\right) }, \\
q(p_{2},m) &=&\frac{4x^{2}\left( c\left( a+b+d\right) -\left( f+g\right)
^{2}\right) }{A_{2}\left( (a+b+2d)(x-y)^{2}+c(x+y)^{2}+2\left( f+g\right)
(x^{2}-y^{2})\right) }
\end{eqnarray*}%
\linebreak

By assumption $\overline{p_{1}p_{2}}$ is non-null, so by the Corollary to
the Null points/lines theorem, $c\left( a+b+d\right) -\left( f+g\right)
^{2}\neq 0$, and so the above expressions are equal precisely when 
\begin{equation*}
y^{2}A_{2}=x^{2}A_{1}
\end{equation*}
has a solution, which occurs precisely when $1-q(p_{1},p_{2})$ is a square.
In fact if

\begin{equation*}
\frac{1}{A_{1}A_{2}}=\sigma _{\left\{ 1,2\right\} }^{2}
\end{equation*}

then the two midpoints are

\begin{equation*}
m\equiv \left[ 1\pm \sigma _{\left\{ 1,2\right\} }A_{1}:1\pm \sigma
_{\left\{ 1,2\right\} }A_{1}:1\mp \sigma _{\left\{ 1,2\right\} }A_{1}\right]
\end{equation*}
and they are perpendicular, since

\begin{equation*}
m_{+}\mathbf{A}m_{-}^{T}=0
\end{equation*}

\bigskip by computations. $\square $

We denote the pair of midpoints as \textbf{opposites}. Where it follows that
the dual line $M$ of a midpoint $m$ is incident with the opposite midpoint.

\begin{theorem}[Vertex bilines]
Dual
\end{theorem}

We want to impose the conditions that all six sides $\overline{a_{i}a_{j}}$
for $i\neq j\in \lbrack 4]$ have midpoints. From the Side midpoint theorem
we know this is true precisely when $1-q(a_{i},a_{j})$ is a square. This
condition is equivalent to having natural numbers $\sigma _{\left\{
1,2\right\} },\sigma _{\left\{ 3,4\right\} },\sigma _{\left\{ 1,3\right\}
},\sigma _{\left\{ 2,4\right\} },\sigma _{\left\{ 1,4\right\} },\sigma
_{\left\{ 2,3\right\} }\in \text{Rat}$, satisfying the \textbf{quadratic
relations}

\begin{align}
\frac{1}{A_{1}A_{2}}& =\sigma _{\left\{ 1,2\right\} }^{2},\quad \frac{1}{%
A_{1}A_{3}}=\sigma _{\left\{ 1,3\right\} }^{2},\quad \frac{1}{A_{1}A_{4}}%
=\sigma _{\left\{ 1,4\right\} }^{2},  \label{Quadratic Relations} \\
\frac{1}{A_{3}A_{4}}& =\sigma _{\left\{ 3,4\right\} }^{2},\quad \frac{1}{%
A_{2}A_{4}}=\sigma _{\left\{ 2,4\right\} }^{2},\quad \frac{1}{A_{2}A_{3}}%
=\sigma _{\left\{ 2,3\right\} }^{2}
\end{align}%
\qquad Clearly all of $\sigma _{\left\{ 1,2\right\} },\sigma _{\left\{
3,4\right\} },\sigma _{\left\{ 1,3\right\} },\sigma _{\left\{ 2,4\right\}
},\sigma _{\left\{ 1,4\right\} },\sigma _{\left\{ 2,3\right\} }$ are
nonzero. These quadratic relations give rise to the \textbf{cubic relations}

\begin{eqnarray}
\frac{1}{A_{1}A_{2}A_{3}} &=&\sigma _{\left\{ 1,2\right\} }\sigma _{\left\{
2,3\right\} }\sigma _{\left\{ 1,3\right\} },\quad \frac{1}{A_{1}A_{2}A_{4}}%
=\sigma _{\left\{ 1,2\right\} }\sigma _{\left\{ 2,4\right\} }\sigma
_{\left\{ 1,4\right\} },\quad   \label{Cubic Relations} \\
\frac{1}{A_{1}A_{3}A_{4}} &=&\sigma _{\left\{ 1,3\right\} }\sigma _{\left\{
3,4\right\} }\sigma _{\left\{ 1,4\right\} },\quad \frac{1}{A_{2}A_{3}A_{4}}%
=\sigma _{\left\{ 2,3\right\} }\sigma _{\left\{ 3,4\right\} }\sigma
_{\left\{ 2,4\right\} },  \notag
\end{eqnarray}

From these relations we get

\begin{eqnarray*}
A_{1} &=&\frac{\sigma _{\left\{ 2,3\right\} }}{\sigma _{\left\{ 1,2\right\}
}\sigma _{\left\{ 1,3\right\} }}=\frac{\sigma _{\left\{ 2,4\right\} }}{%
\sigma _{\left\{ 1,2\right\} }\sigma _{\left\{ 1,4\right\} }}=\frac{\sigma
_{\left\{ 3,4\right\} }}{\sigma _{\left\{ 1,3\right\} }\sigma _{\left\{
1,4\right\} }}, \\
A_{2} &=&\frac{\sigma _{\left\{ 1,3\right\} }}{\sigma _{\left\{ 1,2\right\}
}\sigma _{\left\{ 2,3\right\} }}=\frac{\sigma _{\left\{ 1,4\right\} }}{%
\sigma _{\left\{ 1,2\right\} }\sigma _{\left\{ 2,4\right\} }}=\frac{\sigma
_{\left\{ 3,4\right\} }}{\sigma _{\left\{ 2,3\right\} }\sigma _{\left\{
2,4\right\} }}, \\
A_{3} &=&\frac{\sigma _{\left\{ 1,2\right\} }}{\sigma _{\left\{ 1,3\right\}
}\sigma _{\left\{ 2,3\right\} }}=\frac{\sigma _{\left\{ 1,4\right\} }}{%
\sigma _{\left\{ 1,3\right\} }\sigma _{\left\{ 3,4\right\} }}=\frac{\sigma
_{\left\{ 2,4\right\} }}{\sigma _{\left\{ 2,3\right\} }\sigma _{\left\{
3,4\right\} }}, \\
A_{4} &=&\frac{\sigma _{\left\{ 1,2\right\} }}{\sigma _{\left\{ 1,4\right\}
}\sigma _{\left\{ 2,4\right\} }}=\frac{\sigma _{\left\{ 1,3\right\} }}{%
\sigma _{\left\{ 1,4\right\} }\sigma _{\left\{ 3,4\right\} }}=\frac{\sigma
_{\left\{ 2,3\right\} }}{\sigma _{\left\{ 2,4\right\} }\sigma _{\left\{
3,4\right\} }}.
\end{eqnarray*}

Furthermore these in turn imply the relations

\begin{equation}
\frac{\sigma _{\left\{ 2,3\right\} }}{\sigma _{\left\{ 1,3\right\} }}=\frac{%
\sigma _{\left\{ 2,4\right\} }}{\sigma _{\left\{ 1,4\right\} }},\qquad \frac{%
\sigma _{\left\{ 2,3\right\} }}{\sigma _{\left\{ 1,2\right\} }}=\frac{\sigma
_{\left\{ 3,4\right\} }}{\sigma _{\left\{ 1,4\right\} }},\qquad \frac{\sigma
_{\left\{ 2,4\right\} }}{\sigma _{\left\{ 1,2\right\} }}=\frac{\sigma
_{\left\{ 3,4\right\} }}{\sigma _{\left\{ 1,3\right\} }},
\label{Sigma Relations}
\end{equation}

All these realtions have a strong correlation with the symmetries of four
objects, namely that described realtion is the division of four objects into
three pairs of two, that is the pairing%
\begin{equation*}
\left\{ \left\{ 1,2\right\} ,\left\{ 3,4\right\} \right\} ,~\left\{ \left\{
1,3\right\} ,\left\{ 2,4\right\} \right\} ,~\left\{ \left\{ 1,4\right\}
,\left\{ 2,3\right\} \right\}
\end{equation*}

\bigskip with respect to the order given. We will from now on refer to the
three pairings aboves as \textbf{pairing }$\alpha ,\beta $ and $\gamma $
respectfully.

\subsection{Midpoints}

By the Side midpoint theorem for a side $\overline{a_{i}a_{j}}$ where $%
a_{i}=[v_{i}]$ and $a_{j}=[v_{j}]$ we are able to normalise $v_{i}$ and $%
v_{j}$ such that $v_{i}^{2}=v_{j}^{2}$ giving the midpoints $m_{ij}\equiv
\lbrack v_{i}+v_{j}]$ and $m_{ji}\equiv \lbrack v_{i}-v_{j}]$ of $\overline{%
a_{i}a_{j}}$, where the ordering is arbitrary. In the end of the proof of
the Side midpoints Theorem we see that the midpoints for the side $\overline{%
a_{1}a_{2}}$ are $m\equiv \left[ 1\pm \sigma _{\left\{ 1,2\right\}
}A_{1}:1\pm \sigma _{\left\{ 1,2\right\} }A_{1}:1\mp \sigma _{\left\{
1,2\right\} }A_{1}\right] $, but from above these can be rewritten as 
\begin{equation*}
m\equiv \left[ \sigma _{\left\{ 1,3\right\} }\pm \sigma _{\left\{
2,3\right\} }:\sigma _{\left\{ 1,3\right\} }\pm \sigma _{\left\{ 2,3\right\}
}:\sigma _{\left\{ 1,3\right\} }\mp \sigma _{\left\{ 2,3\right\} }\right] =%
\left[ \sigma _{\left\{ 1,4\right\} }\pm \sigma _{\left\{ 2,4\right\}
}:\sigma _{\left\{ 1,4\right\} }\pm \sigma _{\left\{ 2,4\right\} }:\sigma
_{\left\{ 1,4\right\} }\mp \sigma _{\left\{ 2,4\right\} }\right] ,
\end{equation*}%
or some other combination with respect to the $\alpha $ \textit{sigma
relations}. 

Hence the side midpoints of the quadrangle $\overline{a_{1}a_{2}a_{3}a_{4}}$
have the form

\begin{eqnarray*}
m_{12} &\equiv &\left[ 1-\sigma _{\left\{ 1,2\right\} }A_{1}:1-\sigma
_{\left\{ 1,2\right\} }A_{1}:1+\sigma _{\left\{ 1,2\right\} }A_{1}\right] =%
\left[ \sigma _{\left\{ 1,3\right\} }-\sigma _{\left\{ 2,3\right\} }:\sigma
_{\left\{ 1,3\right\} }-\sigma _{\left\{ 2,3\right\} }:\sigma _{\left\{
1,3\right\} }+\sigma _{\left\{ 2,3\right\} }\right]  \\
m_{21} &\equiv &\left[ 1+\sigma _{\left\{ 1,2\right\} }A_{1}:1+\sigma
_{\left\{ 1,2\right\} }A_{1}:1-\sigma _{\left\{ 1,2\right\} }A_{1}\right] =%
\left[ \sigma _{\left\{ 1,4\right\} }+\sigma _{\left\{ 2,4\right\} }:\sigma
_{\left\{ 1,4\right\} }+\sigma _{\left\{ 2,4\right\} }:\sigma _{\left\{
1,4\right\} }-\sigma _{\left\{ 2,4\right\} }\right]  \\
m_{34} &\equiv &\left[ 1-\sigma _{\left\{ 3,4\right\} }A_{3}:-1+\sigma
_{\left\{ 3,4\right\} }A_{3}:1+\sigma _{\left\{ 3,4\right\} }A_{3}\right] =%
\left[ \sigma _{\left\{ 1,3\right\} }-\sigma _{\left\{ 1,4\right\} }:-\sigma
_{\left\{ 1,3\right\} }+\sigma _{\left\{ 1,4\right\} }:\sigma _{\left\{
1,3\right\} }+\sigma _{\left\{ 1,4\right\} }\right]  \\
m_{43} &\equiv &\left[ 1+\sigma _{\left\{ 3,4\right\} }A_{3}:-1-\sigma
_{\left\{ 3,4\right\} }A_{3}:1-\sigma _{\left\{ 3,4\right\} }A_{3}\right] =%
\left[ \sigma _{\left\{ 2,3\right\} }+\sigma _{\left\{ 2,4\right\} }:-\sigma
_{\left\{ 2,3\right\} }-\sigma _{\left\{ 2,4\right\} }:\sigma _{\left\{
2,3\right\} }-\sigma _{\left\{ 2,4\right\} }\right]  \\
m_{13} &\equiv &\left[ 1+\sigma _{\left\{ 1,3\right\} }A_{1}:1-\sigma
_{\left\{ 1,3\right\} }A_{1}:1+\sigma _{\left\{ 1,3\right\} }A_{1}\right] =%
\left[ \sigma _{\left\{ 1,2\right\} }+\sigma _{\left\{ 2,3\right\} }:\sigma
_{\left\{ 1,2\right\} }-\sigma _{\left\{ 2,3\right\} }:\sigma _{\left\{
1,2\right\} }+\sigma _{\left\{ 2,3\right\} }\right]  \\
m_{31} &\equiv &\left[ 1-\sigma _{\left\{ 1,3\right\} }A_{1}:1+\sigma
_{\left\{ 1,3\right\} }A_{1}:1-\sigma _{\left\{ 1,3\right\} }A_{1}\right] =%
\left[ \sigma _{\left\{ 1,4\right\} }-\sigma _{\left\{ 3,4\right\} }:\sigma
_{\left\{ 1,4\right\} }+\sigma _{\left\{ 3,4\right\} }:\sigma _{\left\{
1,4\right\} }-\sigma _{\left\{ 3,4\right\} }\right] 
\end{eqnarray*}

\begin{eqnarray*}
m_{24} &\equiv &\left[ -1-\sigma _{\left\{ 2,4\right\} }A_{2}:-1+\sigma
_{\left\{ 2,4\right\} }A_{2}:1+\sigma _{\left\{ 2,4\right\} }A_{2}\right] =%
\left[ -\sigma _{\left\{ 1,2\right\} }-\sigma _{\left\{ 1,4\right\}
}:-\sigma _{\left\{ 1,2\right\} }+\sigma _{\left\{ 1,4\right\} }:\sigma
_{\left\{ 1,2\right\} }+\sigma _{\left\{ 1,4\right\} }\right]  \\
m_{42} &\equiv &\left[ -1+\sigma _{\left\{ 2,4\right\} }A_{2}:-1-\sigma
_{\left\{ 2,4\right\} }A_{2}:1-\sigma _{\left\{ 2,4\right\} }A_{2}\right] =%
\left[ -\sigma _{\left\{ 2,3\right\} }+\sigma _{\left\{ 3,4\right\}
}:-\sigma _{\left\{ 2,3\right\} }-\sigma _{\left\{ 3,4\right\} }:\sigma
_{\left\{ 2,3\right\} }-\sigma _{\left\{ 3,4\right\} }\right]  \\
m_{14} &\equiv &\left[ 1-\sigma _{\left\{ 1,4\right\} }A_{1}:1+\sigma
_{\left\{ 1,4\right\} }A_{1}:1+\sigma _{\left\{ 1,4\right\} }A_{1}\right] =%
\left[ \sigma _{\left\{ 1,4\right\} }-\sigma _{\left\{ 2,4\right\} }:\sigma
_{\left\{ 1,4\right\} }+\sigma _{\left\{ 2,4\right\} }:\sigma _{\left\{
1,4\right\} }+\sigma _{\left\{ 2,4\right\} }\right]  \\
m_{41} &\equiv &\left[ 1+\sigma _{\left\{ 1,4\right\} }A_{1}:1-\sigma
_{\left\{ 1,4\right\} }A_{1}:1-\sigma _{\left\{ 1,4\right\} }A_{1}\right] =%
\left[ \sigma _{\left\{ 1,3\right\} }+\sigma _{\left\{ 3,4\right\} }:\sigma
_{\left\{ 1,3\right\} }-\sigma _{\left\{ 3,4\right\} }:\sigma _{\left\{
1,3\right\} }-\sigma _{\left\{ 3,4\right\} }\right]  \\
m_{23} &\equiv &\left[ -1+\sigma _{\left\{ 2,3\right\} }A_{2}:-1-\sigma
_{\left\{ 2,3\right\} }A_{2}:1+\sigma _{\left\{ 2,3\right\} }A_{2}\right] =%
\left[ -\sigma _{\left\{ 2,3\right\} }+\sigma _{\left\{ 1,3\right\}
}:-\sigma _{\left\{ 2,3\right\} }-\sigma _{\left\{ 1,3\right\} }:\sigma
_{\left\{ 2,3\right\} }+\sigma _{\left\{ 1,3\right\} }\right]  \\
m_{32} &\equiv &\left[ -1-\sigma _{\left\{ 2,3\right\} }A_{2}:-1+\sigma
_{\left\{ 2,3\right\} }A_{2}:1-\sigma _{\left\{ 2,3\right\} }A_{2}\right] =%
\left[ -\sigma _{\left\{ 2,4\right\} }-\sigma _{\left\{ 3,4\right\}
}:-\sigma _{\left\{ 2,4\right\} }+\sigma _{\left\{ 3,4\right\} }:\sigma
_{\left\{ 2,4\right\} }-\sigma _{\left\{ 3,4\right\} }\right] 
\end{eqnarray*}

These side midpoints have corresponding \textit{side midlines}, which are
precisely the duals to each side midpoint. That is they are given by the
matrix multiplications $M_{ij}\equiv \mathbf{A}m_{ij}^{T},$ which highlights
the opposite relations aluuded to above. 

There is a natural divison of the quadrangle $\overline{a_{1}a_{2}a_{3}a_{4}}
$ into four distinct triangles triangles $\overline{a_{1}a_{2}a_{3}},%
\overline{a_{1}a_{2}a_{4}},\overline{a_{1}a_{3}a_{4}}$ and $\overline{%
a_{2}a_{3}a_{4}}$,

\begin{theorem}[Circumlines and circumcenters]
Midpoints $m_{ij}$ for $i\neq j\in \left\{ 1,2,3\right\} $ are collinear
three at a time, lying on four distinct \textbf{Circumlines} $C_{1}^{\left(
4\right) },C_{2}^{\left( 4\right) },C_{2}^{\left( 4\right) },$ and $%
C_{3}^{\left( 4\right) }.$ Midlines $M_{ij}$ for $i\neq j\in \left\{
1,2,3\right\} $ of the triangle $\overline{a_{1}a_{2}a_{3}}$ are concurrent
three at a time, meeting at four distinct \textbf{Circumcenters} $%
c_{1}^{\left( 4\right) },c_{2}^{\left( 4\right) },c_{3}^{\left( 4\right) }$
and $c_{4}^{\left( 4\right) }$\textbf{.}
\end{theorem}

The following triples of midpoints $m_{ij}$ for $i\neq j\in \left\{
1,2,3\right\} $ are colinear:

\begin{eqnarray*}
m_{21} &\equiv &\left[ \sigma _{\left\{ 1,3\right\} }+\sigma _{\left\{
2,3\right\} }:\sigma _{\left\{ 1,3\right\} }+\sigma _{\left\{ 2,3\right\}
}:\sigma _{\left\{ 1,3\right\} }-\sigma _{\left\{ 2,3\right\} }\right] , \\
m_{31} &\equiv &\left[ \sigma _{\left\{ 1,2\right\} }-\sigma _{\left\{
2,3\right\} }:\sigma _{\left\{ 1,2\right\} }+\sigma _{\left\{ 2,3\right\}
}:\sigma _{\left\{ 1,2\right\} }-\sigma _{\left\{ 2,3\right\} }\right] , \\
m_{32} &\equiv &\left[ -\sigma _{\left\{ 2,3\right\} }-\sigma _{\left\{
1,3\right\} }:-\sigma _{\left\{ 2,3\right\} }+\sigma _{\left\{ 1,3\right\}
}:\sigma _{\left\{ 2,3\right\} }-\sigma _{\left\{ 1,3\right\} }\right] 
\end{eqnarray*}

on $C_{1}^{\left( 4\right) }\equiv \left\langle \sigma _{\left\{ 1,2\right\}
}-\sigma _{\left\{ 1,3\right\} }:-\sigma _{\left\{ 1,2\right\} }+\sigma
_{\left\{ 2,3\right\} }:\sigma _{\left\{ 1,3\right\} }+\sigma _{\left\{
2,3\right\} }\right\rangle ,$

\begin{eqnarray*}
m_{21} &\equiv &\left[ \sigma _{\left\{ 1,3\right\} }+\sigma _{\left\{
2,3\right\} }:\sigma _{\left\{ 1,3\right\} }+\sigma _{\left\{ 2,3\right\}
}:\sigma _{\left\{ 1,3\right\} }-\sigma _{\left\{ 2,3\right\} }\right] , \\
m_{13} &\equiv &\left[ \sigma _{\left\{ 1,2\right\} }+\sigma _{\left\{
2,3\right\} }:\sigma _{\left\{ 1,2\right\} }-\sigma _{\left\{ 2,3\right\}
}:\sigma _{\left\{ 1,2\right\} }+\sigma _{\left\{ 2,3\right\} }\right] , \\
m_{23} &\equiv &\left[ -\sigma _{\left\{ 2,3\right\} }+\sigma _{\left\{
1,3\right\} }:-\sigma _{\left\{ 2,3\right\} }-\sigma _{\left\{ 1,3\right\}
}:\sigma _{\left\{ 2,3\right\} }+\sigma _{\left\{ 1,3\right\} }\right] 
\end{eqnarray*}

on $C_{2}^{\left( 4\right) }\equiv \left\langle \sigma _{\left\{ 1,2\right\}
}+\sigma _{\left\{ 1,3\right\} }:-\sigma _{\left\{ 1,2\right\} }-\sigma
_{\left\{ 2,3\right\} }:-\sigma _{\left\{ 1,3\right\} }-\sigma _{\left\{
2,3\right\} }\right\rangle ,$

\begin{eqnarray*}
m_{12} &\equiv &\left[ \sigma _{\left\{ 1,3\right\} }-\sigma _{\left\{
2,3\right\} }:\sigma _{\left\{ 1,3\right\} }-\sigma _{\left\{ 2,3\right\}
}:\sigma _{\left\{ 1,3\right\} }+\sigma _{\left\{ 2,3\right\} }\right] , \\
m_{31} &\equiv &\left[ \sigma _{\left\{ 1,2\right\} }-\sigma _{\left\{
2,3\right\} }:\sigma _{\left\{ 1,2\right\} }+\sigma _{\left\{ 2,3\right\}
}:\sigma _{\left\{ 1,2\right\} }-\sigma _{\left\{ 2,3\right\} }\right] , \\
m_{23} &\equiv &\left[ -\sigma _{\left\{ 2,3\right\} }+\sigma _{\left\{
1,3\right\} }:-\sigma _{\left\{ 2,3\right\} }-\sigma _{\left\{ 1,3\right\}
}:\sigma _{\left\{ 2,3\right\} }+\sigma _{\left\{ 1,3\right\} }\right] 
\end{eqnarray*}

on $C_{3}^{\left( 4\right) }\equiv \left\langle -\sigma _{\left\{
1,2\right\} }-\sigma _{\left\{ 1,3\right\} }:\sigma _{\left\{ 1,2\right\}
}-\sigma _{\left\{ 2,3\right\} }:\sigma _{\left\{ 1,3\right\} }-\sigma
_{\left\{ 2,3\right\} }\right\rangle ,$

\begin{eqnarray*}
m_{12} &\equiv &\left[ \sigma _{\left\{ 1,3\right\} }-\sigma _{\left\{
2,3\right\} }:\sigma _{\left\{ 1,3\right\} }-\sigma _{\left\{ 2,3\right\}
}:\sigma _{\left\{ 1,3\right\} }+\sigma _{\left\{ 2,3\right\} }\right] , \\
m_{13} &\equiv &\left[ \sigma _{\left\{ 1,2\right\} }+\sigma _{\left\{
2,3\right\} }:\sigma _{\left\{ 1,2\right\} }-\sigma _{\left\{ 2,3\right\}
}:\sigma _{\left\{ 1,2\right\} }+\sigma _{\left\{ 2,3\right\} }\right] , \\
m_{32} &\equiv &\left[ -\sigma _{\left\{ 2,3\right\} }-\sigma _{\left\{
1,3\right\} }:-\sigma _{\left\{ 2,3\right\} }+\sigma _{\left\{ 1,3\right\}
}:\sigma _{\left\{ 2,3\right\} }-\sigma _{\left\{ 1,3\right\} }\right] 
\end{eqnarray*}

on $C_{4}^{\left( 4\right) }\equiv \left\langle -\sigma _{\left\{
1,2\right\} }+\sigma _{\left\{ 1,3\right\} }:\sigma _{\left\{ 1,2\right\}
}+\sigma _{\left\{ 2,3\right\} }:-\sigma _{\left\{ 1,3\right\} }+\sigma
_{\left\{ 2,3\right\} }\right\rangle .$

This is checked by computing 

\begin{eqnarray*}
&&\det \left( 
\begin{array}{ccc}
\sigma _{\left\{ 1,3\right\} }+\sigma _{\left\{ 2,3\right\} } & \sigma
_{\left\{ 1,3\right\} }+\sigma _{\left\{ 2,3\right\} } & \sigma _{\left\{
1,3\right\} }-\sigma _{\left\{ 2,3\right\} } \\ 
\sigma _{\left\{ 1,2\right\} }-\sigma _{\left\{ 2,3\right\} } & \sigma
_{\left\{ 1,2\right\} }+\sigma _{\left\{ 2,3\right\} } & \sigma _{\left\{
1,2\right\} }-\sigma _{\left\{ 2,3\right\} } \\ 
-\sigma _{\left\{ 2,3\right\} }-\sigma _{\left\{ 1,3\right\} } & -\sigma
_{\left\{ 2,3\right\} }+\sigma _{\left\{ 1,3\right\} } & \sigma _{\left\{
2,3\right\} }-\sigma _{\left\{ 1,3\right\} }%
\end{array}%
\right)  \\
&=&\det \left( 
\begin{array}{ccc}
\sigma _{\left\{ 1,3\right\} }+\sigma _{\left\{ 2,3\right\} } & \sigma
_{\left\{ 1,3\right\} }+\sigma _{\left\{ 2,3\right\} } & \sigma _{\left\{
1,3\right\} }-\sigma _{\left\{ 2,3\right\} } \\ 
\sigma _{\left\{ 1,2\right\} }+\sigma _{\left\{ 2,3\right\} } & \sigma
_{\left\{ 1,2\right\} }-\sigma _{\left\{ 2,3\right\} } & \sigma _{\left\{
1,2\right\} }+\sigma _{\left\{ 2,3\right\} } \\ 
-\sigma _{\left\{ 2,3\right\} }+\sigma _{\left\{ 1,3\right\} } & -\sigma
_{\left\{ 2,3\right\} }-\sigma _{\left\{ 1,3\right\} } & \sigma _{\left\{
2,3\right\} }+\sigma _{\left\{ 1,3\right\} }%
\end{array}%
\right)  \\
&=&\det \left( 
\begin{array}{ccc}
\sigma _{\left\{ 1,3\right\} }-\sigma _{\left\{ 2,3\right\} } & \sigma
_{\left\{ 1,3\right\} }-\sigma _{\left\{ 2,3\right\} } & \sigma _{\left\{
1,3\right\} }+\sigma _{\left\{ 2,3\right\} } \\ 
\sigma _{\left\{ 1,2\right\} }-\sigma _{\left\{ 2,3\right\} } & \sigma
_{\left\{ 1,2\right\} }+\sigma _{\left\{ 2,3\right\} } & \sigma _{\left\{
1,2\right\} }-\sigma _{\left\{ 2,3\right\} } \\ 
-\sigma _{\left\{ 2,3\right\} }+\sigma _{\left\{ 1,3\right\} } & -\sigma
_{\left\{ 2,3\right\} }-\sigma _{\left\{ 1,3\right\} } & \sigma _{\left\{
2,3\right\} }+\sigma _{\left\{ 1,3\right\} }%
\end{array}%
\right)  \\
&=&\det \left( 
\begin{array}{ccc}
\sigma _{\left\{ 1,3\right\} }-\sigma _{\left\{ 2,3\right\} } & \sigma
_{\left\{ 1,3\right\} }-\sigma _{\left\{ 2,3\right\} } & \sigma _{\left\{
1,3\right\} }+\sigma _{\left\{ 2,3\right\} } \\ 
\sigma _{\left\{ 1,2\right\} }+\sigma _{\left\{ 2,3\right\} } & \sigma
_{\left\{ 1,2\right\} }-\sigma _{\left\{ 2,3\right\} } & \sigma _{\left\{
1,2\right\} }+\sigma _{\left\{ 2,3\right\} } \\ 
-\sigma _{\left\{ 2,3\right\} }-\sigma _{\left\{ 1,3\right\} } & -\sigma
_{\left\{ 2,3\right\} }+\sigma _{\left\{ 1,3\right\} } & \sigma _{\left\{
2,3\right\} }-\sigma _{\left\{ 1,3\right\} }%
\end{array}%
\right) =0.
\end{eqnarray*}

The corresponding meets are 

\begin{eqnarray*}
&&\left[ \sigma _{\left\{ 1,3\right\} }+\sigma _{\left\{ 2,3\right\}
}:\sigma _{\left\{ 1,3\right\} }+\sigma _{\left\{ 2,3\right\} }:\sigma
_{\left\{ 1,3\right\} }-\sigma _{\left\{ 2,3\right\} }\right] \times \left[
\sigma _{\left\{ 1,2\right\} }-\sigma _{\left\{ 2,3\right\} }:\sigma
_{\left\{ 1,2\right\} }+\sigma _{\left\{ 2,3\right\} }:\sigma _{\left\{
1,2\right\} }-\sigma _{\left\{ 2,3\right\} }\right]  \\
&=&\left\langle \sigma _{\left\{ 1,2\right\} }-\sigma _{\left\{ 1,3\right\}
}:-\sigma _{\left\{ 1,2\right\} }+\sigma _{\left\{ 2,3\right\} }:\sigma
_{\left\{ 1,3\right\} }+\sigma _{\left\{ 2,3\right\} }\right\rangle \equiv
C_{1}^{\left( 4\right) },
\end{eqnarray*}

and similarly for the other circumlines. The situation with midlines $M_{ij}$
for $i\neq j\in \left\{ 1,2,3\right\} $ is precisely dual. $\square $

\bigskip 

The Circumlines for the triangles $\overline{a_{1}a_{2}a_{4}}$, $\overline{%
a_{1}a_{3}a_{4}}$, and $\overline{a_{2}a_{3}a_{4}}$, are as follows

\begin{eqnarray*}
m_{21} &\equiv &\left[ \sigma _{\left\{ 1,3\right\} }+\sigma _{\left\{
2,3\right\} }:\sigma _{\left\{ 1,3\right\} }+\sigma _{\left\{ 2,3\right\}
}:\sigma _{\left\{ 1,3\right\} }-\sigma _{\left\{ 2,3\right\} }\right] , \\
m_{24} &\equiv &\left[ \sigma _{\left\{ 1,2\right\} }-\sigma _{\left\{
2,3\right\} }:\sigma _{\left\{ 1,2\right\} }+\sigma _{\left\{ 2,3\right\}
}:\sigma _{\left\{ 1,2\right\} }-\sigma _{\left\{ 2,3\right\} }\right] , \\
m_{14} &\equiv &\left[ -\sigma _{\left\{ 2,3\right\} }-\sigma _{\left\{
1,3\right\} }:-\sigma _{\left\{ 2,3\right\} }+\sigma _{\left\{ 1,3\right\}
}:\sigma _{\left\{ 2,3\right\} }-\sigma _{\left\{ 1,3\right\} }\right] 
\end{eqnarray*}

on $C_{1}^{\left( 3\right) }\equiv \left\langle \sigma _{\left\{ 1,2\right\}
}-\sigma _{\left\{ 1,3\right\} }:-\sigma _{\left\{ 1,2\right\} }+\sigma
_{\left\{ 2,3\right\} }:\sigma _{\left\{ 1,3\right\} }+\sigma _{\left\{
2,3\right\} }\right\rangle ,$

\begin{eqnarray*}
m_{21} &\equiv &\left[ \sigma _{\left\{ 1,3\right\} }+\sigma _{\left\{
2,3\right\} }:\sigma _{\left\{ 1,3\right\} }+\sigma _{\left\{ 2,3\right\}
}:\sigma _{\left\{ 1,3\right\} }-\sigma _{\left\{ 2,3\right\} }\right] , \\
m_{42} &\equiv &\left[ \sigma _{\left\{ 1,2\right\} }+\sigma _{\left\{
2,3\right\} }:\sigma _{\left\{ 1,2\right\} }-\sigma _{\left\{ 2,3\right\}
}:\sigma _{\left\{ 1,2\right\} }+\sigma _{\left\{ 2,3\right\} }\right] , \\
m_{41} &\equiv &\left[ -\sigma _{\left\{ 2,3\right\} }+\sigma _{\left\{
1,3\right\} }:-\sigma _{\left\{ 2,3\right\} }-\sigma _{\left\{ 1,3\right\}
}:\sigma _{\left\{ 2,3\right\} }+\sigma _{\left\{ 1,3\right\} }\right] 
\end{eqnarray*}

on $C_{2}^{\left( 3\right) }\equiv \left\langle \sigma _{\left\{ 1,2\right\}
}+\sigma _{\left\{ 1,3\right\} }:-\sigma _{\left\{ 1,2\right\} }-\sigma
_{\left\{ 2,3\right\} }:-\sigma _{\left\{ 1,3\right\} }-\sigma _{\left\{
2,3\right\} }\right\rangle ,$

\begin{eqnarray*}
m_{12} &\equiv &\left[ \sigma _{\left\{ 1,3\right\} }-\sigma _{\left\{
2,3\right\} }:\sigma _{\left\{ 1,3\right\} }-\sigma _{\left\{ 2,3\right\}
}:\sigma _{\left\{ 1,3\right\} }+\sigma _{\left\{ 2,3\right\} }\right] , \\
m_{24} &\equiv &\left[ \sigma _{\left\{ 1,2\right\} }-\sigma _{\left\{
2,3\right\} }:\sigma _{\left\{ 1,2\right\} }+\sigma _{\left\{ 2,3\right\}
}:\sigma _{\left\{ 1,2\right\} }-\sigma _{\left\{ 2,3\right\} }\right] , \\
m_{41} &\equiv &\left[ -\sigma _{\left\{ 2,3\right\} }+\sigma _{\left\{
1,3\right\} }:-\sigma _{\left\{ 2,3\right\} }-\sigma _{\left\{ 1,3\right\}
}:\sigma _{\left\{ 2,3\right\} }+\sigma _{\left\{ 1,3\right\} }\right] 
\end{eqnarray*}

on $C_{3}^{\left( 3\right) }\equiv \left\langle -\sigma _{\left\{
1,2\right\} }-\sigma _{\left\{ 1,3\right\} }:\sigma _{\left\{ 1,2\right\}
}-\sigma _{\left\{ 2,3\right\} }:\sigma _{\left\{ 1,3\right\} }-\sigma
_{\left\{ 2,3\right\} }\right\rangle ,$

\begin{eqnarray*}
m_{12} &\equiv &\left[ \sigma _{\left\{ 1,3\right\} }-\sigma _{\left\{
2,3\right\} }:\sigma _{\left\{ 1,3\right\} }-\sigma _{\left\{ 2,3\right\}
}:\sigma _{\left\{ 1,3\right\} }+\sigma _{\left\{ 2,3\right\} }\right] , \\
m_{42} &\equiv &\left[ \sigma _{\left\{ 1,2\right\} }+\sigma _{\left\{
2,3\right\} }:\sigma _{\left\{ 1,2\right\} }-\sigma _{\left\{ 2,3\right\}
}:\sigma _{\left\{ 1,2\right\} }+\sigma _{\left\{ 2,3\right\} }\right] , \\
m_{14} &\equiv &\left[ -\sigma _{\left\{ 2,3\right\} }-\sigma _{\left\{
1,3\right\} }:-\sigma _{\left\{ 2,3\right\} }+\sigma _{\left\{ 1,3\right\}
}:\sigma _{\left\{ 2,3\right\} }-\sigma _{\left\{ 1,3\right\} }\right] 
\end{eqnarray*}

on $C_{4}^{\left( 4\right) }\equiv \left\langle -\sigma _{\left\{
1,2\right\} }+\sigma _{\left\{ 1,3\right\} }:\sigma _{\left\{ 1,2\right\}
}+\sigma _{\left\{ 2,3\right\} }:-\sigma _{\left\{ 1,3\right\} }+\sigma
_{\left\{ 2,3\right\} }\right\rangle .$

\bigskip 

\begin{eqnarray*}
c_{41} &\equiv &M_{21}M_{32}=M_{32}M_{31},\qquad c_{42}\equiv
M_{21}M_{23}=M_{23}M_{13},\qquad c_{43}\equiv
M_{12}M_{23}=M_{23}M_{31},\qquad c_{44}\equiv M_{12}M_{32}=M_{32}M_{13}, \\
c_{31} &\equiv &M_{21}M_{24}=M_{24}M_{14},\qquad c_{32}\equiv
M_{21}M_{42}=M_{42}M_{41},\qquad c_{33}\equiv
M_{12}M_{24}=M_{24}M_{41},\qquad c_{34}\equiv M_{12}M_{42}=M_{42}M_{14}, \\
c_{21} &\equiv &M_{32}M_{34}=M_{34}M_{14},\qquad c_{22}\equiv
M_{13}M_{34}=M_{34}M_{41,}\qquad c_{23}\equiv
M_{31}M_{43}=M_{43}M_{41},\qquad c_{24}\equiv M_{13}M_{43}=M_{43}M_{14,} \\
c_{11} &\equiv &M_{32}M_{34}=M_{34}M_{24},\qquad c_{12}\equiv
M_{23}M_{34}=M_{34}M_{42},\qquad c_{13}\equiv
M_{23}M_{43}=M_{43}M_{24},\qquad c_{14}\equiv M_{32}M_{43}=M_{43}M_{42}
\end{eqnarray*}
\pagebreak 

\section{\protect\bigskip Introduction}

Throughout this thesis definitions will be given in bold and italics will be
reserved for emphasis.\newline

\textbf{Universal Geometry} Through relatively recent developements in the
field of geometry, Norman Wildberger has shown that hyperbolic geometry can
be considered as an agebraic projective geometry. In the following pages we
will aim to define the fundamental objects of hyperbolic geometry and how
they interact with each other.\newline

\textbf{Projective Geometry} Universal Projective geometry is a geometry in
the space of lines through the origin of a vector space with a metrical
structure given by a symmetric bilinear form.\newline
The complete algebraic nature of Universal geometry implies that we have an
algebraic construction of Projective geometry. The focus of this chapter is
to introduce the main objects of Universal Projective geometry and define
their incidence relations in such a way to induce a complete duality between
points and lines in this projective setting. This concept of complete
duality is a defining characteristic of Projective geometry.\newline
Universal geometry is given as an algebraic geometry, where the algebraic
framework for Universal Projective geometry is given to us through \textit{%
projective linear algebra}. Projective linear algebra is much like normal
linear algebra but vectors and matrices are only defined up to non-zero
scalar multiples. In this thesis the convention of writing the usual affine
vectors and matrices with round brackets and projective vectors and matrices
with square brackets. Hence for a given row vector $v=(1\;\;2\;\;3)$ we
denote the associated projective vector $a=[v]$ as $a=[1\;\;2\;\;3]$ which
by definition also equal to $[-1\;\;-2\;\;-3]$ or to $[2\;\;4\;\;6]$. We
will also use bold labels to denote projective matrices: for for ordinary
matrices 
\begin{equation*}
A=%
\begin{pmatrix}
1 & 1 & 2 \\ 
0 & 2 & 0 \\ 
0 & 0 & 1%
\end{pmatrix}
\;\;\text{and}\;\; B=%
\begin{pmatrix}
2 & -1 & -4 \\ 
0 & 1 & 0 \\ 
0 & 0 & 2%
\end{pmatrix}%
,
\end{equation*}
the associated projective matrices are 
\begin{align*}
\mathbf{A}=%
\begin{bmatrix}
1 & 1 & 2 \\ 
0 & 2 & 0 \\ 
0 & 0 & 1%
\end{bmatrix}%
= 
\begin{bmatrix}
-1 & -1 & -2 \\ 
0 & -2 & 0 \\ 
0 & 0 & -1%
\end{bmatrix}%
, \\
\mathbf{B}=%
\begin{bmatrix}
2 & -1 & -4 \\ 
0 & 1 & 0 \\ 
0 & 0 & 2%
\end{bmatrix}%
= 
\begin{bmatrix}
4 & -2 & -8 \\ 
0 & 2 & 0 \\ 
0 & 0 & 4%
\end{bmatrix}%
\end{align*}
Where infact $\mathbf{A}^{-1}=\mathbf{B}$ in the projective setting, as
scalar multiplies can be disregarded. It turns out that addition of
projective matrices is not well defined but multiplication is.\newline
It is now important to introduce the main objects of projective geometry in
a way that is consistent. A \textbf{(projective) point} is a \textit{non-zero%
} projective row vector $a$ and will be written in either of two ways: 
\begin{equation*}
a\equiv[x\;y\;z]\equiv[x:y:z].
\end{equation*}
A \textbf{(projective) line} is a \textit{non-zero} projective column vector 
$L$ written as 
\begin{equation*}
L\equiv%
\begin{bmatrix}
l \\ 
m \\ 
n%
\end{bmatrix}%
\equiv\langle l:m:n\rangle.
\end{equation*}
For the point $a=[x:y:z]$ and line $L=\langle l:m:n\rangle$ we say they are 
\textbf{incident} precisely when 
\begin{equation}
aL\equiv[x\;y\;z]%
\begin{bmatrix}
l \\ 
m \\ 
n%
\end{bmatrix}%
\equiv0.
\end{equation}
Three or more lines are \textbf{concurrent} precisely when they are all
incident with a point $a$, and dually three or more points are \textbf{%
collinear} precisely when they are all incident with a line $L$.\newline
The \textbf{join} $a_{1}a_{2}$ of distinct points $a_{1}\equiv[%
x_{1}:y_{1}:z_{1}]$ and $a_{2}\equiv[x_{2}:y_{2}:z_{2}]$ is the line 
\begin{equation*}
a_{1}a_{2}\equiv[x_{1}:y_{1}:z_{1}]\times[x_{2}:y_{2}:z_{2}]\equiv\langle
y_{1}z_{2}-z_{1}y_{2}:z_{1}x_{2}-x_{1}z_{2}:x_{1}y_{2}-y_{1}x_{2}\rangle.
\end{equation*}
The \textbf{meet} $L_{1}L_{2}$ of two distinct points $L_{1}\equiv\langle
l_{1}:m_{1}:n_{1}\rangle$ and $l_{2}\equiv\langle l_{2}:m_{2}:n_{2}\rangle$
is the point 
\begin{equation*}
L_{1}L_{2}\equiv\langle l_{1}:m_{1}:n_{1}\rangle\times\langle
l_{2}:m_{2}:n_{2}\rangle\equiv[%
m_{1}n_{2}-n_{1}m_{2}:n_{1}l_{2}-l_{1}n_{2}:l_{1}m_{2}-m_{1}l_{2}].
\end{equation*}
The cross here is the usual Euclidean cross product which is well defined.
This also induces the result that \textit{the join $a_{1}a_{2}$ is a unique
line which is incident with the points $a_{1}$ and $a_{2}$}. Dually \textit{%
the meet $L_{1}L_{2}$ is a unique point which is incident with the lines $%
L_{1}$ and $L_{2}$}.

\pagebreak A \textbf{3-proportion} $x:y:z$ is an ordered triple of numbers $%
x,y$ and $z$, \textit{not all zero}, with the convention that for any
non-zero number $\lambda$ 
\begin{equation*}
x:y:z=\lambda x:\lambda y:\lambda z.
\end{equation*}
This is equivalent to saying that 
\begin{equation*}
x_{1}:y_{1}:z_{1}=x_{2}:y_{2}:z_{2}
\end{equation*}
precisely when the following conditions hold 
\begin{equation}  \label{proportionequal}
x_{1}y_{2}-x_{2}y_{1}=0\;\;y_{1}z_{2}-y_{2}z_{1}=0\;%
\;z_{1}x_{2}-z_{2}x_{1}=0.
\end{equation}

Now that the notion of a proportion is set up we can define the two main
hyperbolic objects. A \textbf{(hyperbolic) point} is a 3-proportion $a\equiv[%
x:y:z]$ enclosed in square brackets. Where a \textbf{(hyperbolic) line} is a
3-proportion $L\equiv(l:m:n)$ enclosed in round brackets.\newline

The definitions of points and lines is equivalent to that of projective
geometry, where the two types of geometry differ becomes obvious in the
notion of duality. The point $a\equiv[x,y,z]$ is \textbf{dual} to the line $%
L\equiv(l:m:n)$ precisely when 
\begin{equation*}
x:y:z=l:m:n.
\end{equation*}
In this case we say that $a^{\perp}=L$ or $L^{\perp}=a$.\newline
From the definition of points and lines we get that each point is dual to
exactly one line, and conversely. This new idea of duality induces the same
property that \textit{there is a complete duality in the theory between
points and lines}, within this new projective geometry.\newline

Now that we have set up the basic objects of hyperbolic geometry, an
important step is to define the incidence of these objects with each other.
The following theorems and definitions aim to do exactly that.\newline

The point $a\equiv[x:y:z]$ \textbf{lies on} the line $L\equiv(l:m:n)$, or
equivalently $L$ \textbf{passes through} $a$, precisely when 
\begin{equation*}
lx+my-nz=0.
\end{equation*}%
\newline

Points $a_{1}\equiv[x_{1}:y_{1}:z_{1}]$ and $a_{2}\equiv[x_{2}:y_{2}:z_{2}]$
are \textbf{perpendicular} precisely when 
\begin{equation*}
x_{1}x_{2}+y_{1}y_{2}-z_{1}z_{2}=0.
\end{equation*}
This is equivalent to the condition that $a_{1}$ is incident with $%
a_{2}^{\perp}$, or that $a_{2}$ is incident with $a_{1}^{\perp}$.\newline

Similarly the line $L_{1}\equiv(l_{1}:m_{1}:n_{1})$ and $L_{2}%
\equiv(l_{2}:m_{2}:n_{2})$ are \textbf{perpendicular} precisely when 
\begin{equation*}
l_{1}l_{2}+m_{1}m_{2}-n_{1}n_{2}=0.
\end{equation*}
This is equivalent to the condition that $L_{1}$ is incident with $%
L_{2}^{\perp}$, or that $L_{2}$ is incident with $L_{1}^{\perp}$.\newline

We will denote by $\mathbb{F}^{3}$ the 3-dimensional space of \textbf{vectors%
} $v\equiv(x,y,z)$. If $v\equiv(x,y,z)$ has coordinates which are not all
zero, then let $[v]\equiv[x:y:z]$ denote the (hyperbolic) point, and $%
(v)\equiv(l:m:n)$ denote the (hyperbolic) line.\newline

\begin{theorem}[Joins of points]
If $a_{1}\equiv[x_{1}:y_{1}:z_{1}]$ and $a_{2}\equiv[x_{2}:y_{2}:z_{2}]$ are
distinct points, then there is exactly one line $L$ which passes through
them both, namely 
\begin{equation*}
L\equiv
a_{1}a_{2}%
\equiv(y_{1}z_{2}-y_{2}z_{1}:z_{1}x_{2}-z_{2}x_{1}:x_{2}y_{1}-x_{1}y_{2}).
\end{equation*}
\end{theorem}

The line $L\equiv a_{1}a_{2}$ is the \textbf{join} of the points $a_{1}$ and 
$a_{2}$.\newline

\begin{theorem}[Meets of lines]
If $L_{1}\equiv(l_{1}:m_{1}:n_{1})$ and $L_{2}\equiv(l_{2}:m_{2}:n_{2})$ are
distinct lines, then there is exactly one point $a$ which lies on both,
namely 
\begin{equation*}
a\equiv L_{1}L_{2}\equiv[%
m_{1}n_{2}-m_{2}n_{1}:n_{1}l_{2}-n_{2}l_{1}:l_{2}m_{1}-l_{1}m_{2}].
\end{equation*}
\end{theorem}

The point $a\equiv L_{1}L_{2}$ is the \textbf{meet} of the lines $L_{1}$ and 
$L_{2}$. These definitions give the following consequence, for any distinct
points $a_{1}$ and $a_{2}$, and distinct lines $L_{1}$ and $L_{2}$, 
\begin{equation*}
(a_{1}a_{2})^{\perp}=a_{1}^{\perp}a_{2}\perp,\;\;\text{and}%
\;\;(L_{1}L_{2})^{\perp}=L_{1}^{\perp}L_{2}^{\perp}.
\end{equation*}

Three or more points which lie on a common line are \textbf{collinear}.
Three or more lines which pass through a common point are \textbf{concurrent}%
.\newline

\begin{theorem}[Collinear points]
The points $a_{1}\equiv[x_{1}:y_{1}:z_{1}],\; a_{2}\equiv[x_{2}:y_{2}:z_{2}]$
and $a_{3}\equiv[x_{3}:y_{3}:z_{3}]$ are collinear precisely when 
\begin{equation*}
x_{1}y_{2}z_{3}-x_{1}y_{3}z_{2}+x_{2}y_{3}z_{1}-x_{2}y_{1}z_{3}+x_{3}y_{1}z_{2}-x_{3}y_{2}z_{1}=0.
\end{equation*}
\end{theorem}

\begin{theorem}[Concurrent lines]
The lines $L_{1}\equiv(l_{1}:m_{1}:n_{1}),\; L_{2}\equiv(l_{2}:m_{2}:n_{2})$
and $L_{3}\equiv(l_{3}:m_{3}:n_{3})$ are concurrent precisely when 
\begin{equation*}
l_{1}m_{2}n_{3}-l_{1}m_{3}n_{2}+l_{2}m_{3}n_{1}-l_{2}m_{1}n_{3}+l_{3}m_{1}n_{2}-l_{3}m_{2}n_{1}=0.
\end{equation*}
\end{theorem}

Now that the fundamental objects and there interactions are defined will
begin to classify these objects. Firstly we say that the point $a\equiv[x:y:z%
]$ is \textbf{null} precisely when it lies on its dual line, that is when 
\begin{equation*}
x^{2}+y^{2}-z^{2}=0,
\end{equation*}
and analogously that the line $L\equiv(l:m:n)$ is \textbf{null} precisely
when it passes through its dual point, that is when 
\begin{equation*}
l^{2}+m^{2}-n^{2}=0.
\end{equation*}
The dual of a null point is a null line and conversely. I'll now use the
following theorem to further this classification of objects.\newline

\begin{theorem}[Line through null points, and Point on null lines]
Any line $L$ passes through at most two null points, and any point $a$ lies
on at most two null lines.
\end{theorem}

Therefore we have a naturally classification of points and lines. For a
non-null point $a$ we say that it is \textbf{internal} precisely when it
lies on no null lines, and is \textbf{external} precisely when it lies on 2
null lines. Whereas a non-null line $L$ is said to be \textbf{external}
precisely when it passes through no null points and \textbf{internal}
precisely when it passes through no null points. That is all points and
lines are either \textit{internal, null} or \textit{external}.\newline
Unlike null points and lines we have that the dual of an internal point is
an external line, and the dual of an external point is an internal line and
conversely.

\pagebreak

We now go onto define the geometric objects of hyperbolic geometry.\newline

A \textbf{side} $\overline{a_{1}a_{2}}$ is a set $\{a_{1},a_{2}\}$ of two
points. A \textbf{vertex} $\overline{L_{1}L_{2}}$ is a set $\{L_{1},L_{2}\}$
of two lines. From the definition it is clear that 
\begin{equation*}
\overline{a_{1}a_{2}}=\overline{a_{2}a_{1}}\;\; \text{and}\;\;\overline{%
L_{1}L_{2}}=\overline{L_{2}L_{1}}.
\end{equation*}
For a side $\overline{a_{1}a_{2}}$ we say that $a_{1}a_{2}$ is the \textbf{%
line} of the side. Whilst for a vertex $\overline{L_{1}L_{2}}$ we say that $%
L_{1}L_{2}$ is the \textbf{point} of the vertex.\newline

Much like the fundamental objects, we can continue to classify these new
objects. We say that a side $\overline{a_{1}a_{2}}$ is a \textbf{nil side}
precisely when at least one of $a_{1}$ or $a_{2}$ is a null point. Thus we
are able to further classify sides, such as the side $\overline{a_{1}a_{2}}$
as a \textbf{singly-nil side}, or a \textbf{doubly-nil side} respectively,
precisely when exactly one of the points $a_{1}$ or $a_{2}$ are null, or
exactly both of the points $a_{1}$ and $a_{2}$ are null points respectively.
Similarly we are able to classify the vertex $\overline{L_{1}L_{2}}$ as a 
\textbf{singly-nil vertex} or a \textbf{doubly-nil vertex} in a natural way.%
\newline

\begin{theorem}[Perpendicular point]
For any side $\overline{a_{1}a_{2}}$ there is a unique point $p$ which is
perpendicular to both $a_{1}$ and $a_{2}$, namely 
\begin{equation*}
p\equiv a_{1}^{\perp}a_{2}^{\perp}=(a_{1}a_{2})^{\perp}.
\end{equation*}
\end{theorem}

The point $p$ is the \textbf{perpendicular point} of $\overline{a_{1}a_{2}}$%
. It is possible that $p$ may lie on $a_{1}a_{2}$; this occurs precisely
when $a_{1}a_{2}$ is a null line.\newline

\begin{theorem}[Perpendicular line]
For any vertex $\overline{L_{1}L_{2}}$ there is a unique line $P$ which is
perpendicular to both $L_{1}$ and $L_{2}$, namely 
\begin{equation*}
P\equiv L_{1}^{\perp}L_{2}^{\perp}=(L_{1}L_{2})^{\perp}.
\end{equation*}
\end{theorem}

The line $P$ is the \textbf{perpendicular line} of $\overline{L_{1}L_{2}}$.
It also may happen that $P$ passes through $L_{1}L_{2}$, which occurs
precisely when $L_{1}L_{2}$ is a null point.\newline

\pagebreak

As we are in a projective setting, the definition of a (hyperbolic)
quadrangle (quadrilateral respectively,) will come from the projective
definition of a complete quadrangle (quadrilateral respectively.)\newline

A \textbf{quadrangle} $\overline{a_{1}a_{2}a_{3}a_{4}}$ is a set $%
\{a_{1},a_{2},a_{3},a_{4}\}$ of points which has the property that no three
are collinear. A \textbf{quadrilateral} $\overline{L_{1}L_{2}L_{3}L_{4}}$ is
a set $\{L_{1},L_{2},L_{3},L_{4}\}$ of lines which has the property that no
three are concurrent.\newline

The quadrangle $\square\equiv \overline{a_{1}a_{2}a_{3}a_{4}}$ has a \textbf{%
dual quadrilateral} $\square^{\perp}\equiv\overline{a_{1}^{\perp}a_{2}^{%
\perp}a_{3}^{\perp}a_{4}^{\perp}}$ consisting of four \textbf{dual lines} of
the quadrangle, namely $a_{1}^{\perp},a_{2}^{\perp},a_{3}^{\perp}$ and $%
a_{4}^{\perp}$.\newline

The quadrilateral $\lozenge\equiv \overline{L_{1}L_{2}L_{3}L_{4}}$ has a 
\textbf{dual quadrangle} $\lozenge^{\perp}\equiv\overline{%
L_{1}^{\perp}L_{2}^{\perp}L_{3}^{\perp}L_{4}^{\perp}}$ consisting of four 
\textbf{dual points} of the quadrilateral, namely $L_{1}^{\perp},L_{2}^{%
\perp},L_{3}^{\perp}$ and $L_{4}^{\perp}$.\newline

There are 6 distinct sides of a quadrangle, namely $\overline{a_{1}a_{2}},%
\overline{a_{3}a_{4}},$ $\overline{a_{1}a_{3}},\overline{a_{2}a_{4}},%
\overline{a_{1}a_{4}}$ and $\overline{a_{2}a_{3}}$. We can naturally divide
these 6 sides into 3 pairs, $\{\overline{a_{1}a_{2}},\overline{a_{3}a_{4}}\}$%
, $\{\overline{a_{1}a_{3}},\overline{a_{2}a_{4}}\}$, and $\{\overline{%
a_{1}a_{4}},\overline{a_{2}a_{3}}\}$. The intersection of these pairs of
sides give three new points called the \textbf{diagonal points} of the
quadrangle.\newline
Similarly there are 6 distinct vertices of a quadrilateral, namely $%
\overline{L_{1}L_{2}},\overline{L_{3}L_{4}},$ $\overline{L_{1}L_{3}},%
\overline{L_{2}L_{4}},\overline{L_{1}L_{4}}$and $\overline{L_{2}L_{3}}$.
These too have a natural divide into 3 pairs, $\{\overline{L_{1}L_{2}},%
\overline{L_{3}L_{4}}\}$, $\{\overline{L_{1}L_{3}},\overline{L_{2}L_{4}}\}$,
and $\{\overline{L_{1}L_{4}},\overline{L_{2}L_{3}}\}$. The join of these
pairs of vertices give three new lines called the \textbf{diagonal lines} of
the quadrilateral.\newline

\begin{theorem}[Diagonal triangle]
The diagonal points $d_{1}\equiv(a_{1}a_{4})(a_{2}a_{3}),$\newline
$d_{2}\equiv(a_{2}a_{4})(a_{1}a_{3})$ and $d_{3}%
\equiv(a_{3}a_{4})(a_{1}a_{2})$ of the quadrangle $\square\equiv\overline{%
a_{1}a_{2}a_{3}a_{4}}$ form the triangle $\triangle\equiv\overline{%
d_{1}d_{2}d_{3}}$.
\end{theorem}

\begin{proof}
\begin{itemize}
\item Use join of points and meets of lines to write out each $d_{i}$.
\item Condition for $d_{i}$s to be collinear
\item $a_{i}$s non collinear follows $d_{i}$s non collinear.
\end{itemize}
\end{proof}

\begin{theorem}[Diagonal trilateral]
The diagonal lines $D_{1}\equiv(L_{1}L_{4})(L_{2}L_{3}),$\newline
$D_{2}\equiv(L_{2}L_{4})(L_{1}L_{3})$ and $D_{3}%
\equiv(L_{3}L_{4})(L_{1}L_{2})$ of the quadrilateral $\lozenge\equiv%
\overline{L_{1}L_{2}L_{3}L_{4}}$ form the trilateral $\triangledown\equiv%
\overline{D_{1}D_{2}D_{3}}$.
\end{theorem}

\begin{proof}
Dual to the previous theorem.
\end{proof}

\pagebreak

\section{\protect\bigskip Geogebra Tools}

Tools that I've made to be used in GeoGebra for use in exploring Universal
Hyperbolic geometry. Firstly I will present a list of all the tool that I
have created within GeoGebra (using it's tool creating system), and then I
will present how I created them.\newline

\begin{itemize}
\item Polar Line (Polar)

\item Pole Point (Pole)

\item Reflections (Reflect\_PiP, Reflect\_PiL, Reflect\_LiP, Reflect\_LiL)

\item Midpoints

\item Sydpoint

\item Smydpoint

\item Diagonal Triangle (DiagTri)
\end{itemize}

\textbf{Polar Line}\newline
For the \textbf{Polar} tool I used the pre-installed \textit{Polar or
Diameter Line} tool in GeoGebra to create the Polar Line to a Point, by
selecting the given point and then the absolute conic, \textit{Polar($l,c$)}.%
\newline
\definecolor{qqqqff}{rgb}{0,0,1} %
\definecolor{dcrutc}{rgb}{0.86275,0.07843,0.23529} 
\begin{tikzpicture}[line cap=round,line join=round,>=triangle 45,x=1.0cm,y=1.0cm]
\clip(-1.69121,-1.11787) rectangle (2.6406,1.12308);
\draw [color=dcrutc] (0,0) circle (1cm);
\draw [domain=-1.69121:2.6406] plot(\x,{(--1--1.12191*\x)/-0.8384});
\begin{scriptsize}
\fill [color=qqqqff] (-1.12191,-0.8384) circle (1.5pt);
\draw[color=qqqqff] (-1.40121,-0.80476) node {$A$};
\end{scriptsize}
\end{tikzpicture}

\textbf{Pole Point}\newline
The \textbf{Pole} tool was created by once again using the pre-installed 
\textit{Polar or Diameter Line} tool on GeoGebra, and is used for producing
a Pole point to a line. Given a line $L$ and the absolute conic $c$, using 
\textit{Polar or Diameter Line} on $L$ and $c$ produces a new line $L_{1}$
which is perpendicular (in the UHG sense) to $L$ and a diameter of $c$. The 
\textit{Intesection of two Objects} tool on $L$ and $L_{1}$ gives the point $%
l_{1}=(LL_{1})$. Now using \textit{Polar} on $l_{1}$ gives the line $L_{2}$
perpendicular to $L_{1}$ and parallel to $L$. The point $l=(L_{1}L_{2})=L^{%
\perp}$ is the pole of $L$.\newline
\textit{Pole} is used by selecting the given line $L$ and then the absolute
conic $c$, \textit{Pole($L,c$)}.\newline

\textbf{Reflections}\newline
The \textbf{Reflect\_PiP} is a tool to used to find the reflection of a
point in a point. Given a reflecting point $a$, reflecter point $b$ and
absolute $c$. First create the line $(ab)$ through the \textit{Line Between
two Points} tool. Then use \textit{Polar or Diameter Line} with $(ab)$ and $%
c $ to create $D_{1}$. If $D_{1}$ intersects $c$ (then it will intersect $c$
twice), we are able to construct a cycle quadrangle from one of the
intersection points and the points $a$ and $b$. We do this by choosing one
of the intersection points, call it $i_{1}$ and then join it to both $a$ and 
$b$, giving the lines $I_{2}$ and $I_{1}$ respectfully. The lines $I_{2}$
and $I_{1}$ will then intersect $c$ once more, say at $i_{2}$ and $i_{4}$
respectfully. If we let $I_{3}=i_{2}b$ then we get another null point $%
i_{3}\in \{I_{3}c\}$. These four points give us the cyclic quadrangle $%
\overline{i_{1}i_{2}i_{3}i_{4}}$ whose diagonal points either lie on $%
B=b^{\perp }$ or are $b$ itself. Finally the reflection of $a$ in $b$ is $%
a_{1}^{\prime }=I_{4}(ab)=(i_{3}i_{4})(ab)$.\newline
If $D_{1}$ does not intersect $c$ (possible if $c$ is a hyperbola) then we
use \textit{Polar or Diameter Line} with $D_{1}$ and $c$ to create another
diameter $D_{2}$ which is perpendicular to $D_{1}$. We can then use the same
constuction as above to create $a_{2}^{\prime }$.\newline
If both $D_{1}$ and $D_{2}$ intersect $c$ then clearly $a_{1}^{\prime
}=a_{2}^{\prime }$. Thus to cover all cases and ensure that the tool
produces one point we use the inbuilt \textit{If, Then Logic} tool as
follows, if $a_{1}^{\prime }$ is defined then $a^{\prime }=a_{1}^{\prime }$
otherwise $a^{\prime }=a_{2}$.\newline
Since the reflection in the point $a$ is equivalent to the reflection in the
line $A=a^{\perp }$ the rest of the tools \textbf{Reflect\_PiL, Reflect\_LiP}
and \textbf{Reflect\_LiL} were created using the \textit{Polar, Pole} and 
\textit{Reflect\_PiP} tools.

\section{Address later}

\bigskip 
\begin{align*}
M_{12}& \equiv \left\langle \left( a+d+f\right) \sigma _{\left\{ 1,3\right\}
}+\left( -a-d+f\right) \sigma _{\left\{ 2,3\right\} }:(d+b+g)\sigma
_{\left\{ 1,3\right\} }+\left( -d-b+g\right) \sigma _{\left\{ 2,3\right\}
}:(f+g+c)\sigma _{\left\{ 1,3\right\} }+\left( -f-g+c\right) \sigma
_{\left\{ 2,3\right\} }\right\rangle ,\;\;\;\;\; \\
M_{21}& \equiv \left\langle (a+d+f)\sigma _{\left\{ 1,4\right\} }+\left(
a+d-f\right) \sigma _{\left\{ 2,4\right\} }:(d+b+g)\sigma _{\left\{
1,4\right\} }+\left( d+b-g\right) \sigma _{\left\{ 2,4\right\}
}:(f+g+c)\sigma _{\left\{ 1,4\right\} }+\left( f+g-c\right) \sigma _{\left\{
2,4\right\} }\right\rangle , \\
M_{34}& \equiv \left\langle (a-d+f)\sigma _{\left\{ 1,3\right\} }+\left(
-a+d+f\right) \sigma _{\left\{ 1,4\right\} }:(d-b+g)\sigma _{\left\{
1,3\right\} }+\left( -d+b+g\right) \sigma _{\left\{ 1,4\right\}
}:(f-g+c)\sigma _{\left\{ 1,3\right\} }+\left( -f+g+c\right) \sigma
_{\left\{ 1,4\right\} }\right\rangle , \\
M_{43}& \equiv \left\langle (a-d+f)\sigma _{\left\{ 2,3\right\} }+\left(
a-d-f\right) \sigma _{\left\{ 2,4\right\} }:(d-b+g)\sigma _{\left\{
2,3\right\} }+\left( d-b-g\right) \sigma _{\left\{ 2,4\right\}
}:(f-g+c)\sigma _{\left\{ 2,3\right\} }+\left( f-g-c\right) \sigma _{\left\{
2,4\right\} }\right\rangle , \\
M_{13}& \equiv \left\langle a+d+f+\left( a-d+f\right) \sigma _{\left\{
1,3\right\} }A_{1}:d+b+g+\left( d-b+g\right) \sigma _{\left\{ 1,3\right\}
}A_{1}:f+g+c+\left( f-g+c\right) \sigma _{\left\{ 1,3\right\}
}A_{1}\right\rangle , \\
M_{31}& \equiv \left\langle a+d+f+\left( -a+d-f\right) \sigma _{\left\{
1,3\right\} }A_{1}:d+b+g+\left( -d+b-g\right) \sigma _{\left\{ 1,3\right\}
}A_{1}:f+g+c+\left( -f+g-c\right) \sigma _{\left\{ 1,3\right\}
}A_{1}\right\rangle ,
\end{align*}

\begin{align*}
M_{24}& \equiv \left\langle -a-d+f+\left( -a+d+f\right) \sigma _{\left\{
2,4\right\} }A_{2}:-d-b+g+\left( -d+b+g\right) \sigma _{\left\{ 2,4\right\}
}A_{2}:-f-g+c+\left( -f+g+c\right) \sigma _{\left\{ 2,4\right\}
}A_{2}\right\rangle , \\
M_{42}& \equiv \left\langle -a-d+f+\left( a-d-f\right) \sigma _{\left\{
2,4\right\} }A_{2}:-d-b+g+\left( d-b-g\right) \sigma _{\left\{ 2,4\right\}
}A_{2}:-f-g+c+\left( f-g-c\right) \sigma _{\left\{ 2,4\right\}
}A_{2}\right\rangle , \\
M_{14}& \equiv \left\langle a+d+f+\left( -a+d+f\right) \sigma _{\left\{
1,4\right\} }A_{1}:d+b+g+\left( -d+b+g\right) \sigma _{\left\{ 1,4\right\}
}A_{1}:f+g+c+\left( -f+g+c\right) \sigma _{\left\{ 1,4\right\}
}A_{1}\right\rangle , \\
M_{41}& \equiv \left\langle a+d+f+\left( a-d-f\right) \sigma _{\left\{
1,4\right\} }A_{1}:d+b+g+\left( d-b-g\right) \sigma _{\left\{ 1,4\right\}
}A_{1}:f+g+c+\left( f-g-c\right) \sigma _{\left\{ 1,4\right\}
}A_{1}\right\rangle , \\
M_{23}& \equiv \left\langle -a-d+f+\left( a-d+f\right) \sigma _{\left\{
2,3\right\} }A_{2}:-d-b+g+\left( d-b+g\right) \sigma _{\left\{ 2,3\right\}
}A_{2}:-f-g+c+\left( f-g+c\right) \sigma _{\left\{ 2,3\right\}
}A_{2}\right\rangle , \\
M_{32}& \equiv \left\langle -a-d+f+\left( -a+d-f\right) \sigma _{\left\{
2,3\right\} }A_{2}:-d-b+g+\left( -d+b-g\right) \sigma _{\left\{ 2,3\right\}
}A_{2}:-f-g+c+\left( -f+g-c\right) \sigma _{\left\{ 2,3\right\}
}A_{2}\right\rangle ,
\end{align*}

these are wrong at the moment, not sure if it is crucial for me to put them
in anyway

spreads 
\begin{align*}
S(a_{1}a_{2},a_{1}a_{3})& =\frac{aD}{(ab-d^{2})(ac-e^{2})}, \\
S(a_{1}a_{2},a_{1}a_{4})& =\frac{aD}{(ab-d^{2})(a(b+c+2f)-(d+e)^{2})}, \\
S(a_{1}a_{3},a_{1}a_{4})& =\frac{aD}{(ac-e^{2})(a(b+c+2f)-(d+e)^{2})}, \\
S(a_{1}a_{2},a_{2}a_{3})& =\frac{bD}{(ab-d^{2})(bc-f^{2})}, \\
S(a_{1}a_{2},a_{2}a_{4})& =\frac{bD}{(ab-d^{2})(b(a+c+2e)-(d+f)^{2})}, \\
S(a_{2}a_{3},a_{2}a_{3})& =\frac{bD}{(bc-f^{2})(b(a+c+2e)-(d+f)^{2})}, \\
S(a_{1}a_{3},a_{2}a_{3})& =\frac{cD}{(ac-e^{2})(bc-f^{2})}, \\
S(a_{1}a_{3},a_{3}a_{4})& =\frac{cD}{(ac-e^{2})(c(a+b+2d)-(e+f)^{2})}, \\
S(a_{2}a_{3},a_{3}a_{4})& =\frac{cD}{(bc-f^{2})(c(a+b+2d)-(e+f)^{2})}, \\
S(a_{1}a_{4},a_{2}a_{4})& =\frac{(a+b+c+2(e+d+f))D}{%
(a(b+c+2f)-(d+e)^{2})(b(a+c+2e)-(d+f)^{2})}, \\
S(a_{1}a_{4},a_{3}a_{4})& =\frac{(a+b+c+2(e+d+f))D}{%
(c(a+b+2d)-(e+f)^{2})(b(a+c+2e)-(d+f)^{2})}, \\
S(a_{2}a_{4},a_{3}a_{4})& =\frac{(a+b+c+2(e+d+f))D}{%
(a(b+c+2f)-(d+e)^{2})(c(a+b+2d)-(e+f)^{2})}.
\end{align*}

\begin{align*}
1-S(a_{1}a_{2},a_{1}a_{3})& =\frac{(af-de)^{2}}{(ab-d^{2})(ac-e^{2})}, \\
1-S(a_{1}a_{2},a_{1}a_{4})& =\frac{(af-de)^{2}}{%
(ab-d^{2})(a(b+c+2f)-(d+e)^{2})}, \\
1-S(a_{1}a_{3},a_{1}a_{4})& =\frac{(af-de)^{2}}{%
(ac-e^{2})(a(b+c+2f)-(d+e)^{2})}, \\
1-S(a_{1}a_{2},a_{2}a_{3})& =\frac{(be-df)^{2}}{(ab-d^{2})(bc-f^{2})}, \\
1-S(a_{1}a_{2},a_{2}a_{4})& =\frac{(be-df)^{2}}{%
(ab-d^{2})(b(a+c+2e)-(d+f)^{2})}, \\
1-S(a_{2}a_{3},a_{2}a_{3})& =\frac{(be-df)^{2}}{%
(bc-f^{2})(b(a+c+2e)-(d+f)^{2})}, \\
1-S(a_{1}a_{3},a_{2}a_{3})& =\frac{(cd-ef)^{2}}{(ac-e^{2})(bc-f^{2})}, \\
1-S(a_{1}a_{3},a_{3}a_{4})& =\frac{(cd-ef)^{2}}{%
(ac-e^{2})(c(a+b+2d)-(e+f)^{2})}, \\
1-S(a_{2}a_{3},a_{3}a_{4})& =\frac{(cd-ef)^{2}}{%
(bc-f^{2})(c(a+b+2d)-(e+f)^{2})}, \\
1-S(a_{1}a_{4},a_{2}a_{4})& =\frac{(ab-d^{2}+af+be-cd-de-df+ef)^{2}}{%
(a(b+c+2f)-(d+e)^{2})(b(a+c+2e)-(d+f)^{2})}, \\
1-S(a_{1}a_{4},a_{3}a_{4})& =\frac{(ac-e^{2}+af-be+cd-de+df-ef)^{2}}{%
(c(a+b+2d)-(e+f)^{2})(b(a+c+2e)-(d+f)^{2})}, \\
1-S(a_{2}a_{4},a_{3}a_{4})& =\frac{(bc-f^{2}-af+be+cd+de-df-ef)^{2}}{%
(a(b+c+2f)-(d+e)^{2})(c(a+b+2d)-(e+f)^{2})}.
\end{align*}

\end{document}
